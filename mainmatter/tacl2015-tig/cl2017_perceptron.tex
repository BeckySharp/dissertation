\section{Learning Model}
\label{sec-cl2017:perceptron}

We perform answer selection by focusing on ranking answer {\em justifications} rather then actual answers. 
Our algorithm (detailed in Algorithm \ref{alg:perceptron}) determines the best answer to a given question by first finding the highest-scoring TAG (i.e., justification) out of all the TAGs for each of its candidate answers, and then selecting the corresponding answer candidate.  
This introduces an additional complication during training: while we know which answer candidate is correct, the actual quality of any given TAG is hidden. That is, many TAGs which connect a correct answer to the question do so in the wrong context, producing a poor justification. Conversely, even an incorrect answer can be connected back to the question, otherwise the multiple choice test itself would be too easy. 

We address this issue with a reranking perceptron~\cite{Shen:Joshi:2005,Surdeanu:11} extended with a latent layer that models the correctness of the justifications. 
During training, we take advantage of the assumption that, with enough knowledge coverage, the best TAG for a correct answer will be better than the best TAG for an incorrect one.  As a result, our algorithm optimizes two goals jointly: choosing a good answer for a given question, and justifying it with a human-readable TAG that connects sentences from our knowledge bases.

\subsection {Learning Algorithm}
\label{sec-cl2017:model}

The input to the reranking algorithm consists of a corpus of $n$ training questions, $\boldsymbol{\mathcal{Q}}$, where each $q_i \in \boldsymbol{\mathcal{Q}}$ has a set of $m$ candidate answers, $\boldsymbol{A}$ (in the particular case of the multiple choice test, these are the four multiple choice answers).  Each $a_j \in \boldsymbol{A}$ for a given $q_i$ has a set of TAGs, $\boldsymbol{X_{i,j}}$, which connect $q_i$ to $a_j$.  
Crucially, for a correct answer, we assume that there is {\em at least one} valid TAG $x_c \in \boldsymbol{X_{i,j}}$, which justifies the correctness of the chosen answer. This assumption is similar to the one made by Hoffmann et al.~\citeyear{hoffmann2011knowledge}, who assumed that, for a binary relation extraction task, for each training relation between two entities, $e_1$ and $e_2$, there exists at least one correct sentence that supports the extraction of the given relation among the sentences that contain both $e_1$ and $e_2$.

The learning algorithm is formalized in Algorithm~\ref{alg:perceptron}. The two fundamental building blocks of the algorithm are the functions $P$ and $F$. 
Intuitively, the function $P$ identifies which justifications (or TAGs) for a given answer best explain the answer according to the current model. For example, for the question {\em Which organism is a producer?} and the answer {\em grass}, a good justification selected by function $P$ is a TAG that connects the two sentences: {\em Producer is an organism that produces its own food and is food for other organisms: usually a green plant.} and {\em Grass is a green, leafy plant that often covers the ground.} 
Because $P$ implements a latent operation (i.e., which justification is correct?), its goodness evolves together with the learned model. 
Note that, at this stage, we do not control how many justifications are produced by function P for a given answer other than constraining it to select {\em at least one}. The function $F$ computes the overall score of a given answer by simply averaging the scores of the justifications chosen by $P$. 

More formally, for a given question $q_i$ and candidate answer $a_j$, $P$ chooses a subset of TAGs from $\boldsymbol{X_{i,j}}$ that are deemed correct, according to the current model. $F$ marginalizes over these TAGs to compute the overall score of the answer $a_j$:

\begin{equation}
\label{eq:F}
F(q_i, a_j) = \frac{\sum_{x \in P(q_i, a_j)} \boldsymbol{\Theta} \cdot \Phi(x)}{|P(q_i, a_j)|}
\end{equation}
where the score of a given TAG, $x$, is computed as the dot product between the parameter vector $\boldsymbol{\Theta}$  and the vector produced by a feature function, $\Phi$, which operates over individual TAGs. 

Given the functions $P$ and $F$ that control the latent operation of choosing valid answer justifications, the algorithm proceeds similarly to the reranking perceptron~\cite{Shen:Joshi:2005}. If a prediction, i.e., a choice for the correct answer for a given question (line 6), is incorrect (line 7), the algorithm updates its parameter vector, $\boldsymbol{\Theta}$, by adding all valid justifications of the correct answer as produced by $P$ (line 9), and subtracting all TAGs considered valid for the current top answer (line 12). 


The fundamental operation here is thus the implementation of $P$. For their classification task, Hoffman et al. implement this function as an edge-cover problem, which guarantees that the overall score for the correct label (similar to our $F$ function) is maximized and at least one sentence supports the correct classification.  
For our reranking task, we found that a conservative interpretation of this {\em at least one} idea, where {\em exactly} one justification is chosen, performed the best.\footnote{In fact, we found that performance consistently dropped as the number of justifications chosen by $P$ increased.} That is, $P$ returns only the top TAG with the highest score according to the dot product with the current $\boldsymbol{\Theta}$. 
We discuss the other extreme, i.e., using all TAGs, in Section \ref{sec-cl2017:controls} (when taken to this level, the latent layer is essentially disregarded entirely, assuming that each TAG for a correct answer is correct, and that each TAG for an incorrect answer is incorrect).  As discussed there, this performed far worse.


%\begin{algorithm}[t]
%%\SetAlgoLined
%%\linesnumbered
%\LinesNumbered
%\KwIn{  $ $\\
%$T$ -- the number of training epochs\;
%$\boldsymbol{\Sigma}$ -- the set of $n$ training questions, each of which has $m$ answer candidates\;
%$\boldsymbol{X_{i,j}}$ -- the set of all paths connecting a question $q_i$ with a candidate answer $a_j$\;}
%\KwOut{ \\
%$\boldsymbol{\Theta}$ -- parameter vector\;}
%\Begin{
%% {\small \tcp{Initialize parameter vector}}
%$\boldsymbol{\Theta} = \boldsymbol{0}$\;
%\For{$t\leftarrow 1$ \KwTo $T$}{
%	\For{$i\leftarrow 1$ \KwTo $n$} {
%			$k = \argmax_{j = 1, \dots, m} F(q_i, a_j)$\; 
%			\If{$k \ne 1$}{
%				\For{$x \leftarrow P(q_i, a_1)$}{
%					$\boldsymbol{\Theta} = \boldsymbol{\Theta} + \Phi(x)$\;
%				}
%				\For{$x \leftarrow P(q_i, a_k)$}{
%					$\boldsymbol{\Theta} = \boldsymbol{\Theta} - \Phi(x)$\;
%				}
%			}
%%			compute $ y_1 = F(a_1, X_1)$\\
%%			\For{$j\leftarrow 2$ \KwTo $m$} {
%%				%$z_{j}^{*}$ = $ argmax_{\bar{z_{j}}}$($\theta \cdot \bar{z_{j}}$)\; 
%%				compute $ y_j = F(a_j, X_j)  $\; 
%%				}
%%		find $p = argmax_m(Y) $\\
%%		\If{$ (y_1 - y_{p}) < \lambda $}{
%%			$ \bar{\theta} = \bar{\theta} + (\Phi(X_1) - \Phi(X_{p})) \tau $ \;
%%		}
%}
%}
%}
%\Return{$\boldsymbol{\Theta}$}
%\vspace{5mm}
%\caption{{\footnotesize Learning algorithm for the latent reranking perceptron. We consider, without loss of generality, that the correct answer appears at position 1 in training.}}
%\label{alg:perceptron}
%\end{algorithm}
\begin{algorithm}[t]                      % enter the algorithm environment
                 % and a label for \ref{} commands later in the document
                 \small
\begin{algorithmic}[1]                    % enter the algorithmic environment
    \STATE \textbf{Input:}{$ $\\
$T$ -- the number of training epochs}\\
$\boldsymbol{\mathcal{Q}}$ -- the set of $n$ training questions, each of which has $m$ answer candidates\\
$\boldsymbol{X_{i,j}}$ -- the set of all TAGs connecting a question $q_i$ with a candidate answer $a_j$\\
    \STATE \textbf{Output:}{ $\boldsymbol{\Theta}$ -- parameter vector\\}
    \STATE $\boldsymbol{\Theta} = \boldsymbol{0}$\;
	\FOR{$t\leftarrow 1$ --to-- $T$}
		\FOR{$i\leftarrow 1$ --to-- $n$} 
			\STATE $k = \argmax_{j = 1, \dots, m} F(q_i, a_j)$
			\IF{$k \ne 1$}
				\FOR{$x \leftarrow P(q_i, a_1)$}
					\STATE $\boldsymbol{\Theta} = \boldsymbol{\Theta} + \Phi(x)$
				\ENDFOR
				\FOR{$x \leftarrow P(q_i, a_k)$}
					\STATE $\boldsymbol{\Theta} = \boldsymbol{\Theta} - \Phi(x)$
				\ENDFOR
			\ENDIF
		\ENDFOR
	\ENDFOR
	\RETURN $\boldsymbol{\Theta}$
\end{algorithmic}
\caption{ Learning algorithm for the latent reranking perceptron. We consider, without loss of generality, that the correct answer appears at position 1 in training.}   
\label{alg:perceptron}
\end{algorithm}



{\flushleft{\bf Inference:}}
%\paragraph{Inference}
During inference, the algorithm ranks all candidate answers for a given question in descending order of their score (as given by $F$, but using the averaged parameter vector, $\boldsymbol{\Theta_{avg}}$) and returns the top answer.

{\flushleft{\bf Practical concerns:}}
Two practical issues were omitted in Algorithm~\ref{alg:perceptron} for clarity, but improved performance in practice. First, we used the averaged percetron at inference time~\cite{Collins:2002:DTM}. That is, instead of using the latest $\boldsymbol{\Theta}$ after training, we averaged all parameter vectors produced during training, and used the averaged vector, $\boldsymbol{\Theta_{avg}}$, for inference.

Second, we used a large-margin perceptron, similar to~\cite{Surdeanu:11}. In particular, we update the model not only when the predicted answer is incorrect (line 5), but also when the current model is not confident {\em enough} -- that is, when the predicted answer is correct, but the difference in $F$ scores between this answer and the second predicted answer is smaller than a small positive hyper parameter $\tau$. 



