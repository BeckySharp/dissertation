%
% File acl2012.tex
%
% Contact: Maggie Li (cswjli@comp.polyu.edu.hk), Michael White (mwhite@ling.osu.edu)
%%
%% Based on the style files for ACL2008 by Joakim Nivre and Noah Smith
%% and that of ACL2010 by Jing-Shin Chang and Philipp Koehn


%\documentclass[11pt]{article}
\documentclass[fullname]{clv2}
%\usepackage{acl2012}


\usepackage{placeins}
\usepackage{times}
\usepackage{latexsym}
\usepackage{amsmath}
\usepackage{fancyvrb, fancyhdr, theorem, latexsym, color, longtable}
\usepackage{multirow}
\usepackage{url}
\usepackage{bm}
\usepackage{amssymb}

\usepackage{fixltx2e}
\usepackage{tabularx}
\usepackage{hyperref}
\usepackage{graphicx}

%\usepackage{algorithm}% http://ctan.org/pkg/algorithms
%\usepackage{algpseudocode}% http://ctan.org/pkg/algorithmicx

%\usepackage[algo2e,lined]{algorithm2e}
\usepackage{algorithm}
\usepackage{algorithmic}

\usepackage{color}
\newcommand{\todo}[1]{\textcolor{red}{TODO: #1}}
\newcommand{\note}[1]{\textcolor{red}{#1}}
\newcommand{\svmr}{{SVM$^{rank}$}}
\newcommand{\code}[1]{{\tt {\small #1}}}
\newcommand{\qn}{{{\bf Q}$^\textbf{{\small N}}$}}
\newcommand{\ssa}{{{\scriptsize $^{*}$}}}


\DeclareMathOperator*{\argmax}{arg\,max}
%\setlength\titlebox{6.5cm}    % Expanding the titlebox

\historydates{Submission received: 24th November, 2015; Revised version received: 22nd August, 2016; Accepted for publication: 28th November, 2016}

\issue{vv}{nn}{yyyy}

%\dochead{Document Head}

\runningtitle{Framing QA as Creating and Ranking Intersentence Answer Justifications}

\runningauthor{Jansen et al.}
%Kinds of inference identified by Clark et al.~\shortcite{clark:2013}. Retrieval-based questions (including \emph{is-a}, dictionary definition, and property identification questions) tend to be answerable using information retrieval methods over structured knowledge bases, including taxonomies, dictionaries, and property knowledge databases. 

\begin{document}

\title{Framing QA as Building and Ranking Intersentence Answer Justifications}

%\author{Another Author\thanks{thanks}}
\author{Peter Jansen\thanks{School of Information, University of Arizona, Tucson, AZ, 85721. E-mail: pajansen@email.arizona.edu}}
\affil{University of Arizona}
%
\author{Rebecca Sharp\thanks{Department of Linguistics, University of Arizona, Tucson, AZ, 85721.}}
\affil{University of Arizona}
%
\author{Mihai Surdeanu\thanks{Department of Computer Science, University of Arizona, Tucson, AZ, 85721.}}
\affil{University of Arizona}
%
\author{Peter Clark\thanks{Allen Institute for Artificial Intelligence, 2157 N Northlake Way Suite 110, Seattle, WA 98103}}
\affil{Allen Institute for Artificial Intelligence}

\maketitle

\begin{abstract}
We propose a question answering (QA) approach for standardized science exams that both identifies correct answers and produces compelling human-readable justifications for why those answers are correct. 
Our method first identifies the actual information need in a question using psycholinguistic concreteness norms, then uses this information need to construct answer justifications by aggregating multiple sentences from different knowledge bases using syntactic and lexical information.  We then jointly rank answers and their justifications using a reranking perceptron that treats justification quality as a latent variable.  We evaluate our method on 1,000 multiple-choice questions from elementary school science exams, and empirically demonstrate that it performs better than several strong baselines, including neural network approaches. Our best configuration answers 44\% of the questions correctly, where the top justifications for 57\% of these correct answers contain a compelling human-readable justification that explains the inference required to arrive at the correct answer.  We include a detailed characterization of the justification quality for both our method and a strong baseline, and show that information aggregation is key to addressing the information need in complex questions. 
\end{abstract}



\chapter{INTRODUCTION\label{chapter:introduction}}

text

\section{Overview\label{sec:overview}}

Some text/outline of the body of the document.
\section{Related Work}
\label{sec:relatedwork}

In one sense, QA systems can be described in terms of their position along a formality continuum ranging from shallow models that rely on information retrieval, lexical semantics, or alignment, to highly structured models based on first order logic (FOL).  

On the shallower end of the spectrum,  QA models can be constructed either from structured text, such as question--answer pairs, or unstructured text.  Alignment models~\cite{Berger:00,echihabi2003noisy,Soricut:06,Riezler:etal:2007,Surdeanu:11,yao2013}  require aligned question--answer pairs for training, a burden which often limits their practical usage (though Sharp et al.~\citeyear{sharp-EtAl:2015:NAACL-HLT} recently proposed a method for using the discourse structure of free text as a surrogate for this alignment structure).
Lexical semantic models such as neural-network language models~\cite{jansen14,sultan-etal:2014:TACL,yih13}, on the other hand, have the advantage of being readily constructed from free text.  
Fried et al.~\citeyear{fried2015higher} called these approaches first-order models because associations are explicitly learned, and introduced a higher-order lexical semantics QA model where indirect associations are detected through traversals of the association graph.  
Other recent efforts have applied deep learning architectures to QA to learn non-linear answer scoring functions that model lexical semantics~\cite{Iyyer2014,nips15_hermann}.
However, while lexical semantic approaches to QA have shown robust performance across a variety of tasks, a disadvantage of these methods is that, even when a correct answer is selected, there is no clear human-readable justification for that selection.   

Closer to the other end of the formality continuum, several approaches were proposed to not only select a correct answer, but also provide a formally valid justification for that answer.  For example, some QA systems have sought to answer questions by creating formal proofs driven by logic reasoning~\cite{moldovan2003cogex,moldovan2007cogex,balduccini2008knowledge,maccartney2009natural,liang2013learning,lewis2013combining}, answer-set programming \cite{baral2006using,baral2011towards,baral2012answering,baral2012knowledge}, or connecting semantic graphs~\cite{banarescu2012amr,sharmatowards}. 
However, the formal representations used in these systems, e.g., logic forms, are both expensive to generate 
and tend to be brittle because they rely extensively on imperfect tools such as complete syntactic analysis and word sense disambiguation.
We offer the lightly-structured sentence representation generated by our approach (see Section \ref{sec:tag}) as a shallower and consequently more robust approximation of those logical forms, and show that they are well-suited for the complexity of our questions.
Our approach allows us to robustly aggregate information from a variety of knowledge sources to create human-readable answer justifications.  
It is these justifications which are then ranked in order to choose the correct answer, using a reranking perceptron with a latent layer that models the correctness of those justifications.


Covering the middle ground between shallow and formal representations, learning to rank methods based on tree-kernels~\cite{Moschitti:04} perform well for various QA tasks, including passage reranking, answer sentence selection, or answer extraction~\cite[inter alia]{Moschitti:07,Moschitti:11,Severyn:12,Severyn:13a,Severyn:13b,Tymoshenko:15}. 
The key to tree kernels' success is their ability to automate feature engineering rather than having to rely on hand-crafted features, which allows them to explore a larger representation space. Further, tree kernels operate over structures that encode syntax and/or shallow semantics such as semantic role labeling~\cite{Severyn:12}, knowledge from structured databases~\cite{Tymoshenko:15}, and higher level semantic information such as question category and focus words~\cite{Severyn:13b}.
Here, we similarly use structural features based on syntax, and enriched with additional information about how the answer candidate, the question, and the aggregated justification relate to each other.  
A key difference between our work and methods based on tree kernels is that rather than selecting a contiguous segment of text (sentence or paragraph) our justifications are aggregated from multiple sentences, often from different documents. Because of this setup, we explore content representations that continue to use syntax, but combined with robust strategies for cross-sentence connections. Further, because our justification search space is increased considerably due to the ability to form cross-sentence justifications, we restrict our learning models to linear classifiers that learn efficiently at this scale. However, as discussed, tree kernels offer distinct advantages over linear models. We leave the adaptation of tree kernels to the problem discussed here as future work.



Information aggregation (or fusion) is broadly defined as the assembly of knowledge from different sources, and has been used in several NLP applications, including summarization and QA.  In the context of summarization, information aggregation has been used to assemble summaries from non-contiguous text fragments~\cite[inter alia]{barzilay1999information,barzilay2005sentence}, while in QA, aggregation has been used to assemble answers to both factoid questions~\cite{pradhan2002building} and definitional questions~\cite{blair2003hybrid}.  Critical to the current work, in an in-depth open-domain QA error analysis, Moldovan et al. \citeyear{Moldovan:2003:PIE:763693.763694} identified a subset of questions for which information from a single source is not sufficient, and designated a separate class within their taxonomy of QA systems for those systems which were capable of performing answer fusion. Combining multiple sources, however, creates the need for context disambiguation -- an issue we tackle through the use of question and answer focus words.

Identifying question focus words, a subtask of question decomposition and identifying information needs, was found relevant for QA (especially factoid) early on~\cite[inter alia]{Harabagiu:00,Moldovan:2003:PIE:763693.763694} mainly as a means to identify answer types (e.g., "What is the {\em capital} of France?" indicates the expected answer type is \emph{City}).  
Recently, Park and Croft~\citeyear{Park:2015} have used focus words to reduce semantic drift in query expansion, by conditioning on the focus words when expanding non-focus query words.
Similarly, here, we use focus words (from both question and answer) to reduce the interference of noise in both building and ranking answer justifications.  By identifying which words are most likely to be important for finding the answer, we are able to generate justifications that preferentially connect sentences together on these focus words.  This results in justifications that are better able to remain on-context, and as we demonstrate in Section \ref{sec:experiments}, this boosts overall performance. 

Once the candidate answer justifications are assembled, our method selects the answer which corresponds to the best (i.e., highest-scoring) justification.  We learn which justifications are indicative of a correct answer by extending ranking perceptrons~\cite{Shen:Joshi:2005}, which have been previously used in QA~\cite{Surdeanu:11}, to include a latent layer that models the correctness of the justifications. Latent-variable perceptrons have been proposed for several other NLP tasks~\cite{liang2006end,zettlemoyer2007online,sun2009latent,hoffmann2011knowledge,fernandes2012latent,bjorkelund2014learning}, but to our knowledge, we are the first to adapt them to reranking scenarios. 

Finally, we round out our discussion of question answering systems with a comparison to the famous Watson QA system, which achieved performance on par with the human champions in the Jeopardy! game~\cite{Ferucci:12}.
Several of the ideas proposed in our work are reminiscent of Watson. 
For example, our component that generates text aggregation graphs (Section 5) shares functionality with the Prismatic engine used in Watson. Similar to Watson, we extract evidence from multiple knowledge bases. However, there are three fundamental differences between Watson and this work. 
First, while Watson includes components for evidence gathering and scoring (we call these justifications), it uses a fundamentally different strategy for evidence generation. Similar to most previous work, the textual evidence extracted by Watson always takes the form of a contiguous segment of text~\cite{murdock2012textual},\footnote{Watson also generates ``structured evidence'' which is obtained by converting texts to structured representations similar to logic forms, which are then matched against structured databases for answer extraction. However, this ``logical representation of a clue and then finding the identical representation'' in a database resulted in ``confident answers less than 2\% of the time''~\cite{Ferucci:12}.} whereas our justifications aggregate texts from different documents or knowledge bases. We demonstrate in this work that information aggregation from multiple knowledge bases is fundamental for answering the science exam questions that are our focus (Section 8). 
Second, our answer ranking approach jointly ranks candidate answers and their justifications using a latent-variable learning algorithm, whereas Watson follows a pipeline approach where first evidence is generated, then answers are ranked~\cite{gondek2012framework}. We show in Section 8 that jointly learning answers and their justifications is beneficial. 
Last but not least, Watson was implemented as a combination of distinct models triggered by the different types of Jeopardy! questions, whereas our approach deploys a single model for all questions. Our analysis in Section~\ref{sec:erroranalysis} suggests that there are limits to our simple approach: we measure a ceiling performance for our single-model approach of approximately 70\%. To surpass this ceiling, one would have to  implemented dedicated domain-specific methods for the difficult problems left unsolved by our approach. 



\begin{figure}[t]
\begin{center}
\includegraphics[width=0.5\textwidth]{mainmatter/emnlp2017-qaj/arch_overall.png}
\caption{ Architecture of our question answering approach.  
Given a question, candidate answer, and a free-text knowledge base as inputs, we generate a pool of candidate justifications, from which we extract feature vectors.  We use a neural network to score each and then use max-pooling to select the current best justification. This serves as the score for the candidate answer itself.  The red border indicates the components that are trained online. }
\label{fig:arch_overall}
\vspace{-5mm}
\end{center}
\end{figure}

\section{Approach}
\label{sec:approach}
One of the primary difficulties with the explainable QA task addressed here is that, while we have supervision for the correct answer, we do not have annotated answer justifications.  
Here we tackle this challenge by using the QA task performance as supervision for the justification reranking, allowing us to 
%extending a neural QA model to jointly learn both how to 
%jointly 
learn to choose both the correct answer and a compelling, human-readable justification for that answer.

Additionally, similar to the strategy Chen and Manning~\citeyear{chen2014fast} applied to parsing, we combine representation-based features with explicit features that capture additional information that is difficult to model through embeddings, especially with limited training data.
%second contribution is that, similar to the strategy Chen and Manning~\citeyear{chen2014fast} applied to parsing, we combine representation-based features with explicit features that capture additional information that is difficult to model through embeddings.



% ms: avoid "system" too engineering-y
The architecture of our approach is summarized in Figure \ref{fig:arch_overall}.  
Given a question and a candidate answer, we first query an textual knowledge base (KB) to retrieve a pool of potential justifications for that answer candidate.  
For each justification, we extract a set of features designed to model the relations between questions, answers, and answer justifications based on word embeddings, lexical overlap with the question and answer candidate, discourse, and information retrieval (IR) (Section \ref{sec:features}).
These features are passed into a simple neural network to generate a score for each justification, given the current state of the model.  A final max-pooling layer selects the top-scoring justification for the candidate answer and this max score is used also as the score for the answer candidate.  
The system is trained using correct-incorrect answer pairs with a pairwise margin ranking loss objective function to enforce that the correct answer be ranked higher than any of the incorrect answers. 

%The key here is that we use the current state of the model to select the best justification for a given answer candidate from a pool of many candidate justifications.  To do this, we modify the training procedure such that at the start of each epoch \todo{minibatch instead of epoch?}, we first compute a forward pass with each candidate justification to find the top-scoring justification for each candidate answer.
%For a given question, answer candidate, and justification, we combine features based on word embeddings, lexical overlap, discourse, and information retrieval (IR) together in a simple neural architecture to generate a score for the answer candidate.    We then use this selected justification to calculate our gradients for updating the model parameters.  

With this end-to-end approach, the model learns to select justifications that allow it to correctly answer questions.  We hypothesize that this approach enables the model to indirectly learn to choose justifications that provide good explanations as to why the answer is correct. We empirically test this hypothesis in Section \ref{sec:results}, where we show that indeed the model learns to correctly answer questions, as well as to select high-quality justifications for those answers. 
% ms: misleading; it reads as if answer selection is better than IR
% better than a strong IR baseline. 

\section{Focus Word Extraction}
\label{sec:focuswords}


%
% Focus word extraction example
%
\begin{table*}[t]
\caption{{Focus word decomposition of an example question, suggesting the question is primarily about measuring the speed of walking, and not about turtles or paths. (Correct answer: ``a stopwatch and meter stick.'') 
For a given word: 
\emph{Conc} refers to the psycholinguistic concreteness score,
\emph{Tag} refers to the focus word category (\emph{FOCUS} signifies a focus word, \emph{EX} an example word, \emph{ATYPE} an answer-type word, and \emph{ST} a stop word),
\emph{Score} refers to the focus word score, and
\emph{Weight} refers to the normalized focus word scores. 
}}
\begin{center}
\begin{footnotesize}
\addtolength{\tabcolsep}{-1.7pt}  
\begin{tabular}{r|cccccccccccc}
Words & What	& tools	& could & determine & the  & {\bf speed} & of   & turtles & {\bf walking} & along & a    & path ? 		\\
Conc  & 2.0A    & 4.6C  & 1.3A  & 2.1A      & 1.4A & {\bf 3.6}   & 1.7A & 5.0C    & {\bf 4.1}     & 2.1A  & 1.5A & 4.4C 		\\
Tag	  & ST      & ATYPE & ST    & ST        & ST   & {\bf FOCUS} & ST   & EX      & {\bf FOCUS}   & ST    & ST   & EX   		\\
Score & --      & 1     & --    & --        & --   & {\bf 14}    & --   & 2       & {\bf 14}      & --    & --   & 3			\\
Weight& --      & 0.03  & --    & --        & --   & {\bf 0.41}  & --   & 0.06    & {\bf 0.41}    & --    & --   & 0.09		\\
\end{tabular}
\addtolength{\tabcolsep}{1.7pt}  
\end{footnotesize}

\label{tab:focusexample}
\end{center}
\end{table*}


Questions often contain text that is not strictly required to arrive at an answer, and that can introduce noise into the answering process. 
This is particularly relevant for science exams, where questions tend to include a narrative or example for the benefit of the reader, and this extra text can make determining a question's \emph{information need} more challenging. 
%
%Elementary science exam questions are often crafted to test a student's knowledge of a single concept, process, or model.
%While some exam questions are short and direct {\em (e.g. What does a thermometer measure?)}, others embed the question in a narrative or example, and we first have to distill what the question is about before we're able to answer.  
For example, the question in Table~\ref{tab:focusexample} could be simplified to {\em What tools can be used to measure speed?}, but instead grounds the question in an example about turtles walking along a path.  As a first step in answering, we need to identify whether the focus of the question is about {\em speed}, {\em turtles}, {\em walking}, or {\em paths}, so that we can appropriately constrain our intersentence aggregation, and decrease the chance of generating noisy or unrelated justifications. 

Note that the {\em focus word} terminology was first introduced in the context of factoid QA, where it represents the question word or phrase that is indicative of the expected answer type, which is then used to constrain the search for candidate answers~\cite{Harabagiu:00,Moldovan:2003:PIE:763693.763694}. For example, {\em capital} is the focus word in the question {\em What is the capital of France?}. This information is used to constrain the search for answers to  entities that are of names of cities. 
In contrast, such words (e.g., {\em tools} for the previous exam question) are often of low importance for multiple choice science exams, as this information is already implicitly provided in the multiple choice answers.  Instead, our task is to identify the central concept the question is testing (e.g., {\em  measuring speed}), and to eliminate words that are part of an example or narrative (e.g., {\em  turtle, path}) that are unlikely to contribute much utility (or, may introduce noise) to the QA process. However, because at a high level focus words identify the information need of a question, which is what we aim to do as well, we continue to use the same terminology in this work.

Our approach borrows from cognitive psychology, which suggests that elementary school students tend to reason largely with concrete concepts (i.e., those that are easy to mentally picture), because their capacity for abstract thinking develops much more slowly into adulthood (e.g., Piaget ~\citeyear{Piaget1954}).  Recently Brysbaert et al. ~\citeyear{brysbaert:2014} collected a large set of psycholinguistic concreteness norms, rating 40 thousand generally known English lemmas on a numerical scale from 1 (highly abstract) to 5 (highly concrete).  We have observed that highly abstract words (e.g., {\em expertise:1.6, compatible:2.3, occurrence:2.6)} tend to be part of a question's narrative or too abstract to form the basis for an elementary science question, while highly concrete words (e.g., {\em  car:4.9, rock:4.9, turtle:5.0)} are often part of examples.  Words that are approximately 50\% to 80\% concrete {\em (e.g., energy:3.1, measure:3.6, electricity:3.9, habitat:3.9)} tend to be at an appropriate level of abstraction for the cognitive abilities of elementary science students, and are often the central concept that a question is testing.\footnote{
While the above concreteness thresholds may be particular to elementary science exams, we hypothesize that the information need of a question tends to be more abstract than the examples grounding that question.  We believe this intuition may be general and applicable to other domains.
}

Making use of this observation, we identify these focus words with the algorithm below. The algorithm is implemented as a sequence of ordered sieves applied in decreasing order of precision~\cite{Lee:13}. Each of the five sieves attempts to assign question words into one of the following categories\footnote{This same process is used to extract focus words for each of the multiple choice answer candidates, though it is generally much simpler given that the answers tend to be short.}: 

\begin{enumerate}

\item {\bf Lists and sequences:} Lists in questions generally contain highly important terms.  We identify comma delimited lists of the form {\small {\tt X, Y, ..., <and/or> Z}} (e.g., {\em sleet, rain, and hail}). Given the prevalence of questions that involve causal or process knowledge, we also identify from/to sequences (e.g., {\em from a solid to a liquid}) using paired {\tt prep\_from} and {\tt prep\_to} Stanford dependencies~\cite{de2008stanford}. 

\item {\bf Focus words:} Content lemmas (nouns, verbs, adjectives, and adverbs) with concreteness scores between 3.0 and 4.2 in the concreteness norm database of Brysbaert et al. ~\citeyear{brysbaert:2014} are flagged as focus words. 

\item {\bf Abstract, concrete, and example words:} Content lemmas with concreteness scores between 1.0 and 3.0 are flagged as highly abstract, while those with concreteness scores between 4.2 and 5.0 are flagged as highly concrete.  Named entities recognized as either durations or locations (e.g., {\em In New York State}) are flagged as belonging to examples.  

\item {\bf Answer type words:} Answer type words are flagged using both four common syntactic patterns shown in Table~\ref{tab:answertypewords}, as well as a short list of transparent nouns (e.g., {\em kind, type, form}). 

\item {\bf Stop words: } A list of general and QA-specific stop words and phrases, as well as any remaining words not captured by earlier seives, are flagged as stop words.

\end{enumerate}

\begin{table*}[t]
\caption{{Syntactic patterns used to detect answer type words.  Square brackets represent optional elements. }}
\begin{center}
\begin{footnotesize}
\begin{tabular}{ll}
\hline
\multicolumn{1}{l}{Pattern} & \multicolumn{1}{l}{Example} 	\\
\hline
(SBARQ (WHNP (WHNP (WDT) {\bf (NN)) [(PP)]}...						& What {\em kind of energy} ...    	\\
(SBARQ (WHNP (WP)) (SQ (VBZ is) {\bf (NP)}...	 					& What is {\em one method} that ... 	\\
(S (NP (NP (DT A) {\bf (NN)}) (SBAR (WHNP (WDT that)) ...			& A {\em tool} that ..					\\
(S (NP {\bf (NP)} (PP)) (VP (VBZ is) ...    						& The {\em main function} of ... is to ...	\\

\end{tabular}
\end{footnotesize}

\label{tab:answertypewords}
\end{center}
\end{table*}





{\flushleft {\bf Scoring and weights:}} We then assign a score to each question word based on its perceived relevance to the question as follows. 
Stop words are not assigned a score, and are not included in further processing.
All answer type words are given a score of 1.  Words flagged as abstract, concrete, or example words are sorted based on their distance from the concreteness boundaries of 3.0 (for abstract words) or 4.2 (for concrete words), where words closer to the concreteness boundary tend to be more relevant to the question, and should receive higher scores.
These words are then given incrementally increasing scores starting at 2 for the most distant word, and increasing by one until all words in this category have been assigned a score.
To create an artificial separation between focus words and the less relevant words, and to encourage our intersentence aggregation method to preferentially make use of focus words, we assign all focus words a uniform score 10 points higher than the highest-scoring abstract, concrete, or example word.  
Finally, list words, the most important category, receive a uniform score one higher than focus words.  The scores of all words are converted into weights by normalizing the scores to sum to one. 
An example of the scoring process is shown in Table~\ref{tab:focusexample}. 

It is important to note that the proposed focus word extraction algorithm is simple and leaves considerable room for improvement. We hypothesize that better algorithms could be implemented by switching to a learning-based approach. However, the simple unsupervised algorithm proposed requires no data annotations, and it captures the crucial intuition that some words in the question  contribute more towards the overall information need than others.  We demonstrate that this has a considerable impact on the performance of our overall QA system in the ablation studies discussed in Section~\ref{sec:controls}.





\input{tig}
\input{perceptron}
\input{results}

%space{-1mm}
\section{Discussion}
\label{sec-naacl2015:discussion}
%space{-2mm}

%
% Alignment performance by part-of-speech association (noun/noun, verb/verb, noun/verb, etc)
%
\begin{comment}
\begin{table}[t!]
\begin{center}
%\begin{scriptsize}
\begin{footnotesize}
\begin{tabular}{llc}
\multicolumn{1}{l}{ } & \multicolumn{1}{l}{ } & \multicolumn{1}{l}{P@1} \\
\multicolumn{1}{l}{ Model/Features } & \multicolumn{1}{l}{P@1} & \multicolumn{1}{l}{Impr.} \\
\cline{2-3}

\hline
\multicolumn{3}{l}{\textit{Yahoo! Answers}} \\ % 185q (sent) ret=1p c=0.1 
\hline
CR Baseline 							& 19.00 					&	  				\\
Full model  							& 29.00 					& +XX\% 			\\
Verb  $\rightarrow$ Verb 				& {\bf XX.XX\ssa}			& {\bf +XX\%}		\\
Verb  $\rightarrow$ Noun 				& {\bf XX.XX\ssa}			& {\bf +XX\%}		\\
Noun  $\rightarrow$ Noun 				& {\bf XX.XX\ssa}			& {\bf +XX\%}		\\

\end{tabular}
%\end{scriptsize}
\end{footnotesize}
%\vspace{-2mm}
\caption{{\footnotesize  Performance of an alignment model trained with only verb$\rightarrow$verb, verb$\rightarrow$noun, or noun$\rightarrow$noun associations on the Y!A corpus.  Performance suggests the effect is primarily driven by XX$\rightarrow$XX associations. 
}}
\label{tab:associationtype}
%\vspace{-4mm}
\end{center}
\end{table}
\end{comment}

The utility of the proposed approach is clear -- by imposing structure over free text,  monolingual alignment can used (with extremely sparse data) to achieve large performance gains in non-factoid QA.  When discourse parsing is possible for the free text in the domain in question, intersentence and intrasentence alignments can be combined to reach maximal performance.  Otherwise, just a straightforward aligning of adjacent sentences can attain a close approximation of those results.  Both of these models far outperform an RNNLM when trained on limited amounts of data.  

In fact, it is notable how quickly we see performance increases from the alignment models in YA.  In both domains, results are increased due to word self-associations (when the same word appears in both the question and the answer), but with YA there also is a slight correlation between answer length and correctness.  The magnitude of the performance increase due to these factors is essentially shown in the results for the smallest sample size, as one document doesn't contain enough data to provide a real alignment contribution.  Taking this into account, the scores still increase quite rapidly for YA.  It could be the case that associations between high frequency verbs drive much of this performance.  If so, then perhaps for the more technical Bio domain, more world-knowledge (encoded in nouns and other parts of speech) is needed,  explaining why we don't see the same immediate performance gain.  To test this, matrices could be made with only verbs, only nouns, or other combinations to determine if one type of association is driving much of the performance and how the domains compare in this respect. 

After a rapid increase, the performance in YA quickly levels off relative to the amount of input for training.~\footnote{Only a fraction of 1 file is used to achieve the maximal performance for the alignment models.  In pilot experiments, including as many as 20 files did not improve performance. }  This may be a result of a decrease in the signal to noise ratio.  Unlike YA, gigaword is within the news domain.  As more and mode documents are added to training, perhaps the model becomes too far biased towards news to continue to increase performance.  If more sophisticated techniques are implemented, such as topic filtering, potentially the performance in this domain could be even higher.  We don't see the same plateau effect with Bio that we do with YA.  Likely this is due to the fact that the in-domain texbook for Bio was much smaller than gigaword.  With more in-domain data we would expect the results for all models to increase, and perhaps the performance curves would resemble those from YA.

\todo{strong close}





\input{erroranalysis}
\section{Conclusion}
\label{sec:conclusion}

We have proposed an approach for QA where producing human-readable justifications for answers, and evaluating answer justification quality, is the critical component.
Our interdisciplinary approach to building and evaluating answer justifications includes cognitively-inspired aspects, such as making use of psycholinguistic concreteness norms for focus word extraction, and making use of age-appropriate knowledge bases, which together help move our approach towards approximating the qualities of human inference on the task of question answering for science exams. Intuitively, our structured representations for answer justifications can be interpreted as a robust approximation of more formal representations, such as logic forms~\cite{moldovan2001logic}. However, our approach does not evaluate the quality of connections in these structures by their ability to complete a logic proof, but through a reranking model that measures their correlations with good answers.

In our quest for explainability, we have designed a system that generates answer justifications by chaining sentences together. Our experiments showed that this approach improves explainability, and, at the same time, answers questions out of reach of information retrieval systems, or systems that process contiguous text.  
We evaluated our approach on 1,000 multiple-choice questions from elementary school science exams, and experimentally demonstrated that our method outperforms several strong baselines at both selecting correct answers, and producing compelling human-readable justifications for those answers.  We further validated our three critical contributions: (a) modeling the high-level task of determining justification quality by using a latent variable model is important for identifying both correct answers and good justifications, (b) identifying focus words using psycholinguistic concreteness norms similarly benefits QA for elementary science exams, and (c) modeling the syntactic and lexical structure of answer justifications allows good justifications to be assembled and detected. 

We performed a detailed error analysis that suggests several important directions for future work. 
First, though the majority of errors can be addressed within the proposed formalism and by improving focus word extraction, 47.5\% of incorrectly answered questions would also benefit from more complex inference mechanisms, ranging from causal and process reasoning, to modeling quantifiers and negation.
This suggests that our robust approach for answer justification may complement deep reasoning methods for QA in the scientific domain~\cite{baral2011towards}.
Second, our text aggregation graphs currently capture intersentence connections solely through lexical overlap. We hypothesize that extending these structures to capture lexical-semantic overlap driven by word embeddings~\cite{mikolov13}, which have been demonstrated to be beneficial for QA~\cite{yih13,jansen14,fried2015higher}, would also be beneficial here, and increase robustness on small knowledge bases, where exact lexical matching is often not possible. 
Finally, while our answer justifications are currently short, future justifications might be quite long, and aggregate sentences from knowledge bases of different domains and genres.  In these situations, combining our procedure for constructing justifications with methods that improve text coherence~\cite{barzilay2008modeling} would likely improve the overall user experience for reading and making use of answer justifications from automated QA systems. 

To increase reproducibility, all the code behind this effort is released as open-source software\footnote{\url{https://github.com/clulab/releases/tree/master/cl2017-qa}}, which allows other researchers to use our entire science QA system as is, or to explore adapting the various components to other tasks. 


\newpage
\starttwocolumn
\bibliographystyle{fullname}
\bibliography{refs}


\end{document}
