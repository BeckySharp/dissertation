\documentstyle[fullname]{article}

\addtolength{\textwidth}{40pt}
\addtolength{\oddsidemargin}{-20pt}

\title{The Fullname Citation Style}
\author{Stuart Shieber}

\begin{document}

\maketitle

The fullname bibliography and citation style is implemented with two files, a
Bib\TeX\ style file {\tt fullname.bst} and a \LaTeX\ document-style
option file {\tt fullname.sty}.  These are specified in the 
\LaTeX\ source file as follows:

\begin{verbatim}
      \documentstyle[fullname]{article}
      \bibliographystyle{fullname}
\end{verbatim}

Three citation macros are defined.  Normal citations are specified
with \verb|\cite|, which specifies a parenthesized citation giving the
authors' last names (up to three authors) and the year of publication.
Here are citations for articles of from one to four authors: 

\begin{verse}\begin{tabular}{ll}
\verb|\cite{dewey1}| & \cite{dewey1}\\
\verb|\cite{dewey2}| & \cite{dewey2}\\
\verb|\cite{dewey3}| & \cite{dewey3}\\
\verb|\cite{dewey4}| & \cite{dewey4}
\end{tabular}\end{verse}

An optional argument allows for specifying further information in the
parentheses.

\begin{verse}\begin{tabular}{ll}
\verb|\cite[page 15]{dewey1}| & \cite[page 15]{dewey1}\\
\verb|\cite[page 15]{dewey2}| & \cite[page 15]{dewey2}\\
\verb|\cite[page 15]{dewey3}| & \cite[page 15]{dewey3}\\
\verb|\cite[page 15]{dewey4}| & \cite[page 15]{dewey4}
\end{tabular}\end{verse}

Multiple citations can be placed in a single parenthetical.

\begin{verse}\begin{tabular}{l}
\verb|\cite{dewey2,cheatham,howe}| \\
  \qquad  \cite{dewey2,cheatham,howe}\\
\verb|\cite[page 15]{dewey2,cheatham,howe}| \\
  \qquad  \cite[page 15]{dewey2,cheatham,howe}
\end{tabular}\end{verse}

The \verb|\namecite| macro places the names of the authors outside of
and before the parenthesized date.  This is useful for citations in
the text where the names are part of the sentence, as first described
by \namecite{dewey2}.

\begin{verse}\begin{tabular}{ll}
\verb|\namecite{dewey1}| & \namecite{dewey1}\\
\verb|\namecite{dewey2}| & \namecite{dewey2}\\
\verb|\namecite{dewey3}| & \namecite{dewey3}\\
\verb|\namecite{dewey4}| & \namecite{dewey4}\\
\verb|\namecite[page 15]{dewey1}| & \namecite[page 15]{dewey1}\\
\verb|\namecite[page 15]{dewey2}| & \namecite[page 15]{dewey2}\\
\verb|\namecite[page 15]{dewey3}| & \namecite[page 15]{dewey3}\\
\verb|\namecite[page 15]{dewey4}| & \namecite[page 15]{dewey4}
\end{tabular}\end{verse}

The \verb|\shortcite| macro does not place the authors' names at all.
This is useful if the authors' names occur in the sentence, but not
next to where the citation itself should be.  Dewey and Cheatham first
used this method \shortcite{dewey2}.  Here are examples of the use of 
\verb|\shortcite| in its various forms.

\begin{verse}\begin{tabular}{ll}
\verb|\shortcite{dewey1}| & \shortcite{dewey1}\\
\verb|\shortcite{dewey2}| & \shortcite{dewey2}\\
\verb|\shortcite{dewey3}| & \shortcite{dewey3}\\
\verb|\shortcite{dewey4}| & \shortcite{dewey4}\\
\verb|\shortcite[page 15]{dewey1}| & \shortcite[page 15]{dewey1}\\
\verb|\shortcite[page 15]{dewey2}| & \shortcite[page 15]{dewey2}\\
\verb|\shortcite[page 15]{dewey3}| & \shortcite[page 15]{dewey3}\\
\verb|\shortcite[page 15]{dewey4}| & \shortcite[page 15]{dewey4}
\end{tabular}\end{verse}


It doesn't make sense to use \verb|\namecite| or \verb|\shortcite|
with multiple citations.  The output would look like this:

\begin{verse}
\begin{tabular}{ll}
\verb|\namecite{dewey2,cheatham,howe}| \\
  \qquad  \namecite{dewey2,cheatham,howe}\\
\verb|\shortcite{dewey2,cheatham,howe}| \\
  \qquad  \shortcite{dewey2,cheatham,howe}
\end{tabular}
\end{verse}

The corresponding bibliography style places entries in alphabetical
order by first author's last name, and inverts the first author's name
to emphasize the ordering.

\bibliographystyle{fullname}
\bibliography{fullname-doc}

\end{document}
