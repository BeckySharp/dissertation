\chapter{RELATED WORK\label{chapter:related_work}}

In one sense, QA systems can be described in terms of their position along a formality continuum ranging from shallow models that rely on information retrieval, lexical semantics, or alignment, to highly structured models based on first order logic (FOL).

On the shallower end of the spectrum,  QA models can be constructed either from structured text, such as question--answer pairs, or unstructured text.  Alignment models~\citep{Berger:00,echihabi2003noisy,Soricut:06,Riezler:etal:2007,Surdeanu:11,yao2013}  require aligned question--answer pairs for training, a burden which often limits their practical usage.  In Chapter \ref{chapter:naacl2015} we address this burden by proposing a method for using the discourse structure of free text as a surrogate for this alignment structure. %\todo{ref to a related work section in that chapter? or keep it here in a subsection?}

Lexical semantic models such as neural-network language models~\citep{jansen14,sultan-etal:2014:TACL,yih13}, on the other hand, have the advantage of being readily constructed from free text.  
\citet{fried2015higher} called these approaches first-order models because associations are explicitly learned, and introduced a higher-order lexical semantics QA model where indirect associations are detected through traversals of the association graph.  
Other recent efforts have applied deep learning architectures to QA to learn non-linear answer scoring functions that model lexical semantics~\citep{Iyyer2014,nips15_hermann}.

These alignment and lexical semantic approaches, however do not take into account the wide variety of question type which often exist in any given question set, and therefore attempt to answer \emph{all} questions with the same set of word association values.  To address this, we propose an approach in Chapter \ref{chapter:emnlp2016} for training a dedicated set of word embeddings that are customized to a particular semantic relation (here, causality) and demonstrate that the semantic information contained in the embeddings is complementary to that which is contained in standard word embeddings and useful in a causal QA task.

While these word association approaches to QA have shown robust performance across a variety of tasks, one continuing disadvantage of these methods is that, even when a correct answer is selected, there is no clear human-readable justification for that selection.  This limits our ability to effectively understand model performance and adjust for errors accordingly.

Closer to the other end of the formality continuum, several approaches were proposed to not only select a correct answer, but also provide a formally valid justification for that answer.  For example, some QA systems have sought to answer questions by creating formal proofs driven by logic reasoning~\citep{moldovan2003cogex,moldovan2007cogex,balduccini2008knowledge,maccartney2009natural,liang2013learning,lewis2013combining}, answer-set programming \citep{baral2006using,baral2011towards,baral2012answering,baral2012knowledge}, or connecting semantic graphs~\citep{banarescu2012amr,sharmatowards}. 
However, the formal representations used in these systems, e.g., logic forms, are both expensive to generate and tend to be brittle because they rely extensively on imperfect tools for unsolved problems such as complete syntactic analysis and word sense disambiguation.

In Chapter \ref{chapter:cl2017}, we offer a lightly-structured sentence representation generated by our approach (see Section \ref{sec-cl2017:tag}) as a shallower and consequently more robust approximation of those logical forms, and show that they are well-suited for the complex questions (see Section \ref{sec:mcqa} for more details) we tackle.
This approach allows us to robustly aggregate information from a variety of knowledge sources to create human-readable answer justifications.  
It is these justifications which are then ranked in order to choose the correct answer, using a reranking perceptron with a latent layer that models the correctness of those justifications.

While this approach is shallower than many of the approaches based on formal representations, it still requires decomposing free text resources into lightly-structured representations and performing a fairly expensive aggregation.  In practice, this limits its use with very large text corpora.  Additionally, the perceptron-based learning framework is inherently limited by its assumption of linearlly-separable data.  To address these, in Chapter \ref{chapter:emnlp2017} we propose a shallower version of this model that can learn an embedded representation of the question, answer, and candidate justification texts and that can re-rank justifications using these representations alongside a small set of explicit features.  

In both of these latter approaches, the way we have formulated our justification selection (as a re-ranking of knowledge base sentences) is related to, but yet distinct from the task of answer sentence selection \cite[][inter alia]{Wang2010ProbabilisticTM, Severyn:12,Severyn:13a,Severyn:13b,Severyn2015LearningTR,wang2015long}.  Answer sentence selection is typically framed as a fully or semi-supervised task for factoid questions, where a correctly selected sentence fully contains the answer text.
Here, we have a variety of questions, many of which are non-factoid.  Additionally, we have no direct supervision for our justification selection (i.e., no labels as to which sentences are good justifications for our answers), motivating our distant supervision approach where the performance on our QA task serves as supervision for selecting good justifications.  Further, we are not actually looking for sentences that \emph{contain} the answer choice, as with answer sentence selection, but rather sentences which close the "lexical chasm" \cite{Berger:00} between question and answer (demonstrated in the example in Table \ref{tab:question_example}).
