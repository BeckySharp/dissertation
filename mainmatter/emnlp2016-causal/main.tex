
%\twocolumn[\centering \Large \bf Creating Causal Embeddings for Question Answering \\with Minimal Supervision \par
%~\\
%
%\large \bf Anonymous EMNLP submission
%~\\
%~\\
%]

%\begin{abstract}
%A common model for question answering (QA) is that a good answer is one that is closely related to the question, where relatedness is often determined using general-purpose lexical models such as word embeddings. 
%We argue that a better approach is to look for answers that are related to the question in a {\em relevant way}, according to the information need of the question,
%%We argue that a better approach is to look for answers that are related to the question in {\em the right way} \todo{"the right way" doesn't say much... Can we say something like "in a way relevant to the type of information need, or something like that?}, 
%which may be determined through task-specific embeddings. 
%With causality as a use case, we implement this insight in three steps. First, we generate causal embeddings cost-effectively by bootstrapping cause-effect pairs extracted from free text using a small set of seed patterns. Second, we train dedicated embeddings over this data, by using task-specific contexts, i.e., the context of a cause is its effect. Finally, we extend a state-of-the-art reranking approach for QA to incorporate these causal embeddings. We evaluate the causal embedding models both \emph{directly} with a casual implication task,
%% \todo{"Word similarity" makes it sound like we do lexical similarity... Can you say something causal implication?}, 
% and \emph{indirectly}, in a downstream causal QA task using data from Yahoo! Answers. We show that explicitly modeling causality improves performance in both tasks. In the QA task our best model achieves 37.3\% P@1, significantly outperforming a strong baseline by 7.7\% (relative). 
%% ms: not sure if we should discuss the differences between the 2 tasks here; we might not have space; it might dilute the message.
%
%%
%%Question answering (QA) is a difficult task, complicated by the variety of question types represented in any given question set.  In this work we propose addressing question types individually through the use of dedicated relation embeddings, and here focus on causal relations. 
%%We train causal embeddings (as well as two other popular distributional similarity models) on causal tuples extracted from free text resources with minimal supervision, using a small set of high-precision patterns.   
%%We evaluate these causal models both \emph{directly} in terms of their ability to detect causality, and \emph{indirectly}, in terms of their utility on a causal subset of Yahoo! Answers.
%%In both tasks, we show that explicitly modeling causality significantly improves performance, and in the QA task our best model achieves 37.3\% precision at one, outperforming a strong information retrieval and lexical semantic baseline by 7.7\% (relative). 
%%Importantly, we also show that the results of these two evaluations are \emph{not} identical: a given model's performance on the direct evaluation does not necessarily transfer to the more complex, real-world QA task.  
%\end{abstract}

\chapter{EMNLP2016 - CAUSAL EMBEDDINGS\label{chapter:emnlp2016}}




\section{Introduction}
\label{sec-emnlp2016:introduction}
%\vspace{-2mm}

Question answering (QA), i.e., finding short answers to natural language questions, is one of the most important but challenging 
tasks on the road towards natural language understanding~\cite{Etzioni:11}. 
A common approach for QA is to prefer answers that are closely related to the question, where relatedness is often determined using lexical semantic models such as word embeddings~\cite{yih13,jansen14,fried2015higher}. 
%Many QA systems answer questions by looking for answers that are closely related to the question, often as determined with lexical semantic word embeddings~\todo{citation}.  
While appealing for its robustness to natural language variation, this one-size-fits-all approach does not take into account the wide range of distinct question types that can appear in any given question set, and that are best addressed individually~\cite{chu2004ibm,ferrucci2010building,clark2013study}.  

Given the variety of question types, we suggest that a better approach is to look for answers % which are not simply closely related to the question, 
that are related to the question \emph{through the appropriate relation}, e.g., a causal question should have a cause-effect relation with its answer.
If we adopt this view, and continue to work with embeddings as a mechanism for assessing relationship,
this raises a key question: how do we train and use task-specific embeddings cost-effectively? 
Using causality as a use case, we answer this question with a framework for producing causal word embeddings with minimal supervision, and a demonstration that such task-specific embeddings significantly benefit causal QA. 
%Adopting this view, here we propose using task-specific word embeddings that combine the robustness and versatility of word embeddings with the precision of addressing a specific question type.  
%As a use case, we focus on producing custom embeddings with minimal supervision that capture causality and that are thus directly applicable to causal QA.

%\todo{REMOVE?: One important hurdle for QA is that there isn't a single ``universal engine'' that can answer any question, but rather a collection of methods, each tailored to a specific question type (e.g., factoid, definitional, or causal).
%This has been repeatedly observed throughout QA research, in various domains
%\cite{chu2004ibm,ferrucci2010building,clark2013study}. 
%Building from this observation, this paper proposes a framework for developing specific QA solving methods, and, in particular, on the rapid bootstrapping of knowledge resources needed by these solving methods. We encode this knowledge as customized embedding vectors, and demonstrate that they can be generated with minimal supervision. As a use case, we focus on producing custom embeddings that capture \textit{causality} and that are thus directly applicable to causal QA. }

In particular, the contributions of this work are:

%{\flushleft {\bf (1)}} 
%A novel approach for question answering that uses task-specific distributional similarity models 

{\flushleft {\bf (1)}} 
A methodology for generating causal embeddings cost-effectively by bootstrapping cause-effect pairs extracted from free text using a small set of seed patterns, e.g., {\em X causes Y}. 
%We propose a method to generate knowledge resources for causal questions 
%We demonstrate that knowledge resources for causal questions can be generated by bootstrapping cause-effect pairs extracted from free text using a small set of high-precision patterns, e.g., {\em X causes Y}. 
We then train dedicated embedding (as well as two other distributional similarity) models over this data. \citet{levy2014dependency} have modified the algorithm of\citet{mikolov2013distributed} to use an arbitrary, rather than linear, context. Here we make this context task-specific, i.e., the context of a cause is its effect.
%embedding models (as well as alignment and convolutional neural network models) over this data. 
Further, to mitigate sparsity and noise, our models are bidirectional, and noise aware (by incorporating the likelihood of noise in the training process). 
%We achieve the latter by weighting the examples based on the likelihood that they are truly causal rather than simply associative. 

{\flushleft {\bf (2)}} The insight that QA benefits from task-specific embeddings. % , and a demonstration that this approach significantly improves performance. 
We implement a QA system that uses the above causal embeddings to answer questions and demonstrate that they significantly improve performance over a strong baseline. Further, we show that causal embeddings encode complementary information to vanilla embeddings, even when trained from the same knowledge resources. 

{\flushleft {\bf (3)}} An analysis of direct vs. indirect evaluations for task-specific word embeddings. 
We evaluate our causal models both  {\em directly}, in terms of measuring their capacity to rank causally-related word pairs over word pairs of other relations, as well as {\em indirectly} in the downstream causal QA task. 
%Importantly, the above knowledge acquisition process is completely independent from these evaluation tasks, e.g., the objective function of the embedding model does not include any information from the QA task, which guarantees modularity. 
In both tasks, our analysis indicates that including causal models significantly improves performance. 
However, from the direct evaluation, it is difficult to estimate which models will perform best in real-world tasks. Our analysis re-enforces recent observations about the limitations of word similarity evaluations~\cite{faruqui2016problems}: we show that they have limited coverage and may align poorly with real-world tasks.

%{\flushleft {\bf (3)}} For causal QA, we show that causal embeddings encode complementary information to vanilla embeddings, even when trained from the same knowledge resources. 

%the models that include causal embeddings perform significantly better than the models that do not. Further, for causal QA, we show that causal embeddings are complementary to vanilla embeddings, underlining the complexity of this QA task, which must simultaneously capture causality and associations driven by distributional similarity. \todo{reword? seems like we're contradicting our earlier statement...}

%{\flushleft {\bf (4)}} Finally, we show that there are discrepancies between direct and indirect evaluations, i.e., no model performs best in both tasks. Our analysis re-enforces recent observations about the limitations of narrow word similarity evaluations~\cite{faruqui2016problems}, in that they both have limited coverage, and can poorly align with real world tasks.

\section{Related Work}
\label{sec:relatedwork}

In one sense, QA systems can be described in terms of their position along a formality continuum ranging from shallow models that rely on information retrieval, lexical semantics, or alignment, to highly structured models based on first order logic (FOL).  

On the shallower end of the spectrum,  QA models can be constructed either from structured text, such as question--answer pairs, or unstructured text.  Alignment models~\cite{Berger:00,echihabi2003noisy,Soricut:06,Riezler:etal:2007,Surdeanu:11,yao2013}  require aligned question--answer pairs for training, a burden which often limits their practical usage (though Sharp et al.~\citeyear{sharp-EtAl:2015:NAACL-HLT} recently proposed a method for using the discourse structure of free text as a surrogate for this alignment structure).
Lexical semantic models such as neural-network language models~\cite{jansen14,sultan-etal:2014:TACL,yih13}, on the other hand, have the advantage of being readily constructed from free text.  
Fried et al.~\citeyear{fried2015higher} called these approaches first-order models because associations are explicitly learned, and introduced a higher-order lexical semantics QA model where indirect associations are detected through traversals of the association graph.  
Other recent efforts have applied deep learning architectures to QA to learn non-linear answer scoring functions that model lexical semantics~\cite{Iyyer2014,nips15_hermann}.
However, while lexical semantic approaches to QA have shown robust performance across a variety of tasks, a disadvantage of these methods is that, even when a correct answer is selected, there is no clear human-readable justification for that selection.   

Closer to the other end of the formality continuum, several approaches were proposed to not only select a correct answer, but also provide a formally valid justification for that answer.  For example, some QA systems have sought to answer questions by creating formal proofs driven by logic reasoning~\cite{moldovan2003cogex,moldovan2007cogex,balduccini2008knowledge,maccartney2009natural,liang2013learning,lewis2013combining}, answer-set programming \cite{baral2006using,baral2011towards,baral2012answering,baral2012knowledge}, or connecting semantic graphs~\cite{banarescu2012amr,sharmatowards}. 
However, the formal representations used in these systems, e.g., logic forms, are both expensive to generate 
and tend to be brittle because they rely extensively on imperfect tools such as complete syntactic analysis and word sense disambiguation.
We offer the lightly-structured sentence representation generated by our approach (see Section \ref{sec:tag}) as a shallower and consequently more robust approximation of those logical forms, and show that they are well-suited for the complexity of our questions.
Our approach allows us to robustly aggregate information from a variety of knowledge sources to create human-readable answer justifications.  
It is these justifications which are then ranked in order to choose the correct answer, using a reranking perceptron with a latent layer that models the correctness of those justifications.


Covering the middle ground between shallow and formal representations, learning to rank methods based on tree-kernels~\cite{Moschitti:04} perform well for various QA tasks, including passage reranking, answer sentence selection, or answer extraction~\cite[inter alia]{Moschitti:07,Moschitti:11,Severyn:12,Severyn:13a,Severyn:13b,Tymoshenko:15}. 
The key to tree kernels' success is their ability to automate feature engineering rather than having to rely on hand-crafted features, which allows them to explore a larger representation space. Further, tree kernels operate over structures that encode syntax and/or shallow semantics such as semantic role labeling~\cite{Severyn:12}, knowledge from structured databases~\cite{Tymoshenko:15}, and higher level semantic information such as question category and focus words~\cite{Severyn:13b}.
Here, we similarly use structural features based on syntax, and enriched with additional information about how the answer candidate, the question, and the aggregated justification relate to each other.  
A key difference between our work and methods based on tree kernels is that rather than selecting a contiguous segment of text (sentence or paragraph) our justifications are aggregated from multiple sentences, often from different documents. Because of this setup, we explore content representations that continue to use syntax, but combined with robust strategies for cross-sentence connections. Further, because our justification search space is increased considerably due to the ability to form cross-sentence justifications, we restrict our learning models to linear classifiers that learn efficiently at this scale. However, as discussed, tree kernels offer distinct advantages over linear models. We leave the adaptation of tree kernels to the problem discussed here as future work.



Information aggregation (or fusion) is broadly defined as the assembly of knowledge from different sources, and has been used in several NLP applications, including summarization and QA.  In the context of summarization, information aggregation has been used to assemble summaries from non-contiguous text fragments~\cite[inter alia]{barzilay1999information,barzilay2005sentence}, while in QA, aggregation has been used to assemble answers to both factoid questions~\cite{pradhan2002building} and definitional questions~\cite{blair2003hybrid}.  Critical to the current work, in an in-depth open-domain QA error analysis, Moldovan et al. \citeyear{Moldovan:2003:PIE:763693.763694} identified a subset of questions for which information from a single source is not sufficient, and designated a separate class within their taxonomy of QA systems for those systems which were capable of performing answer fusion. Combining multiple sources, however, creates the need for context disambiguation -- an issue we tackle through the use of question and answer focus words.

Identifying question focus words, a subtask of question decomposition and identifying information needs, was found relevant for QA (especially factoid) early on~\cite[inter alia]{Harabagiu:00,Moldovan:2003:PIE:763693.763694} mainly as a means to identify answer types (e.g., "What is the {\em capital} of France?" indicates the expected answer type is \emph{City}).  
Recently, Park and Croft~\citeyear{Park:2015} have used focus words to reduce semantic drift in query expansion, by conditioning on the focus words when expanding non-focus query words.
Similarly, here, we use focus words (from both question and answer) to reduce the interference of noise in both building and ranking answer justifications.  By identifying which words are most likely to be important for finding the answer, we are able to generate justifications that preferentially connect sentences together on these focus words.  This results in justifications that are better able to remain on-context, and as we demonstrate in Section \ref{sec:experiments}, this boosts overall performance. 

Once the candidate answer justifications are assembled, our method selects the answer which corresponds to the best (i.e., highest-scoring) justification.  We learn which justifications are indicative of a correct answer by extending ranking perceptrons~\cite{Shen:Joshi:2005}, which have been previously used in QA~\cite{Surdeanu:11}, to include a latent layer that models the correctness of the justifications. Latent-variable perceptrons have been proposed for several other NLP tasks~\cite{liang2006end,zettlemoyer2007online,sun2009latent,hoffmann2011knowledge,fernandes2012latent,bjorkelund2014learning}, but to our knowledge, we are the first to adapt them to reranking scenarios. 

Finally, we round out our discussion of question answering systems with a comparison to the famous Watson QA system, which achieved performance on par with the human champions in the Jeopardy! game~\cite{Ferucci:12}.
Several of the ideas proposed in our work are reminiscent of Watson. 
For example, our component that generates text aggregation graphs (Section 5) shares functionality with the Prismatic engine used in Watson. Similar to Watson, we extract evidence from multiple knowledge bases. However, there are three fundamental differences between Watson and this work. 
First, while Watson includes components for evidence gathering and scoring (we call these justifications), it uses a fundamentally different strategy for evidence generation. Similar to most previous work, the textual evidence extracted by Watson always takes the form of a contiguous segment of text~\cite{murdock2012textual},\footnote{Watson also generates ``structured evidence'' which is obtained by converting texts to structured representations similar to logic forms, which are then matched against structured databases for answer extraction. However, this ``logical representation of a clue and then finding the identical representation'' in a database resulted in ``confident answers less than 2\% of the time''~\cite{Ferucci:12}.} whereas our justifications aggregate texts from different documents or knowledge bases. We demonstrate in this work that information aggregation from multiple knowledge bases is fundamental for answering the science exam questions that are our focus (Section 8). 
Second, our answer ranking approach jointly ranks candidate answers and their justifications using a latent-variable learning algorithm, whereas Watson follows a pipeline approach where first evidence is generated, then answers are ranked~\cite{gondek2012framework}. We show in Section 8 that jointly learning answers and their justifications is beneficial. 
Last but not least, Watson was implemented as a combination of distinct models triggered by the different types of Jeopardy! questions, whereas our approach deploys a single model for all questions. Our analysis in Section~\ref{sec:erroranalysis} suggests that there are limits to our simple approach: we measure a ceiling performance for our single-model approach of approximately 70\%. To surpass this ceiling, one would have to  implemented dedicated domain-specific methods for the difficult problems left unsolved by our approach. 



%\begin{figure}[t!]
%\begin{center}
%\includegraphics[width=75mm]{emnlpArchDraft.pdf}
%%\vspace{-4mm}
%\caption{{\small Our proposed pipeline for training a causal embedding space from free text resources. \todo{fill in learning with final model}}}
%\vspace{-6mm}
%\label{fig:arch}
%\end{center}
%\end{figure}
%
\section{Approach}
\label{sec:approach}
%\vspace{-2mm}

Our focus is on reranking answers to causal questions using using task-specific distributional similarity methods.
%on the the task of question answering, which is complicated by the variety of question types (e.g., definitional, causal),  each having different information needs that potentially require dedicated solving methods~\cite{clark2013study}.  
%Longer term, we propose a QA approach which uses dedicated, task-specific word embeddings for each question type, aimed at maintaining the robustness of word embeddings while gaining the specificity of dedicated solving methods. 
%Here, we address one particular information need: causality.
%
Our approach operates in three steps:

{\flushleft (1)} We start by bootstrapping a large number of cause-effect pairs from free text using a small number of syntactic and surface patterns (Section \ref{sec:causalextraction}).

{\flushleft (2)} We then use these bootstrapped pairs to build several task-specific embedding (and other distributional similarity) models (Section \ref{sec:models}). We evaluate these models directly on a causal-relation identification task (Section \ref{sec:directeval}).  

{\flushleft (3)} Finally, we incorporate these models into a reranking framework for causal QA and demonstrate that the resulting approach performs better than the reranker without these task-specific models, even if trained on the same data (Section ~\ref{sec:indirecteval}).  


%We approach the task of creating and evaluating a task-specific relational vector space by considering one relation in particular -- causality.  The architecture of our system is shown in Figure \ref{fig:arch}. \todo{do I really need an architecture here?}

%rule-based framework which analyzes free text and returns cause-effect tuples.  
%These pairs are then used to learn a set of high-dimensional word embeddings which are particular to the desired relation.  
%In particular, we make use of the Levy and Goldberg extension of the skipgram algorithm to learn embeddings from these pairs by predicting the effect-text given the cause-text.
%MOVE:
%using the extension of the Skipgram algorithm~\todo{cite and link} proposed by \citep{levy2014dependency}.  
%\todo{does this belong here? intro?}
%While the learning algorithm returns two distinct vector space embeddings for each item in the vocabulary, often only the target embeddings are ever used.  In this work, however, we make use of both sets of embeddings to capture the inherent \emph{directionality} of the causal relation.

%\todo{remove the evaluation info from here?}
%Once trained, we then evaluate the quality this mapping, or vector space, in two ways.  First, we evaluate it directly by attempting to rank a set of cause-effect pairs higher than entity pairs from other relations.  Second, we evaluate the mapping indirectly, by using it in the down-stream task of question answering (QA).


\input{mainmatter/emnlp2016-causal/causalextraction}
\input{mainmatter/emnlp2016-causal/models}
\input{mainmatter/emnlp2016-causal/directeval}
\input{mainmatter/emnlp2016-causal/qamodel}
\input{mainmatter/emnlp2016-causal/qa}

%\vspace{-1mm}
\section{Conclusion}
%\vspace{-1mm}
We presented a framework for creating customized embeddings tailored to the information need of causal questions.  We trained three popular models (embedding, alignment, and CNN) using causal tuples extracted with minimal supervision by bootstrapping cause-effect pairs from free text, and evaluated their performance both directly (i.e., the degree to which they capture causality), and indirectly (i.e., their real-world utility on a high-level question answering task). 


%We note that the results of these two evaluations are not identical; higher performance on the direct evaluation does \emph{not} necessarily correlate with higher performance in the QA task.
We showed that models that incorporate a knowledge of causality perform best for both tasks. 
Our analysis suggests that the models that perform best in the real-world QA task are those that have consistent performance across the precision-recall curve in the direct evaluation.
In QA, where the vocabulary is much larger, precision must be balanced with high-recall, and this is best achieved by our causal embedding model.  Additionally, we showed that vanilla and causal embedding models address different information needs of questions, and can be combined to improve performance. 

Extending this work beyond causality, we hypothesize that additional embedding spaces customized to the different information needs of questions would allow for robust performance over a larger variety of questions, and that these customized embedding models should be evaluated both directly and indirectly to accurately characterize their performance. 

%We introduced a methodology for producing causal embedding models cost-effectively by bootstrapping cause-effect pairs extracted from free text using a small set of seed patterns, and then training dedicated embedding models over this data using task-specific contexts, i.e., where the context of a cause is its effect. We then used these causal embedding models to implement a dedicated reranking model for causal QA. 

%We evaluated the generated embedding models both directly, in a word similarity task, and indirectly, in the downstream QA task. Our analysis yielded multiple observations. First, causal embeddings significantly outperform vanilla embeddings in both tasks, demonstrating the importance of having dedicated models for the task at hand. Second, for QA, the causal embeddings stack well with vanilla ones, highlighting that QA is a complex task, where solving methods need to address multiple information needs. Third, we note that the results of these two evaluations are not identical; higher performance on the direct evaluation does not necessarily correlate with higher performance in the QA task. Our analysis suggests that the performance on the direct evaluation is driven by precision, whereas for the real-world QA task, where the vocabulary is much larger, the precision must be balanced with high-recall which is best achieved by our causal embedding model.  

%We hypothesize that additional embedding spaces customized to the different information needs of questions would allow for robust performance over a larger variety of questions, and that these customized embedding models should be evaluated both directly and indirectly to accurately characterize their performance. 

\section*{Resources}
All code and resources needed to reproduce this work are  available at \url{http://clulab.cs.arizona.edu/data/emnlp2016-causal/}.

% ms: removed for anonymous submission
\section*{Acknowledgments}
We thank the Allen Institute for Artificial Intelligence for funding this work.
Additionally, this work was partially funded by the Defense Advanced
Research Projects Agency (DARPA) Big Mechanism
program under ARO contract W911NF-14-1-0395.


\newpage
%\bibliography{emnlp2016}
%\bibliographystyle{emnlp2016}

%\end{document}
