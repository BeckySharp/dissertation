%space{-1mm}
\section{Data}

%space{-1mm}
\subsection{Collection and preprocessing}

Twitter accounts were randomly selected for annotation on the basis of having a (non-default) profile picture and at least \edit{10 English-language tweets}. %10--7,000 tweets containing at least one word from a custom dictionary of food ingredient, dish, and preparation words, e.g., \textit{crab}, \textit{lasagna}, and \textit{saut\'{e}}. 
Amazon Mechanical Turk workers annotated each account's profile picture as \textit{overweight}, \textit{not overweight}, or \textit{can't tell}, with two annotations per account. Only accounts \edit{on which} both workers agreed \edit{on a label of \textit{overweight} or \textit{not overweight}} were retained. Interannotator agreement was 79.8\% \todo{use Kappa here not pct} on all three labels\edit{, and a by-hand analysis of fifty randomly selected accounts agreed on by two Turkers found one mis-annotated account. \textit{Can't tell} images, making up 41\% of agreed-on annotations, include those with a single person who was hard to classify, but much more often were images with either no humans or more than one human.}

\todo{Here discuss how much the ensemble of workers agree with the ground truth}

Of the 5,507 accounts successfully annotated, 676 (12.3\%) were annotated as overweight, the remainder being annotated as not overweight. This is much lower than 69.8\%, the United States Center for Disease Control (CDC)'s estimate for the proportion of adult Americans who have a body mass index (BMI) %at or over 
$\geq$~25.0~kg/m\textsuperscript{2} \cite{nhanes2013}. We speculate that this difference is due to several factors: first, Twitter users are, on average, younger\footnote{\url{http://www.pewinternet.org/2013/12/30/social-media-update-2013/twitter-users/}} and thus fitter than the average American; second, our sample is not guaranteed to be American, merely English-using; and third, because of factors such as social stigma associated with overweight and obesity, overweight individuals may be less likely than low-BMI individuals to post photographs of themselves. %Another explanations is selection bias, because accounts were selected on the basis of having tweets about food. It may be that the sample of Twitter users who tweet about food is biased toward low BMI. Finally, the annotators may have themselves been biased in their annotations.
\edit{We evaluate all classifiers and baselines on both our collected sample (12.3\% overweight) and a random subsample chosen to reflect US demographics (69.8\% overweight).}%To mitigate this difference, we selected 676 non-overweight accounts at random to match the number of overweight accounts for our classification.

Using the Twitter API\footnote{\url{https://dev.twitter.com/rest/public}}, we retrieved up to 3,200 of each Twitter user's most recent tweets (average 2,045).
% ms: removed for space, since we're not using this data
%along with their account information including text description, followers, and followees. 
% ms: removed for space  :(
%The tweets and descriptions were tokenized and part-of-speech tagged using ARK Tweet NLP \cite{owoputi2013improved}. Tokens from a preset stoplist of common English words and function-word or noise-indicating tags were removed. Web addresses, numbers, and user mentions were normalized. % and tokens assigned the tag \texttt{G} (a catch-all for foreign words and noisy input) or \texttt{-} (Twitter-specific discourse markers for retweets) were removed. Tokens assigned the \texttt{U} (URL, uniform resource locator), \texttt{@} (a user mention), or \texttt{\$} (cardinal number) tags were standardized to \texttt{<URL>}, \texttt{<@MENTION>}, and \texttt{<NUMBER>}, respectively. After this preprocessing, each tweet contained 15 tokens on average. 
The dataset is summarized in Table~\ref{tab:data}.
%\todo{this might be shortened if needed} % ms: done

% \begin{table*}
% \centering\small
% \begin{tabular}{l c c c c c c c c c c c}
% \toprule
% 		& \multicolumn{5}{c}{\textit{Overweight}} & & \multicolumn{5}{c}{\textit{Non-overweight}} \\ \cmidrule{2-6}\cmidrule{8-12}
% \textit{dataset}	& \textit{accts}	& \textit{tweets}	& \textit{tweets/acct}	& \textit{tokens}	& \textit{tok/tweet}	& & \textit{accts}	& \textit{tweets}	& \textit{tweets/acct}	& \textit{tokens}	& \textit{tok/tweet}	\\
% \midrule
% train	& 405	& 823~K	& 2032	& 12.6~M	& 15.3	& & 405	& 862~K	& 2128	& 13.2~M	& 15.4	\\
% dev		& 135	& 285~K	& 2113	& 4.44~M	& 15.6	& & 135	& 297~K	& 2202	& 4.62~M	& 15.5	\\
% test	& 136	& 277~K	& 2039	& 4.17~M	& 15.0	& & 136	& 286~K	& 2100	& 4.52~M	& 15.8	\\
% \midrule
% total	& 676	& 1385~K	& 2049	& 21.2~M	& 15.3	& & 676	& 1445~K	& 2137	& 22.4~M	& 15.5	\\
% \bottomrule
% \end{tabular}
% \caption{\todo{caption}}
% \label{tab:data}
% \end{table*}
\begin{table}
\centering\small
\begin{tabular}{@{}l l l l l l@{}}
\toprule
\textit{label} 	& \textit{accts}	& \textit{tweets}	& \textit{tweets/acct}	& \textit{toks}	& \textit{tok/tweet}	\\ 
\midrule
$+$	& 676	& 1.44~M	& 2137	& 22~M	& 15.5	\\
$-$	& 4831	& 9.82~M	& 2032	& 149~M	& 15.2	\\
\bottomrule
\end{tabular}
\caption{{\small summary of the size and qualities of the overweight ($+$) and non-overweight ($-$) accounts in the dataset collected for this work.}}
\label{tab:data}
%space{-3mm}
\end{table}
 
\subsection{By-hand analysis}

Two authors analyzed 46 randomly selected accounts by hand in order to establish how informative the text of Twitter accounts were for this task. Disagreements were discussed until 100\% agreement was achieved. Accounts were annotated in terms of whether they showed considerable transition (i.e., weight loss causing a person to no longer be overweight), 
%which is generally difficult to %judge from tweet text to an even great degree than overweight itself. 
which is difficult to determine.

Three accounts (7\%) were found to be in transition, in all cases from overweight to not overweight. Accounts were also annotated as 
``healthy" or ``unhealthy" based on having at least one tweet supportive of such diets and/or about exercise. 
%to whether they had at least one tweet indicative of unhealthy diet, and whether they had at least one tweet indicative of healthy diet and/or exercise. 
%Of the accounts analyzed, 40 (87\%) were judged to have at least one ``unhealthy'' tweet, 38 (83\%) were judged to have at least one ``healthy'' tweet, and 33 (72\%) at least one of each. Only one (2\%) was deemed uninformative, having neither of these informative kinds of tweets. Additionally, 4 (9\%) were misleading, e.g., they had only ``unhealthy'' tweets but were not overweight.
Finally, accounts having neither kind of tweet were annotated uninformative, and accounts having only tweets disagreeing with their account classification were annotated misleading. Results are summarized in Table~\ref{tab:byhandanalysis}

\begin{table}
% ms: shortened some of the tweets for space
\centering\small\renewcommand{\arraystretch}{1.5}
\begin{tabular}{p{2.25cm} l p{3.5cm}}
\toprule
\textit{Type}	& \textit{N (\%)}	& \textit{Example}	\\
\midrule
In transition	& 3 (7)	& paprika brown rice and wild rice with veg \#weightlossjourney \ldots \texttt{<}URL\texttt{>} \\ % \#weightloss \#healthyeating … \texttt{<}URL\texttt{>}	\\
$>0$ ``unhealthy''	& 40 (83)	& having charleys philly steak burger and fries for dinner, enjoying the yummy \ldots \\ % taste  of philidelphia's south market !!! \texttt{<}URL\texttt{>}	\\
$>0$ ``healthy'' & 38 (83)	& Just done day 8 and 9 of \#21dfx plus the \#hardcore one. Help! Might even do \#Pilates tonight \ldots \\ %to get up to par with my boyfriend. @21DayFix	\\
Both healthy and unhealthy tweets	& 33 (72)	& --- \\
Only uninformative	& 1 (2)	& ate something off the floor , can only taste floor now	\\
Misleading	& 4 (9)	& --- \\
\bottomrule
\end{tabular}
\caption{{\small Number and proportion of Twitter users matching annotations in the by-hand analysis. Examples show tweets from users belonging to this category.}} %Accounts labeled \textit{misleading} are those whose tweets only indicate the opposing label to their gold annotated label.}
\label{tab:byhandanalysis}
%space{-3mm}
\renewcommand{\arraystretch}{1}
\end{table}

%Normal-weight 
Non-overweight Twitter users wrote more relevant tweets about both food and exercise, though their overall number of tweets was slightly less on average. Our hypothesis is that because individuals on social media are motivated to positively present themselves to others \cite{walther2007selective}, they %generally
draw attention away from socially stigmatized qualities, such as being overweight. A similar bias toward the socially acceptable has been shown in survey responses \cite{phillips1972some}. The exception proving the rule would be accounts rebelling against fat shaming, employing terms such as \textit{\#effyourbodystandards}. This hypothesis %meshes 
aligns with our observation that overweight images are nearly six times less prevalent on Twitter (12.3\% in our sample) than one would expect from US overweight rates for adults (69.8\%).

Also agreeing with our self-censorship hypothesis, only overweight individuals had exclusively ``uninformative'' tweets, i.e., they had no tweets about diet or exercise. Likewise, only overweight individuals in this sample were consistently ``misleading'', i.e., tweeting only about healthy foods and activities. 
%By hypothesis, we propose that this is partly
This could be due to self-censorship (which non-overweight individuals are not compelled to in this case), and to the reality that it is possible to eat a healthy diet and exercise while still being overweight.

Based on the rate of misleading and uninformative twitter accounts in this analysis, totaling 11\%, we 
conclude that on a text-only basis, ceiling performance is approximately 89\% on this task.

