\section{Introduction}
\label{sec-emnlp2017:intro}

Developing interpretable machine learning (ML) models, that is, models where a human user can \emph{understand} what the model is learning, is considered by many to be crucial for ensuring usability and accelerating progress \cite{craven1996extracting,Kim2015MindTG, letham2015interpretable, Ribeiro2016WhySI}.  
% bs: removed for space, we talk more about this in related work
%As such, it has received much attention in recent years, especially as deep learning and complex architectures have seen dramatic gains in many tasks.
For many applications of question answering (QA), i.e., finding short answers to natural language questions, simply providing an answer is not sufficient. A complete approach must be interpretable, i.e., able to {\em explain} why an answer is correct. 
For example, in the medical domain, a QA approach that answers treatment questions would not be trusted if the treatment recommendation is not explained in terms that can be understood by the human user. 


One approach to interpreting complex models is to make use of human-interpretable information % metric % ms: not a metric...
 generated by the model to gain insight into what the model is learning.  This can be an intermediate representation used by the model, as with the model-generated text spans of \citet{Lei2016RationalizingNP}, that serve as input to another classification network.  
By learning these intermediate representations end-to-end with a downstream task, they are optimized to correlate with what the model learns is discriminatory for the task, and they can be evaluated against what a human would consider to be important.
\todo{need to define what a downstream task means to use this term}, they are optimized to correlate with what the model learns is discriminatory for the task, and they can be evaluated against what a human would consider to be important.
Here we apply this general framework for model interpretability to QA.


\begin{table}[t]
\begin{center}
\begin{footnotesize}
\begin{tabularx}{\linewidth}{p{0.13cm}p{6.8cm}}
\multicolumn{2}{p{8cm}}{\textbf{Question:} Which of these is a response to an internal stimulus?} \\
 (A) & A sunflower turns to face the rising sun. \\
 (B) & A cucumber tendril wraps around a wire. \\
 (C) &  A pine tree knocked sideways in a landslide grows upward in a bend. \\
 (\textbf{D}) &\textbf{Guard cells of a tomato plant leaf close when there is little water in the roots .} \\
\\
\multicolumn{2}{p{7.2cm}}{\textbf{Justification:} 
Plants rely on hormones to send signals within the plant in order to respond to internal stimuli such as a lack of water or nutrients. } \\

\end{tabularx}
\end{footnotesize}
\caption{{  Example of an 8th grade science question with a justification for the correct answer.  Note the lack of direct lexical overlap present between the justification and the correct answer, demonstrating the difficulty of the task of finding justifications using traditional distant supervision methods. }}
%space{-6mm} 
\label{tab:question_example}
\end{center}
\end{table}

In this work, we focus on answering multiple-choice science exam questions (Clark \citeyear{clark:2015}; see example in Table~\ref{tab:question_example}). 
This domain is challenging as: (a) approximately 70\% of science exam question shave been shown to require complex forms of inference to solve \cite{clark:2013,jansen-EtAl:2016:COLING}, and (b) there are few structured knowledge bases to support this inference.  
Within this domain, we propose an approach that learns to both select and explain answers, when the only supervision available is for which answer is correct (but not how to explain it).
Intuitively, our approach chooses the justifications that provide the most help towards ranking the correct answers higher than incorrect ones.
More formally, our neural network approach alternates between using the current model with max-pooling to choose the highest scoring justifications for correct answers, and optimizing the answer ranking model given these justifications. 
Crucially, these reranked texts serve as our human-readable answer justifications, and by examining them, we gain insight into what the model learned was useful for the QA task.   


The specific contributions of this work are:
\begin{enumerate}
\item We propose an end-to-end neural method for learning to answer questions and select a high-quality justification for those answers. 
Our approach re-ranks free-text answer justifications without the need for structured knowledge bases. 
With supervision only for the correct answers, we learn this re-ranking through a form of distant supervision -- i.e., the answer ranking supervises the justification re-ranking. 

\item We investigate two distinct categories of features in this ``little data'' domain: explicit features, and learned representations. We show that, with limited training, explicit features perform far better despite their simplicity. 

\item We demonstrate a large (+9\%) improvement in generating high-quality justifications over a strong information retrieval (IR) baseline, while maintaining near state-of-the-art performance on the multiple-choice science-exam QA task, demonstrating the success of the end-to-end strategy.
\end{enumerate}
