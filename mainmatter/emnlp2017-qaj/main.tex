%\title{emnlp 2017 instructions}
% File emnlp2017.tex
%

\documentclass[11pt,letterpaper]{article}
\usepackage{emnlp2017}

\usepackage{times}
\usepackage{latexsym}
 
% Uncomment this line for the final submission:
%\emnlpfinalcopy

%  Enter the EMNLP Paper ID here:
\def\emnlppaperid{***}

% Our packages
\usepackage{float}
\usepackage[font=small]{caption}
\usepackage{color}
\usepackage{graphicx}
\usepackage{booktabs}
\usepackage{subcaption}
\usepackage{hyperref}
\usepackage[utf8]{inputenc}
\usepackage{tabularx}
\usepackage{tablefootnote}
\usepackage{amsmath}
\usepackage{amssymb}
\usepackage{placeins}
\usepackage{scrextend}

\special{papersize=210mm,297mm}

%\usepackage{fixltx2e}

\newcommand{\todo}[1]{\textcolor{red}{TODO: #1}}
\newcommand{\edit}[1]{\textcolor{red}{#1}}
\newcommand{\bs}[1]{\textcolor{orange}{bs: #1}}

% To expand the titlebox for more authors, uncomment
% below and set accordingly.
% \addtolength\titlebox{.5in}    

%\title{Tell Me Why: Learning to Rank Answers and Their Justifications \todo{Jointly}}
\title{Tell Me Why: Using Question Answering as Distant Supervision for Answer Justification}


% Author information can be set in various styles:
% For several authors from the same institution:
% \author{Author 1 \and ... \and Author n \\
%         Address line \\ ... \\ Address line}
% if the names do not fit well on one line use
%         Author 1 \\ {\bf Author 2} \\ ... \\ {\bf Author n} \\
% For authors from different institutions:
% \author{Author 1 \\ Address line \\  ... \\ Address line
%         \And  ... \And
%         Author n \\ Address line \\ ... \\ Address line}
% To start a seperate ``row'' of authors use \AND, as in
% \author{Author 1 \\ Address line \\  ... \\ Address line
%         \AND
%         Author 2 \\ Address line \\ ... \\ Address line \And
%         Author 3 \\ Address line \\ ... \\ Address line}
% If the title and author information does not fit in the area allocated,
% place \setlength\titlebox{<new height>} right after
% at the top, where <new height> can be something larger than 2.25in

\date{}

\begin{document}

\maketitle

\begin{abstract}

For many applications of question answering (QA), being able to explain why a given model chose an answer is critical.  However, the lack of labeled data for  answer justifications makes learning this difficult and expensive.  Here we propose an approach that uses answer ranking as distant supervision for learning how to select informative justifications. 
We propose a neural network architecture for QA that reranks answer justifications as an intermediate (and human-interpretable) step in answer selection. Our approach is informed by a set of features designed to combine both learned representations and explicit features to capture the connection between questions, answers, and answer justifications.
 We show that with this end-to-end approach we are able to significantly improve upon a strong IR baseline in both justification ranking (+9\% rated highly relevant) and answer selection (+6\% P@1).  %We also provide an error analysis that demonstrates that we can use the justifications returned to better understand what our model has learned.
\end{abstract}

\section{Introduction}
\label{sec-emnlp2017:intro}

Developing interpretable machine learning (ML) models, that is, models where a human user can \emph{understand} what the model is learning, is considered by many to be crucial for ensuring usability and accelerating progress \cite{craven1996extracting,Kim2015MindTG, letham2015interpretable, Ribeiro2016WhySI}.  
% bs: removed for space, we talk more about this in related work
%As such, it has received much attention in recent years, especially as deep learning and complex architectures have seen dramatic gains in many tasks.
For many applications of question answering (QA), i.e., finding short answers to natural language questions, simply providing an answer is not sufficient. A complete approach must be interpretable, i.e., able to {\em explain} why an answer is correct. 
For example, in the medical domain, a QA approach that answers treatment questions would not be trusted if the treatment recommendation is not explained in terms that can be understood by the human user. 
%, the need for interpretable models:
%QA is one of the most challenging natural language understanding tasks~\cite{Etzioni:11}.  Adding to this challenge is the fact that, 
%for many applications of QA, simply providing an answer is not sufficient. A complete approach must also {\em explain} why an answer is correct. 
%For example, in the medical domain, a QA approach that answers treatment questions would not be trusted if the treatment recommendation is not explained in terms that can be understood by the human user. 
%Despite the importance of explanatory QA, the task of providing justifications as to \emph{why} the extracted answers are correct often takes a backseat to the accuracy of the system, and is not evaluated.
%is often overlooked and not evaluated. 

One approach to interpreting complex models is to make use of human-interpretable information % metric % ms: not a metric...
 generated by the model to gain insight into what the model is learning.  This can be an intermediate representation used by the model, as with the model-generated text spans of \citet{Lei2016RationalizingNP}, that serve as input to another classification network.  
By learning these intermediate representations end-to-end with a downstream task, they are optimized to correlate with what the model learns is discriminatory for the task, and they can be evaluated against what a human would consider to be important.
%\todo{need to define what a downstream task means to use this term}, they are optimized to correlate with what the model learns is discriminatory for the task, and they can be evaluated against what a human would consider to be important.
Here we apply this general framework for model interpretability to QA.
% ms: redundant with paragraph below
% propose a QA model that is able to provide free-text passages as justifications for the answers it selects.  

\begin{table}[t]
\begin{center}
\begin{footnotesize}
\begin{tabularx}{\linewidth}{p{0.13cm}p{6.8cm}}
\multicolumn{2}{p{8cm}}{\textbf{Question:} Which of these is a response to an internal stimulus?} \\
 (A) & A sunflower turns to face the rising sun. \\
 (B) & A cucumber tendril wraps around a wire. \\
 (C) &  A pine tree knocked sideways in a landslide grows upward in a bend. \\
 (\textbf{D}) &\textbf{Guard cells of a tomato plant leaf close when there is little water in the roots .} \\
\\
\multicolumn{2}{p{7.2cm}}{\textbf{Justification:} 
Plants rely on hormones to send signals within the plant in order to respond to internal stimuli such as a lack of water or nutrients. } \\

\end{tabularx}
\end{footnotesize}
\caption{{  Example of an 8th grade science question with a justification for the correct answer.  Note the lack of direct lexical overlap present between the justification and the correct answer, demonstrating the difficulty of the task of finding justifications using traditional distant supervision methods. }}
%space{-6mm} 
\label{tab:question_example}
\end{center}
\end{table}

In this work, we focus on answering multiple-choice science exam questions (Clark \citeyear{clark:2015}; see example in Table~\ref{tab:question_example}). 
This domain is challenging as: (a) approximately 70\% of science exam question shave been shown to require complex forms of inference to solve \cite{clark:2013,jansen-EtAl:2016:COLING}, and (b) there are few structured knowledge bases to support this inference.  
Within this domain, we propose an approach that learns to both select and explain answers, when the only supervision available is for which answer is correct (but not how to explain it).
%Within this domain, we propose an approach that jointly learns to select and explain answers, when the only supervision available is for which answer is correct (but not how to explain it) 
%\todo{PC's comment was remove reference to joint learning so it doesn't get us into trouble.  Do we need to say that we're learning the QA task while providing latent information for the explanation task?}. 
Intuitively, our approach chooses the justifications that provide the most help towards ranking the correct answers higher than incorrect ones.
More formally, our neural network approach alternates between using the current model with max-pooling to choose the highest scoring justifications for correct answers, and optimizing the answer ranking model given these justifications. %\todo{ms: tried to simplify, please check}
%bs - very good! thanks!
% scores all possible justifications for a chosen answer with the current model, and then uses max-pooling 
% re-ranks supporting textual evidence for a chosen answer then uses max-pooling 
%\todo{first reference to deep learning -- needs to be introduced before hand} 
%to choose the highest scoring justification and assign the score of the top-ranked justification to the answer candidate itself for use in the answer selection.  
%such that \todo{expand a bit; mention max-pooling?}.  
Crucially, these reranked texts serve as our human-readable answer justifications, and by examining them, we gain insight into what the model learned was useful for the QA task.   


The specific contributions of this work are:
\begin{enumerate}
\item We propose an end-to-end neural method for learning to answer questions and select a high-quality justification for those answers. 
Our approach re-ranks free-text answer justifications without the need for structured knowledge bases. 
%We formulate this such that the answer ranking supervises the justification re-ranking. 
With supervision only for the correct answers, % but not the quality of their corresponding justifications, % ms: redundant with previous text
we learn this re-ranking through a form of distant supervision -- i.e., the answer ranking supervises the justification re-ranking. 
%bs: I added detail in body of intro
%\todo{this is not very clear; say explicitly what we do with max pooling? or, actually, probably ok to remove since it's redundant to main body of the intro; better explain it there, and just summarize the novelty here}
%Because in our setup we have supervision only for the correct answers, but not for the corresponding answer justifications, .

\item We investigate two distinct categories of features in this ``little data'' domain: explicit features, and learned representations. % \todo{Deep learning jargon, needs to be introduced} \bs{I mainly diagree... and due to serious lack of space I can't intro... so idk.  mihai?}.  % ms: not necessary; well known in this community
We show that, with limited training, explicit features perform far better despite their simplicity. %\todo{Are they simpler? They were generated by a human carefully considering the data}.
%bs: i think we can support their simplicity --they're based on like, proportion of lexical overlap and IR scores.

\item We demonstrate a large (+9\%) improvement in generating high-quality justifications over a strong information retrieval (IR) baseline, while maintaining near state-of-the-art performance on the multiple-choice science-exam QA task, demonstrating the success of the end-to-end strategy.
%Importantly, our approach explains its selected answers significantly better than  a strong IR baseline, demonstrating the success of the joint strategy.
%We demonstrate near state-of-the-art performance on the multiple-choice science-exam QA task: our system would have placed 7th in a recent Kaggle challenge (top 4\%).\footnote{{\scriptsize \url{https://www.kaggle.com/c/the-allen-ai-science-challenge/leaderboard}}} 
%Further, our approach outperforms a recent model that relies on structured knowledge bases~\citep{khot2017tupleinf}, despite having a minimally-tuned, simple architecture, and a smaller set of text-only resources. 
% ms: I took it out because the natural question is why don't we evaluate on that dataset as well?
%which achieves state-of-the-art performance on a related question set. 
\end{enumerate}



%\todo{remove me later -- had weird splitting citation issue again.......... } 

%\todo{move this para to related work}

%The way we have formulated our justification selection (as a re-ranking of knowledge base sentences) is related to, but distinct from the task of answer sentence selection \cite[][inter alia]{Wang2010ProbabilisticTM, Severyn:12,Severyn:13a,Severyn:13b,Severyn2015LearningTR,wang2015long}.  Answer sentence selection is typically framed as a fully or semi-supervised task for factoid questions, and 
%.  Additionally, even when this isn't the case and a semi-supervised approach is necessary, 
%the problem is designed such that a correctly selected sentence will fully contain the answer text.
%the exact span of the answer is fully contained within the correct sentence.   
%Here, we have a variety of questions, many of which are non-factoid.  Additionally, we also have no direct supervision for our justification selection (i.e., we have no labels as to which sentences are good jutsifications for our answers), requiring a form of distant supervision where the performance on our QA task serves as our only signal as to whether or not we are selecting good jutstifcations.   Further, we are not actually looking for sentences that \emph{contain} the answer choice, as with answer sentence selection, but rather sentences which close the "lexical gap" \cite{Berger:00} between question and answer (as demonstrated in the question in Table \ref{tab:question_example}). 


%
%\todo{choose either ``explanation'' or ``justification'' and be consistent}
%\bs{i vote justification and tried to be consistent -- arguments? I'll leave the todo til the end to make sure we're good} 
%
%Question answering (QA), i.e., finding short answers to natural language questions, is one of the most challenging natural language understanding tasks~\cite{Etzioni:11}. Adding to this complexity is the fact that, for many applications of QA, simply providing an answer is not sufficient. A complete approach must also {\em explain} why an answer is correct. 
%For example, in the medical domain, a QA approach that answers treatment questions would not be trusted if the treatment recommendation is not explained in terms that can be understood by the human user. 
%Despite the importance of explanatory QA, the task of providing justifications as to why the extracted answers are correct is often overlooked and not evaluated. 
%
%In this work we argue that answers and their justifications re-enforce each other, and, thus, answer selection and justification should be addressed jointly.
%%bs: removed -- 2 ppl now think this is a bit of a jump/stretch
%%We see this as an initial step towards explainable artificial intelligence.\footnote{For more details see DARPA's Explainable Artificial Intelligence program: {\scriptsize \url{http://www.darpa.mil/program/explainable-artificial-intelligence}}.}
%In particular, we propose a joint approach to select and explain answers to multiple-choice science exam questions~\todo{cite ai2} by re-ranking supporting evidence in the form of free-text. The domain of science exam questions is challenging as: (a) many questions require natural language inference to be answered (see, for example, the question in Table~\ref{tab:question_example}) and (b) there are few structured knowledge bases to support this inference.  
%
%%Additionally, the task of selecting a justification for an answer in this domain is difficult because unlike in typical answer sentence selection tasks with factoid questions (where answer sentences generally completely contain the answer text)~\todo{cite moschitti}, here there is often a ``lexical chasm''~\cite{Berger:00} between answer choices and their justifications.  Also, answer sentence selection is typically a fully-supervised problem but here we have no labels for good justifications.
%The way we have formulated our justification selection (as a re-ranking of knowledge base sentences) is related to, but distinct from the task of answer sentence selection \cite[][inter alia]{Wang2010ProbabilisticTM, Severyn:12,Severyn:13a,Severyn:13b,Severyn2015LearningTR,wang2015long}.  Answer sentence selection is typically framed as a fully supervised task for factoid questions.  Additionally, even when this isn't the case and a semi-supervised approach is necessary, the problem is designed such that the answer is fully contained within the desired KB sentence.   Here, we have a variety of questions, many of which are non-factoid.  We also have no direct supervision for our justification selection, requiring a form of distant supervision (i.e., our joint-learning approach), and to further complicate the task, we are not specifically looking for sentences that \emph{contain} the answer choice, but rather sentences which close the "lexical gap" \cite{Berger:00} between question and answer (as demonstrated in the question in Table \ref{tab:question_example}). 

%Additionally, the task of selecting a justification for an answer in this domain is difficult because unlike in typical answer sentence selection tasks with factoid questions (where answer sentences generally completely contain the answer text)~\todo{cite moschitti}, here there is often a ``lexical chasm''~\cite{Berger:00} between answer choices and their justifications.  Also, answer sentence selection is typically a fully-supervised problem but here we have no labels for good justifications.

%\todo{reformulate -- our justs may or may not contain the answer - may be lexical chasm... not designed to contain the answer, but to address the lex chasm}


%Intro: Explanatory ML. Not �"black boxes". Here we focus on explainable QA. Selecting correct answers is important, but providing justifications as to why those answers are correct is critical in many domains.  This is often overlooked and not evaluated, and limits the ultimate utility and applicability of these systems.

%What we do: focus on science exam questions. hard for two reasons: ``lexical chasm'' which requires natural language inference to be solved. 
%We propose and evaluate an deep learning based approach that jointly extracts and explains answers. 

\section{Related work}

In many ways, deep learning has become the canonical example of the "black box" of machine learning and many of the approaches to explaining it can be loosely categorized into two types: approaches that try to interpret the parameters themselves (e.g., with visualizations and heat maps \citep{Zeiler2014VisualizingAU,nips15_hermann, Li2016VisualizingAU}, and approaches that generate a human-interpretable metric that is ideally correlated with what is being learned inside the model (e.g., \citet{Lei2016RationalizingNP}). Our approach falls into the latter type -- 
we use our model's reranking of human-readable justifications to give us insight into what the model considers informative for answering questions.  This allows us to see where we do well (Section \ref{sec:justification_results}), and where we can improve (Section  \ref{sec:erroranalysis}).

Deep learning has been successfully applied to many recent QA approaches and related tasks \cite[][inter alia]{Bordes2015LargescaleSQ,nips15_hermann, He2016CharacterLevelQA, dong2015question, Tan2016ImprovedRL}.
%are enticing, and for good reason -- many recent approaches to question answering and related tasks have had much success with various deep learning models \cite[][inter alia]{Bordes2015LargescaleSQ,nips15_hermann, He2016CharacterLevelQA, dong2015question, Tan2016ImprovedRL}.  
However, large quantities of data are needed to train the millions of parameters often contained in these models.  
%Of potentially greater utility in low-data domains, 
Recently, simpler model architectures have been proposed that greatly reduce the number of parameters while maintaining high performance \cite[e.g.,][]{Iyyer2015,chen2016thorough,Parikh2016ADA}.  
%For example, \citet{Iyyer2015}'s show that with their Deep Averaged Network, which replaces complex recurrent neural networks with an average of embeddings and a few, albeit large, dense layers, they improved performance on both a sentiment analysis and a QA task.  For natural language inference, \citet{Parikh2016ADA} used a simpler neural alignment  approach with an attention mechanisms to greatly reduce the size of their model while reaching then state-of-the-art performance.  
We take inspiration from this trend and propose a simple neural architecture for our task to offset the limited available training data. 

Another way to mitigate sparse training data is to include higher-level explicit features.  Like \citet{sachan2016science}, we make use of explicit features alongside features from distributed representations to capture connections between questions, answers, and supporting text.  However, we use a simpler set of features and while they use structured and semi-structured knowledge bases, we use only free-text.  %Additionally, though we also learn to select support from our knowledge base (in some ways similar to \citeauthor{sachan2016science}'s latent answer-entailing structure), since we are explicitly trying to perform \emph{explainable} question answering, here we evaluate the justifications learned by our approach and show that they are significantly better than a  strong IR baseline (Section \ref{sec:justification_results}).   

Our approach to learning justification reranking end-to-end with answer selection is similar to the \citet{jansen2017framing} latent reranking perceptron,  which also operates over free text.  However, our approach does not require decomposing the text into an intermediate representation, allowing our technique to more easily extend to larger textual knowledge bases.  
%ir approach is able to aggregate information from a variety of sources into one justification, our light-weight approach however, does not rely on a large set of heuristics and heavy pre-processing to transform free-text into a structured knowledge base.  

The way we have formulated our justification selection (as a re-ranking of knowledge base sentences) is related to, but distinct from the task of answer sentence selection \cite[][inter alia]{Wang2010ProbabilisticTM, Severyn:12,Severyn:13a,Severyn:13b,Severyn2015LearningTR,wang2015long}.  Answer sentence selection is typically framed as a fully or semi-supervised task for factoid questions, where a correctly selected sentence fully contains the answer text.
%and the problem is designed such that a correctly selected sentence will fully contain the answer text.
Here, we have a variety of questions, many of which are non-factoid.  Additionally, we have no direct supervision for our justification selection (i.e., no labels as to which sentences are good justifications for our answers), motivating our distant supervision approach where the performance on our QA task serves as supervision for selecting good justifications.  Further, we are not actually looking for sentences that \emph{contain} the answer choice, as with answer sentence selection, but rather sentences which close the "lexical chasm" \cite{Berger:00} between question and answer (demonstrated in the example in Table \ref{tab:question_example}). 


%Related work: QA, multiple choice, kaggle challenge, explanation-centered inference, model-specific work (NNs, etc)

%Latest QA papers (latest trends, attn, etc)
%Inspired by this literature (cite DAN \& SNLI, simpler better), but we add this latent layer they don’t have (justification quality). For simple archs: see Danqi Chen at ACL 2016 (similar to DAN but for reading comprehension). 

%Attention models (ACL 2016). Simple especially when you don’t have enough data.

%(visualizations vs analysis of embeddings VS correlated metric EMNLP paper, etc -- interpreting wts if DL is impossible, rather, find correlations betw those and explanatory natural language)

 
%\todo{Add a paragraph comparing against the task of answer sentence selection (see that reviewer from CL, and all those kernel-based papers by Moschitti). The difference is that in our case the answer and its justification arrive from different sources (i.e., the answer is provided in the multiple-choice exam, whereas the justification is extracted from study guides), whereas in previous answer sentence selection work the answer is included in the sentence. In our case we have to deal with bigger ``lexical chasm'' between question/answer/justification.}

%\todo{Discriminative information retrieval for question answering sentence selection(Chen and Van-Durme): Presented a method that selects sentences which contain potential answers for questions from a very large corpus (10\^7 sentences, requiring several thousand questions for training). Their results are dramatically better than Lucene across two datasets and several evaluation measures.}
%(Yih et al.,2013; Wang and Manning, 2010; Heilman and Smith, 2010; Yao et al., 2013a) and recently using neural networks (Yu et al., 2014; Severyn and Moschitti,2015; Wang and Nyberg, 2015; Yin et al.,2016)

%Answer sentence selection diff because we don't need (or want?) complete answer inclusion.

%Our sentences being selected are unlabeled.

%TAG paper!!! 

\begin{figure}[t]
\begin{center}
\includegraphics[width=0.5\textwidth]{mainmatter/emnlp2017-qaj/arch_overall.png}
\caption{ Architecture of our question answering approach.  
Given a question, candidate answer, and a free-text knowledge base as inputs, we generate a pool of candidate justifications, from which we extract feature vectors.  We use a neural network to score each and then use max-pooling to select the current best justification. This serves as the score for the candidate answer itself.  The red border indicates the components that are trained online. }
\label{fig:arch_overall}
\vspace{-5mm}
\end{center}
\end{figure}

\section{Approach}
\label{sec:approach}
One of the primary difficulties with the explainable QA task addressed here is that, while we have supervision for the correct answer, we do not have annotated answer justifications.  
Here we tackle this challenge by using the QA task performance as supervision for the justification reranking, allowing us to 
%extending a neural QA model to jointly learn both how to 
%jointly 
learn to choose both the correct answer and a compelling, human-readable justification for that answer.

Additionally, similar to the strategy Chen and Manning~\citeyear{chen2014fast} applied to parsing, we combine representation-based features with explicit features that capture additional information that is difficult to model through embeddings, especially with limited training data.
%second contribution is that, similar to the strategy Chen and Manning~\citeyear{chen2014fast} applied to parsing, we combine representation-based features with explicit features that capture additional information that is difficult to model through embeddings.



% ms: avoid "system" too engineering-y
The architecture of our approach is summarized in Figure \ref{fig:arch_overall}.  
Given a question and a candidate answer, we first query an textual knowledge base (KB) to retrieve a pool of potential justifications for that answer candidate.  
For each justification, we extract a set of features designed to model the relations between questions, answers, and answer justifications based on word embeddings, lexical overlap with the question and answer candidate, discourse, and information retrieval (IR) (Section \ref{sec:features}).
These features are passed into a simple neural network to generate a score for each justification, given the current state of the model.  A final max-pooling layer selects the top-scoring justification for the candidate answer and this max score is used also as the score for the answer candidate.  
The system is trained using correct-incorrect answer pairs with a pairwise margin ranking loss objective function to enforce that the correct answer be ranked higher than any of the incorrect answers. 

%The key here is that we use the current state of the model to select the best justification for a given answer candidate from a pool of many candidate justifications.  To do this, we modify the training procedure such that at the start of each epoch \todo{minibatch instead of epoch?}, we first compute a forward pass with each candidate justification to find the top-scoring justification for each candidate answer.
%For a given question, answer candidate, and justification, we combine features based on word embeddings, lexical overlap, discourse, and information retrieval (IR) together in a simple neural architecture to generate a score for the answer candidate.    We then use this selected justification to calculate our gradients for updating the model parameters.  

With this end-to-end approach, the model learns to select justifications that allow it to correctly answer questions.  We hypothesize that this approach enables the model to indirectly learn to choose justifications that provide good explanations as to why the answer is correct. We empirically test this hypothesis in Section \ref{sec:results}, where we show that indeed the model learns to correctly answer questions, as well as to select high-quality justifications for those answers. 
% ms: misleading; it reads as if answer selection is better than IR
% better than a strong IR baseline. 

\input{models}

%\section{Experiments and Evaluation}
%\label{sec:experiments}
%%\vspace{-2mm}
%
%We evaluate our approach at two levels.  First, we directly evaluate the quality of the embedding space in regards to the desired semantic relation.  Then, we evaluate the utility of the approach in an open-domain QA task.

\section{Direct Evaluation}
\label{sec:directeval}

To determine whether or not the proposed approach is able to capture the semantic relation of interest, causality, we replicate the quantitative evaluation of Levy and Goldberg~\shortcite{levy2014dependency}.  To demonstrate that their embeddings encoded functional similarity as opposed to topical similarity, they used their embeddings to rank a set of word pairs (each of which reflected one of the two types of similarity) using cosine similarity and showed that the word pairs which were functionally similar tended to be ranked higher than those with topical similarity.  Here, we do the same.

%\flushleft{\textbf{Data:}} 
\subsection{Data:}
We evaluate on a set of word pairs drawn from the SemEval 2010 Task 8 \todo{cite}, a multiway classification of semantic relations between nominals.  From the training set of this task, we used a total of XX nominal pairs, XX of which were cause-effect and XX which were randomly selected from the other XX relations.  This set was then randomly divided into equally-sized development and test partitions.

%\flushleft{\textbf{Baselines:}} 
\subsection{Baselines:}
We compared our embeddings against a standard \texttt{word2vec} model with a sliding window of XX as well as a random baseline where pairs were randomly shuffled. Additionally, we compared against a look-up baseline which counted the number of times a given nominal pair was found in the extracted cause-effect event database. 

\begin{table}[t!]
\begin{center}
%\begin{scriptsize}
\begin{footnotesize}
\begin{tabular}{ll}
\hline
\multicolumn{1}{l}{ Model } & \multicolumn{1}{l}{MAP} \\ %\multicolumn{1}{l}{Impr.} \\
%\cline{1-2}

\hline
%\multicolumn{2}{l}{\textit{Yahoo! Answers}} \\ % 185q (sent) ret=1p c=0.1 
%\hline
Random 			& 48.9 	\\
Lookup			& 93.1 	\\
Standard  		& 58.4	\\
Casual  			& 68.2*	\\

\end{tabular}
\end{footnotesize}
\caption{{\small Ability of each of the models and baselines to rank causal nominal pairs above non-causal pairs, as measured with mean average precision (MAP). Statistical significance (indicated by *) was determined through bootstrap resampling with 10,000 iterations.}}
\label{tab:MAP}
\end{center}
\end{table}
%MAP for Custom Vectors: 0.6816675505751166
%MAP for E2C Vectors: 0.6871338630607531
%MAP for Bidir Vectors: 0.6684350003582593
%MAP for Comparison (Baseline) Vectors: 0.5835158858435683
%MAP for Translation Model with lamda of 0.5 : 0.6156522257806402
%MAP for counting Matches: 0.9312613087523468
%MAP for Keras: 0.6752727545259546
%MAP for Random: 0.4892479543525109
%p < 0.01

\begin{figure}[t!]
\begin{center}
%\includegraphics[width=75mm]{rpcurves_all.png}
%\vspace{-4mm}
\caption{{\small Recall-precision curve showing the ability of each model to rank causal pairs above non-causal pairs. }}
\vspace{-6mm}
\label{fig:rpcurve_all}
\end{center}
\end{figure}

%\flushleft{\textbf{Results:}} 
\subsection{Results:}
In Table \ref{tab:MAP} we report the mean average precision (MAP) for each of the models and baselines.  The highest by far is for the look-up baseline.  However, this score is positively skewed by the fact that when pairs are found, there is extremely high confidence that they are of the relation of interest coupled with the fact that approximately 65\% of the pairs were not found in the database and so were all tied in last place.  As a consequence, there were far fewer average precisions to be combined when calculating the MAP, and most of the ones which were there had extremely high precision.  The MAP when using the customized vectors was significantly higher than that of the standard \texttt{word2vec} vectors (68\% versus 58\%), and both were higher than the baseline.  This suggests that while the standard implementation of \texttt{word2vec} encodes some causality information, our method encodes it far more directly. \todo{better word}.

\subsection{Discussion:}
We also examined the recall-precision curves for all models, shown in Figure \ref{fig:rpcurve}.  The curve for the customized vectors shows an atypical shape, where, despite the success of the customized vectors once recall reaches approximately 15\%, the highest ranked pairs (i.e. lower recall) had far \emph{worse} precision rather than higher precision.  To examine this, we analyzed the top-ranked 15\% of the pairs from the causal vectors.  

\todo{HEDGE THIS - more?}
Rather than finding that the high associations were due to noise in the training data, we found that the model appeared to be performing approximate inference, but here it was noisy inference.  For example, the top-ranked pair was (\emph{platform}, \emph{scaffold}) and there were no instances in the extracted causal events where \emph{platform} and \emph{scaffold} were found together in the same cause-effect pair.  

Instead, we found that there were only three extracted events with \emph{scaffold} in the effect text.  We checked to see if the words in the cause text of those events had other effect texts which overlapped lexically with \emph{platform}.  Indeed, we found the link illustrated in Figure \ref{fig:noisyinf}, where both \emph{platform} and \emph{malfunction} cause \emph{loss} thus bringing them closer together in the target embedding space, since they cause the same things.  This resulted in \emph{platform} being close to the effects of \emph{malfunction}, including \emph{scaffold}.  This demonstrates that the inference is influenced by frequency effects, as words like \emph{scaffold} and \emph{platform} are too infrequent to have robust representations in the embedding space.  

As this effect is entirely directional, we trained a set of vectors by reversing the input, such that the effects served as the targets and the causes were the contexts.  The recall-precision curve when using these embeddings to rank the SemEval pairs is shown in Figure \ref{fig:rpcurve_all}, labelled as E2C.  As expected, the curve follows that of the original vectors, as it suffers from the same issues, but with different noisy pairs ranked highly.  We then ranked the SemEval pairs using an average of the scores returned by the two causal embeddings, to mitigate the frequency effects.  That final, bidirectional curve is also shown in Figure \ref{fig:rpcurve_all}.
%Instead, we found that many of the causal events which whose cause argument contained the word \emph{platform} had effects which overlapped lexically with those whose cause arguments contained the word \emph{malfunction}, and that there w.  To illustrate, consider the following sentences 
%we had several extracted causal events whose cause contained the word \emph{platform} and whose effect contained the words \emph{}

\section{Indirect Evaluation}
\label{sec:indirecteval}

After having determined that the learned embeddings encoded causality, we designed an experiment to evaluate their utility in a down-stream task.  We chose question-answering (QA) as it has been shown that many questions require complex inference, including causality, in order to be explanably solved \todo{cite}.  

\subsection{Data:}
As we trained our embeddings using events extracted from open-domain resources, we chose to evaluate on a set of causal questions from Yahoo! Answers \todo{link}, hand-extracted with very simple surface patterns such as \emph{What causes ...}\todo{footnote with details?}.  There were a total of \todo{XX} questions, from which we used \todo{50\%} for training, \todo{25\%} for development, and \todo{25\%} for testing.
These questions were filtered such that each had at least four candidate answers
\todo{detail about the avg num answers, etc}

\subsection{Experimental Design:}
We evaluate the contribution of the causal embeddings model using the standard reranking architecture \todo{cite} of \todo{cite naacl and peters acl}.
%In pilot experiments, we found that aligning only nouns, verbs, adjectives, and adverbs yielded higher performance. TALK ABOUT?
In this architecture, the candidate answers are initially ranked using a shallow candidate retrieval (CR) component, then they are re-ranked using the model features along with the CR score. 
We used SVM rank\todo{cite/link}, a Support Vector Machine adapted for ranking, for our learning framework.
We compare the performance of the causal embeddings model against the same standard skipgram model used in Section \ref{sec:directeval}.
As features, we use the same set as \todo{cite naacl and acl}: the maximum and average pairwise cosine similarity between question and answer words, as well as the overall similarity between the composite question and answer vectors.  When using the causal embeddings however, we first determine (through lexical pattern matching) whether the question text is the cause or the effect, thus determining which embeddings to use for the question and candidate answer texts.  For example, in a question such as "Q: What causes X? A: Y", the cosine similarities would be found using the effect vectors for the question words and the cause vectors for the answer candidate words. \todo{good grief, redo this}   


\input{results}
%\begin{table}[!th]
%\begin{center}
%\begin{footnotesize}
%\hfill
%\begin{tabular}{ll}
%\hline
%Error Type & Percent \\ 
%\hline
%Short justification/High lexical overlap  & 53.3\%\\
%Complex inference required   & 43.3\% \\
%Knowledge Base Noise  & 6.7\% \\
%Word order necessary	 & 6.7\% \\
%Coverage & 6.7\% \\
%Negation	& 3.3\% \\
%Other & 6.7\% \\
%\end{tabular}
%\end{footnotesize}
%\caption{{\footnotesize Summary of the findings of the 30 question error analysis.  
%%Examples of several categories provided in separate tables. 
%Note that a given question may fall into more than one category.}} 
%\label{tab:erroranalysis}
%\vspace{-5mm}
%\end{center}
%\end{table}


%\begin{table}[t]
%\begin{center}
%\begin{footnotesize}
%\begin{tabular}{p{1cm}p{6cm}}
%\hline
%Type: & \textbf{Short justification/High lexical overlap}\\
%\hline
%Question: & The length of time between night and day on Earth varies throughout the year. This time variance is explained primarily by $\rule{1cm}{0.15mm}$. \\
%Correct: & Earth 's angle of tilt \\
%			 & \textit{ ... the days are very short in the winter because the sun's rays hit the earth at an extreme angle ... due to the tilt of the earth's axis. } \\
%Chosen: &  Earth 's distance from the Sun \\
%			& \textit{ Is light year time or distance? Distance}	\\
%\end{tabular}
%\hfill
%\end{footnotesize}
%\caption{{\footnotesize Example of the system preferring a justification for which all the terms were found in either the question or answer candidate. (Justifications shown in italics)
%}} 
%\label{tab:ex_lex_overlap}
%\vspace{-5mm}
%\end{center}
%\end{table}

\subsection{Error Analysis}
\label{sec:erroranalysis}

We performed an error analysis of 30 incorrectly answered questions, %summarized in Table \ref{tab:erroranalysis},
  examining the top 5 justifications returned for both the correct and chosen answers of each.    
%Among the questions analyzed we found some interesting trends.   
Notably, 50\% of the questions had one or more \emph{Good} justifications. %, but the system incorrectly ranked another answer's justification higher.  
The most common form of error (53.3\%) was due to the system's preference for short justifications with a large degree of lexical overlap with the question and answer choice, %, particularly when the correct answer required more "explanation" to connect the question to the answer.  
% shown by the example in Table \ref{tab:ex_lex_overlap}.  
 %This effect was magnified when the correct answer required more "explanation" to connect the question to the answer.  
suggesting the system has learned that generally many unmatched words are indicative of an incorrect answer.  %While this may typically be true, extending the system to be able to prefer the \emph{opposite} with certain types of questions would potentially help with these errors.  
The second largest source of errors (43.3\%) came from questions requiring complex inference (causal, process, quantitative, or model-based reasoning), demonstrating the difficulty of the question set and the need for systems that can robustly handle a variety of question types.
Aside from these main groups, there were some smaller trends including KB noise, and our system's lack of using word order and recognizing negation.\footnote{A much more detailed error analysis is available, and if this paper is accepted will be included.}
 
% as with the question:%.  For example, to answer the question:
 %\begin{quote}
%\begin{addmargin}[1em]{2em}% 1em left, 2em right 
% \begin{footnotesize}
%  \textit{Q: Mr. Harris mows his lawn ...[and leaves] the clippings on the ground. Which long term effect will this most likely have on his lawn? \\
%  A: It will provide the lawn with needed nutrients.}
% \end{footnotesize}
%%\end{quote}
%\end{addmargin}
%To answer this, you would need to link together: \textit{cut grass left on the ground $\rightarrow$ grass decomposes $\rightarrow$ decomposed material provides nutrients}. 
%These questions constitute a large portion of our errors,   demonstrating not only the difficulty of the question set but also the need for systems that can robustly handle a variety of question types and their corresponding information needs.  

%Aside from these main groups, there were some smaller trends including:  
%7\% of the incorrectly chosen answers actually had justifications which "validated" them due to KB noise, 7\% required word-order to answer (e.g., \emph{X divided by Y} vs. \emph{Y divided by X}), %another 7\% of questions suffered from lack of coverage of the question concept in the knowledge base,
%%(see example in Table \ref{tab:ex_coverage}), 
% and 3\% failed to appropriately handle negation (i.e., questions of the format \emph{Which of the following are NOT ...}). 


% bs: no room: - Negative results?


%\begin{table}[t]
%\begin{center}
%\begin{footnotesize}
%\begin{tabular}{p{1cm}p{6cm}}
%\hline
%Type: & \textbf{Coverage}\\
%\hline
%Question: & Which activity most effectively ensures the proper functioning of osteocytes? \\
%Correct: & consuming mineral-rich foods\\
%			& \textit{ most lipids consumed from food are in the form of triglycerids}	\\	
%Chosen & increasing the respiratory rate 	\\
%			& \textit{ hyperventilation increased respiratory rate}	\\
%\end{tabular}
%\hfill
%\end{footnotesize}
%\caption{{\footnotesize Example of a question for which coverage was an issue.  The KB had no coverage for the concept of \emph{osteocyte}.}} % , so the system grasped at proverbial straws.}} 
%\label{tab:ex_coverage}
%\end{center}
%\end{table}


%\begin{table}[!th]
%\begin{center}
%\begin{footnotesize}
%\begin{tabular}{p{1cm}p{6cm}}
%\hline
%Type: & \textbf{Complex inference required}\\
%\hline
%Question: & Mr. Harris mows his lawn twice each month. He claims that it is better to leave the clippings on the ground. Which long term effect will this most likely have on his lawn? \\
%Correct: &  It will provide the lawn with needed nutrients. 	\\
%\end{tabular}
%\hfill
%\end{footnotesize}
%\caption{{\footnotesize Example of a question for which complex inference is required.  In order to answer the question, you would need to assemble the following chain of events: cut grass left on the ground $\rightarrow$ grass decomposes $\rightarrow$ decomposed material provides nutrients.}} 
%\label{tab:ex_complex_inf}
%\end{center}
%\end{table}


\section{Conclusion}
\label{sec:conclusion}

We have proposed an approach for QA where producing human-readable justifications for answers, and evaluating answer justification quality, is the critical component.
Our interdisciplinary approach to building and evaluating answer justifications includes cognitively-inspired aspects, such as making use of psycholinguistic concreteness norms for focus word extraction, and making use of age-appropriate knowledge bases, which together help move our approach towards approximating the qualities of human inference on the task of question answering for science exams. Intuitively, our structured representations for answer justifications can be interpreted as a robust approximation of more formal representations, such as logic forms~\cite{moldovan2001logic}. However, our approach does not evaluate the quality of connections in these structures by their ability to complete a logic proof, but through a reranking model that measures their correlations with good answers.

In our quest for explainability, we have designed a system that generates answer justifications by chaining sentences together. Our experiments showed that this approach improves explainability, and, at the same time, answers questions out of reach of information retrieval systems, or systems that process contiguous text.  
We evaluated our approach on 1,000 multiple-choice questions from elementary school science exams, and experimentally demonstrated that our method outperforms several strong baselines at both selecting correct answers, and producing compelling human-readable justifications for those answers.  We further validated our three critical contributions: (a) modeling the high-level task of determining justification quality by using a latent variable model is important for identifying both correct answers and good justifications, (b) identifying focus words using psycholinguistic concreteness norms similarly benefits QA for elementary science exams, and (c) modeling the syntactic and lexical structure of answer justifications allows good justifications to be assembled and detected. 

We performed a detailed error analysis that suggests several important directions for future work. 
First, though the majority of errors can be addressed within the proposed formalism and by improving focus word extraction, 47.5\% of incorrectly answered questions would also benefit from more complex inference mechanisms, ranging from causal and process reasoning, to modeling quantifiers and negation.
This suggests that our robust approach for answer justification may complement deep reasoning methods for QA in the scientific domain~\cite{baral2011towards}.
Second, our text aggregation graphs currently capture intersentence connections solely through lexical overlap. We hypothesize that extending these structures to capture lexical-semantic overlap driven by word embeddings~\cite{mikolov13}, which have been demonstrated to be beneficial for QA~\cite{yih13,jansen14,fried2015higher}, would also be beneficial here, and increase robustness on small knowledge bases, where exact lexical matching is often not possible. 
Finally, while our answer justifications are currently short, future justifications might be quite long, and aggregate sentences from knowledge bases of different domains and genres.  In these situations, combining our procedure for constructing justifications with methods that improve text coherence~\cite{barzilay2008modeling} would likely improve the overall user experience for reading and making use of answer justifications from automated QA systems. 

To increase reproducibility, all the code behind this effort is released as open-source software\footnote{\url{https://github.com/clulab/releases/tree/master/cl2017-qa}}, which allows other researchers to use our entire science QA system as is, or to explore adapting the various components to other tasks. 

\FloatBarrier


%\section*{Acknowledgments}

\bibliography{refs}
\bibliographystyle{emnlp_natbib}

\end{document}
