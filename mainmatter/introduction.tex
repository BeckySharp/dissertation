\chapter{INTRODUCTION\label{chapter:introduction}}

\address{Start by explaining why NLI and QA are fundamental NLP tasks. Give examples to search, and personal assistants.}

\address{Natural language inference (NLI)}

\address{while FOL methods are attractive for their formality, it is this same formality that renders them too brittle to be of much use outside of small, toy domains.}

\address{Here we focus on approximating this inference using natural language instead of formal representations (e.g. ccg parses, etc)}

\address{methods that can be applied in much broader domains, with an emphasis on semi- and distant supervision, rather than needing a large amount of hand-generated training data.}

\address{With the relaxation of the formality, the danger is in losing the explainability, and so we use/include models that use a human-readable intermediate output generated by the model to provide an explanation for the inference performed by the model.}

%\address{We try/test our NLI methods in the domain of Question answering (QA):}

% \address{QA hard bc of}

% emnlp2016 intro
Question answering (QA), i.e., finding short answers to natural language questions, is one of the most important but challenging 
tasks on the road towards natural language understanding~\cite{Etzioni:11}. 
%\address{Require bridging lexical chasm}
Unlike search or information retrieval, answers infrequently contain lexical overlap with the question (e.g. {\em What should we eat for breakfast? -- Zoe's Diner has good pancakes}), and require QA models to draw upon more complex methods to bridge this "lexical chasm" \cite{Berger:00}.  These methods range from robust shallow models based on lexical semantics, to deeper, explainably-correct, but much more brittle inference methods based on first order logic.  

%\address{One approach to bridging lexical chasm is use of monolingual alignment models, but: }

%\address{these are expensive to train (need lots of gold QA pairs)}
% naacl2015 intro
Berger et al.~\citeyear{Berger:00} proposed that this "lexical chasm" might be partially bridged by repurposing statistical machine translation (SMT) models for QA. Instead of translating text from one language to another, these monolingual alignment models learn to translate from question to answer\footnote{In practice, alignment for QA is often done from answer to question, as answers tend to be longer and provide more opportunity for association~\cite{Surdeanu:11}.}, learning common associations from question terms such as {\em eat} or {\em breakfast} to answer terms like {\em kitchen, pancakes, or cereal}.

While monolingual alignment models have enjoyed a good deal of recent success in QA (see related work), they have expensive training data requirements,  
requiring a large set of aligned in-domain question-answer pairs for training.
%In most domains these pairs are expensive to generate, and one of the current methodological challenges in QA is locating or building high-quality QA pairs for training and testing. Even large open-domain international evaluations and workshops such as the Text REtrieval Conference (TREC)\footnote{\url{http://trec.nist.gov}} and the Cross Language Evaluation Forum (CLEF),\footnote{\url{http://www.clef-initiative.eu}} are often limited to sets of a few hundred factoid questions, many of which are highly related.  As a result, for open domain QA one often makes use of Community Question Answering (CQA) data from websites such as Yahoo! Answers or Stack Overflow, which offer tens of thousands of questions, but of highly variable quality.  
For low-resource languages or specialized domains like science or biology, often the only option is to enlist a domain expert to generate gold QA pairs --  a process that is both expensive and time consuming.  All of this means that only in rare cases are we accorded the luxury of having enough high-quality QA pairs to properly train an alignment model, and so these models are often underutilized or left struggling for resources. 

%\address{We propose a method for generating artificial training data (can be thought of as a form of distant supervision?? If you squint?) (NAACL2015)}
% naacl2015 intro
Making use of recent advancements in discourse parsing \cite{feng12}, in Chapter \ref{chapter:naacl2015} we address this issue, and investigate whether alignment models for QA can be trained from artificial question-answer pairs generated from discourse structures imposed on free text. \todo{work in that can be thought of as a form of distant supervision?? If you squint?}
% by imposing structure on inexpensive free text resources instead of using QA pairs.  
We evaluate our methods on two corpora, generating alignment models for an open-domain community QA task using Gigaword\footnote{LDC catalog number LDC2012T21}, and for a biology-domain QA task using a biology textbook. 


\address{Diff info needs and so the “manner” by which the lexical chasm is bridged should hopefully be robust to that}

This alignment approach for QA can be considered as falling into a larger group of approaches which prefer answers that are closely related to the question, where the relatedness is determined by the associations of the alignment model or by associations provided by other lexical semantic models such as word embeddings~\cite{yih13,jansen14,fried2015higher}. 
While appealing for its robustness to natural language variation, this one-size-fits-all category of approaches does not take into account the wide range of distinct question types that can appear in any given question set, and that are best addressed individually~\cite{chu2004ibm,ferrucci2010building,clark2013study}.  

%These don’t address diff types of inference/info needs:
%\address{We propose a framework for learning customized alignments/associations for a specific info need using semi-supervised methods (EMNLP2016)}
Given the variety of question types, we suggest that a better approach is to look for answers that are related to the question \emph{through the appropriate relation}, e.g., a causal question should have a cause-effect relation with its answer.
Adopting this view, and working with embeddings as a mechanism for assessing relationship, we address a key question: how do we train and use task-specific embeddings cost-effectively? 
In Chapter \ref{chapter:emnlp2016}, using causality as a use case, we answer this question with a framework for producing causal word embeddings with minimal supervision, and a demonstration that such task-specific embeddings significantly benefit causal QA. 


\address{BUT, these shallower methods lose some of the explainability (get only a set of associations and weights on association features), so we develop approaches that focus on explanation, aggregation, and robustness}

\address{two approaches (one more structured, one with learned representations) for answering science questions and providing compelling, human-readable explanations for the answers.}

\address{Latent layer/intermediate output learned during training which correlates with what the model is learning}

% ------ pasted from emnlp2017-------------
\todo{I don't think this text really fits here at all - perhaps move back to the paper/chapter intro and write something else for here}
\textcolor{orange}{
Developing interpretable machine learning (ML) models, that is, models where a human user can \emph{understand} what the model is learning, is considered by many to be crucial for ensuring usability and accelerating progress \cite{craven1996extracting,Kim2015MindTG, letham2015interpretable, Ribeiro2016WhySI}.  
% bs: removed for space, we talk more about this in related work
%As such, it has received much attention in recent years, especially as deep learning and complex architectures have seen dramatic gains in many tasks.
For many applications of question answering (QA), i.e., finding short answers to natural language questions, simply providing an answer is not sufficient. A complete approach must be interpretable, i.e., able to {\em explain} why an answer is correct. 
For example, in the medical domain, a QA approach that answers treatment questions would not be trusted if the treatment recommendation is not explained in terms that can be understood by the human user. }

\textcolor{orange}{One approach to interpreting complex models is to make use of human-interpretable information % metric % ms: not a metric...
 generated by the model to gain insight into what the model is learning.  This can be an intermediate representation used by the model, as with the model-generated text spans of \citet{Lei2016RationalizingNP}, that serve as input to another classification network.  
By learning these intermediate representations end-to-end with a downstream task, they are optimized to correlate with what the model learns is discriminatory for the task, and they can be evaluated against what a human would consider to be important.
\todo{need to define what a downstream task means to use this term}, they are optimized to correlate with what the model learns is discriminatory for the task, and they can be evaluated against what a human would consider to be important.
Here we apply this general framework for model interpretability to QA.}


\begin{table}[t]
\begin{center}
\begin{tabularx}{\linewidth}{p{1cm}p{13cm}}
\multicolumn{2}{p{15cm}}{\textbf{Question:} Which of these is a response to an internal stimulus?} \\
 (A) & A sunflower turns to face the rising sun. \\
 (B) & A cucumber tendril wraps around a wire. \\
 (C) &  A pine tree knocked sideways in a landslide grows upward in a bend. \\
 (\textbf{D}) &\textbf{Guard cells of a tomato plant leaf close when there is little water in the roots .} \\
\\
\multicolumn{2}{p{15cm}}{\textbf{Justification:} 
Plants rely on hormones to send signals within the plant in order to respond to internal stimuli such as a lack of water or nutrients. } \\
\end{tabularx}

\caption{{  Example of an 8th grade science question with a justification for the correct answer.  Note the lack of direct lexical overlap present between the justification and the correct answer, demonstrating the difficulty of the task of finding justifications using traditional distant supervision methods. }}

\label{tab:question_example}
\end{center}
\end{table}

\textcolor{orange}{In this work, we focus on answering multiple-choice science exam questions (Clark \citeyear{clark:2015}; see example in Table~\ref{tab:question_example}). 
This domain is challenging as: (a) approximately 70\% of science exam question shave been shown to require complex forms of inference to solve \cite{clark:2013,jansen-EtAl:2016:COLING}, and (b) there are few structured knowledge bases to support this inference.  
Within this domain, we propose an approach that learns to both select and explain answers, when the only supervision available is for which answer is correct (but not how to explain it).
Intuitively, our approach chooses the justifications that provide the most help towards ranking the correct answers higher than incorrect ones.
More formally, our neural network approach alternates between using the current model with max-pooling to choose the highest scoring justifications for correct answers, and optimizing the answer ranking model given these justifications. 
Crucially, these reranked texts serve as our human-readable answer justifications, and by examining them, we gain insight into what the model learned was useful for the QA task.  }



%------end paste from emnlp 2017--------------------------


\address{Applied to science MCQA }

\address{Description (clark and jansen stuff)}

\address{Lure answers}

To encourage a emphasis on the task of explainable inference for question answering (QA), \citet{clark:2015} introduced the Aristo challenge, a QA task focused on developing methods of automated inference capable of passing standardized elementary school science exams, while also providing human-readable explanations (or justifications) for those answers.  Science exams are an important proving ground for QA because inference is often required to arrive at a correct answer, and, commonly, incorrect answers that are high semantic associates of either the question or correct answer are included to ``lure'' students (or automated methods) away from the correct response.
%
%  In spite of being multiple choice, these questions tend to be  more challenging than factoid QA, which is highly amenable to retrieval-based models that operate over large structured knowledge bases such as Freebase (e.g. \note{Liang?, etc}). \todo{The previous sentence must be defended.} Multiple choice exams also commonly contain "lure" answers -- incorrect answers that are high semantic associates of either the question or correct answer -- which further reduce the efficacy of retrieval or lexical semantic/associative methods. 
%- Semantic Parsing for QA / Freebase (Percy Liang), much easier problem
%
%-- Science exams as a challenging QA problem
%- More than just factoid lookup, despite being multiple choice
%- Interesting proving ground for QA -- questions are challenging, and well-constructed multiple-choice questions typically have lure answers that are incorrect but are highly associated with either the question or correct answer, making shallow methods difficult (REF). 
%
Adding to the challenge, not only is inference required for science exams, but several kinds of inference are present.
In an analysis of three years of standardized science exams, \citet{clark:2013} identified three main categories of questions based on the methods likely required to answer them correctly. Examples of these questions can be seen in Table~\ref{tab:inferenceexamples}, highlighting that 65\% of questions require some form of inference to be answered.

%
% Justification example (IR)
%
\begin{table*}[t]
\caption{ 
Categories of questions and their relative frequencies as identified by \citet{clark:2013}. Retrieval-based questions (including \emph{is--a}, dictionary definition, and property identification questions) tend to be answerable using information retrieval methods over structured knowledge bases, including taxonomies and dictionaries. 
More complex general inference questions make use of either simple inference rules that apply to a particular situation, a knowledge of causality, or a knowledge of simple processes (such as \emph{solids melt when heated}).
Difficult model-based reasoning questions require a domain-specific model of how a process works, like how gravity causes planets to orbit stars, in order to be correctly answered.
Note here that we do not include diagram questions, as they require specialized spatial reasoning that is beyond the scope of this work. 
}
\begin{center}

\small

\begin{tabularx}{\textwidth}{p{2cm}p{5cm}p{5.9cm}}
\hline
Category &	\multicolumn{2}{l}{Example} \\
\hline
Retrieval	&	\multicolumn{2}{l}{Q: The movement of soil by wind or water is called:} \\
(35\%)		&   (A) condensation   	&	(B) evaporation   \\
			&	(C) erosion   		&	(D) friction \\
\\
General 	&	\multicolumn{2}{l}{Q: Which example describes an organism taking in nutrients?} \\
Inference	&   (A) A dog burying a bone			&   (B) A girl eating an apple	\\
(39\%)		&	(C) An insect crawling on a leaf	&  (D) A boy planting tomatoes in the garden  \\
\\
Model-based & 	\multicolumn{2}{l}{Q: When a baby shakes a rattle, it makes a noise. Which form of energy was} \\
Inference	& 	\multicolumn{2}{l}{changed to sound energy?} \\
(26\%)		&	(A) electrical	&   (B) light   \\
			&	(C) mechanical	&   (D) heat  \\
			
\end{tabularx}



\label{tab:inferenceexamples}
%space{-5mm}
\end{center}
\end{table*}


%-- Approximate Inference/Continuum
%- benefits/drawbacks
%- Alignment/Lexical semantic end: Robust but lacks justifications
%- First-order Logic end: Brittle, but provably correct justifications
%- Meeting in the middle (relax FOL, or impose more structure into LS)


%-- Justifications as central to question answering
%- In many applications, answering without justifications is pointless
%- Example (medical -- you need surgery, but not explain why)
%- Model QA as generating and evaluating explanations/justifications for why a particular answer candidate is correct

We propose a QA approach that jointly addresses answer extraction and justification.
We believe that justifying why the QA algorithm believes an answer is correct is, in many cases, a critical part of the QA process.
% -- in some cases perhaps more important than the answer itself.  
For example, in the medical domain, a user would not trust a system that recommends invasive procedures without giving a justification as to why (e.g., ``Smith (2005) found procedure \emph{X} healed 90\% of patients with heart disease who also had secondary pulmonary complications'').  A QA tool is clearly more useful when its human user can identify both when it functions correctly, and when it delivers an incorrect or misleading result -- especially in situations where incorrect results carry a high cost.  


%-- Reframing QA as a generating and evaluating explanations/justifications for why a particular answer candidate is correct

To address these issues, here we reframe QA from the task of scoring (or reranking) answers to 
a process of \emph{generating and evaluating justifications} for why a particular answer candidate is correct. 
We focus on answering science exam questions, where many questions require complex inference, and building and evaluating answer justifications is challenging. 

\address{To get a complete and valid explanation for selection of answer choice, may need to aggregate info from multiple distinct resources}

\address{For aggregation, to prevent semantic drift, use structured representations (parts of CL2017)}
In particular, we construct justifications by {\em aggregating} multiple sentences from a number of textual knowledge bases (e.g., study guides, science dictionaries), which, in combination, aim to explain the answer.
We then rank these candidate justifications based on a number of measures designed to assess how well integrated, relevant, and on-topic a given justification is, and select the answer that corresponds to the highest-ranked justification.










\address{Robustness - shallower, no structure, learned representations (EMNLP2017-hopeful)}



% ----------CONTRIBUTIONS-
The contributions of this work are:
\begin{enumerate}

% naacl 2015
\item We demonstrate that by exploiting the discourse structure of free text, monolingual alignment models can be trained to surpass the performance of models built from expensive in-domain question-answer pairs. 

\item We compare two methods of discourse parsing: a simple sequential model, and a deep model based on Rhetorical Structure Theory (RST)~\cite{mann88}.  We show that the RST-based method captures within and across-sentence alignments and performs better than the sequential model, but the sequential model is an acceptable approximation when a discourse parser is not available.  

\item We evaluate the proposed methods on two corpora, including a low-resource domain where training data is expensive (biology).

\item We experimentally demonstrate that monolingual alignment models trained using our method considerably outperform state-of-the-art neural network language models in low resource domains.
\end{enumerate}

% emnlp 2016
{\flushleft {\bf (1)}} 
A methodology for generating causal embeddings cost-effectively by bootstrapping cause-effect pairs extracted from free text using a small set of seed patterns, e.g., {\em X causes Y}. 
%We propose a method to generate knowledge resources for causal questions 
%We demonstrate that knowledge resources for causal questions can be generated by bootstrapping cause-effect pairs extracted from free text using a small set of high-precision patterns, e.g., {\em X causes Y}. 
We then train dedicated embedding (as well as two other distributional similarity) models over this data. \citet{levy2014dependency} have modified the algorithm of\citet{mikolov2013distributed} to use an arbitrary, rather than linear, context. Here we make this context task-specific, i.e., the context of a cause is its effect.
%embedding models (as well as alignment and convolutional neural network models) over this data. 
Further, to mitigate sparsity and noise, our models are bidirectional, and noise aware (by incorporating the likelihood of noise in the training process). 
%We achieve the latter by weighting the examples based on the likelihood that they are truly causal rather than simply associative. 

{\flushleft {\bf (2)}} The insight that QA benefits from task-specific embeddings. % , and a demonstration that this approach significantly improves performance. 
We implement a QA system that uses the above causal embeddings to answer questions and demonstrate that they significantly improve performance over a strong baseline. Further, we show that causal embeddings encode complementary information to vanilla embeddings, even when trained from the same knowledge resources. 

{\flushleft {\bf (3)}} An analysis of direct vs. indirect evaluations for task-specific word embeddings. 
We evaluate our causal models both  {\em directly}, in terms of measuring their capacity to rank causally-related word pairs over word pairs of other relations, as well as {\em indirectly} in the downstream causal QA task. 
%Importantly, the above knowledge acquisition process is completely independent from these evaluation tasks, e.g., the objective function of the embedding model does not include any information from the QA task, which guarantees modularity. 
In both tasks, our analysis indicates that including causal models significantly improves performance. 
However, from the direct evaluation, it is difficult to estimate which models will perform best in real-world tasks. Our analysis re-enforces recent observations about the limitations of word similarity evaluations~\cite{faruqui2016problems}: we show that they have limited coverage and may align poorly with real-world tasks.

%{\flushleft {\bf (3)}} For causal QA, we show that causal embeddings encode complementary information to vanilla embeddings, even when trained from the same knowledge resources. 

%the models that include causal embeddings perform significantly better than the models that do not. Further, for causal QA, we show that causal embeddings are complementary to vanilla embeddings, underlining the complexity of this QA task, which must simultaneously capture causality and associations driven by distributional similarity. \todo{reword? seems like we're contradicting our earlier statement...}

%{\flushleft {\bf (4)}} Finally, we show that there are discrepancies between direct and indirect evaluations, i.e., no model performs best in both tasks. Our analysis re-enforces recent observations about the limitations of narrow word similarity evaluations~\cite{faruqui2016problems}, in that they both have limited coverage, and can poorly align with real world tasks.
% end paste from emnlp2016 -------------------------

The specific contributions of this work are:
\begin{enumerate}
\item We propose an end-to-end neural method for learning to answer questions and select a high-quality justification for those answers. 
Our approach re-ranks free-text answer justifications without the need for structured knowledge bases. 
With supervision only for the correct answers, we learn this re-ranking through a form of distant supervision -- i.e., the answer ranking supervises the justification re-ranking. 

\item We investigate two distinct categories of features in this ``little data'' domain: explicit features, and learned representations. We show that, with limited training, explicit features perform far better despite their simplicity. 

\item We demonstrate a large (+9\%) improvement in generating high-quality justifications over a strong information retrieval (IR) baseline, while maintaining near state-of-the-art performance on the multiple-choice science-exam QA task, demonstrating the success of the end-to-end strategy.
\end{enumerate}
% end paste from emnlp2017 -------------------------

% end paste from cl2017 --------------------------
The specific contributions of this work are: 

\begin{enumerate}
\item We propose a method to construct answer justifications through information aggregation (or fusion). 
In particular, we aggregate multiple sentences into hierarchical graph structures (called text aggregation graphs) that capture both intrasentence syntactic structures and intersentence lexical overlaps. 
Further, we model whether the intersentence lexical overlap is between contextually relevant keywords critical to the justification, or other words which may or may not be relevant. 
Our empirical analysis demonstrates that modeling the contextual relevance of intersentence connections is crucial for good performance.  These requirements highlight the fundamental differences between selecting a single answer sentence or short passage in an answer sentence selection task~\cite[inter alia]{Severyn:12,Severyn:13a,Severyn:13b}, and the task of generating complete answer justifications through information aggregation. 


\item 
We introduce a latent-variable ranking perceptron algorithm that learns to jointly rank answers and justifications, where the quality of justifications is modeled as the latent variable. 
%Manually annotating the quality of thousands of candidate answer justifications (per question) is an intractable task.  We demonstrate that it is possible to model answer justification quality as a latent variable, and extend a ranking perceptron to incorporate this latent information and learn to preferentially rerank high-quality answer justifications automatically. 

\item 
We evaluate our system on a large corpus of 1,000 elementary science exam questions from third to fifth grade, and demonstrate that our system significantly outperforms several strong learning-to-rank baselines at the task of choosing the correct answer.  Further, we manually annotate answer justifications provided by the best baseline model and our intersentence aggregation method, and show that the intersentence aggregation method produces good justifications for 57\% of questions answered correctly, significantly outperforming the best baseline method. 

\item Through an in-depth error analysis we show that most of the issues encountered by the intersentence aggregation method center on solvable surface issues rather than complex inference issues.  To our knowledge, this is the largest evaluation and most in-depth error analysis for explainable inference in the context of elementary science exams. 


\end{enumerate}

% end paste from cl2017 --------------------------

\section{Overview\label{sec:overview}}

chapter 2: common related work

chapter 3: naacl 2015

chapter 4: emnlp 2016 - causal

chapter 5: TAG

chapter 6: emnlp 2017 hopeful

chapter 7: discussion/conclusion