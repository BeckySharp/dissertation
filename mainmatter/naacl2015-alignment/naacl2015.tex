%
% File eacl2014.tex
%
% Contact g.bouma@rug.nl yannick.parmentier@univ-orleans.fr
%
% Based on the instruction file for ACL 2013 
% which in turns was based on the instruction files for previous 
% ACL and EACL conferences

%% Based on the instruction file for EACL 2006 by Eneko Agirre and Sergi Balari
%% and that of ACL 2008 by Joakim Nivre and Noah Smith

\documentclass[11pt,letterpaper]{article}
\usepackage{naaclhlt2015}
\usepackage{times}
\usepackage{latexsym}

%\documentclass[11pt]{article}
%\usepackage{acl2014}
%\usepackage{times}
%\usepackage{latexsym}
\usepackage{url}
\usepackage{graphicx}
\usepackage{color}
\usepackage{comment}
\usepackage{times}
\usepackage{amsmath,amsthm,amssymb}
\usepackage{multirow}
\usepackage{url}
\usepackage{verbatim}
\usepackage{caption}
\usepackage{subcaption}
\usepackage{fixltx2e}
%\usepackage[options]{natbib}
\usepackage{float}
%\usepackage{covington}
\usepackage{supertabular,booktabs}
%\usepackage[ruled,vlined,linesnumbered]{algorithm2e}
\usepackage{enumerate}
\usepackage{longtable}
\usepackage{afterpage}
\usepackage{array}
%\usepackage{setspace}
\captionsetup{font={footnotesize}}


%\special{papersize=210mm,297mm} % to avoid having to use "-t a4" with dvips 
\setlength\titlebox{5cm}  % You can expand the title box if you really have to

\newcommand{\svmr}{{SVM$^{rank}$}}
\newcommand{\code}[1]{{\tt {\small #1}}}
\newcommand{\qn}{{{\bf Q}$^\textbf{{\small N}}$}}
\newcommand{\ssa}{{{\scriptsize $^{*}$}}}
\newcommand{\todo}[1]{\textcolor{red}{#1}}

\title{Spinning Straw into Gold: Using Free Text to Train Monolingual Alignment Models for Non-factoid Question Answering}

\author{Rebecca Sharp\textsuperscript{1}, Peter Jansen\textsuperscript{1}, Mihai Surdeanu\textsuperscript{1}, {\textnormal {and}} Peter Clark\textsuperscript{2}  \\
  \textsuperscript{1} University of Arizona, Tucson, AZ, USA \\
  \textsuperscript{2} Allen Institute for Artificial Intelligence, Seattle, WA, USA \\
  {\tt \{bsharp,pajansen,msurdeanu\}@email.arizona.edu} \\
  {\tt peterc@allenai.org} \\
  \\}
%\author{Rebecca Sharp, Peter Jansen, {\textnormal {and}} Mihai Surdeanu \\
 % University of Arizona \\
 % Tucson, AZ, USA \\
 % {\tt \{bsharp,pajansen,msurdeanu\}} \\
 % {\tt @email.arizona.edu} 
 % \\}

%\date{}

\begin{document}
\maketitle


\begin{abstract}
Monolingual alignment models have been shown to boost the performance of question answering systems by "bridging the lexical chasm" between questions and answers.
%and associating question words like {\em breakfast} with answer words like {\em pancakes} or {\em cereal}.  %Unfortunately these models have been historically challenging to generate, requiring large and expensive corpora of aligned QA pairs to train.  
The main limitation of these approaches is that they require semistructured training data in the form of question-answer pairs, which is difficult to obtain in specialized domains or low-resource languages.
%While these models currently require semistructured training data in the form of question-answer pairs, 
We propose two inexpensive methods for training alignment models solely using free text, by generating artificial question-answer pairs from discourse structures. Our approach is driven by two representations of discourse: a shallow sequential representation, and a deep one based on Rhetorical Structure Theory. 
We evaluate the proposed model on two corpora from different genres and domains: one from Yahoo! Answers and one from the biology domain, and two types of non-factoid questions: manner and reason. We show that these alignment models trained directly from discourse structures imposed on free text
improve performance considerably over an information retrieval baseline and a neural network language model trained on the same data.
%
%imposing discourse structure on free text allows high-quality alignment models to be inexpensively trained, and that these models can improve performance up to 49\% (relative) over strong information retrieval and neural network language model baselines. 

\end{abstract}


\section{Introduction}
\vspace{-2mm}

 Question Answering (QA) is a challenging task that draws upon many aspects of NLP.  Unlike search or information retrieval, answers infrequently contain lexical overlap with the question (e.g. {\em What should we eat for breakfast? -- Zoe's Diner has good pancakes}), and require QA models to draw upon more complex methods to bridge this "lexical chasm" \cite{Berger:00}.  These methods range from robust shallow models based on lexical semantics, to deeper, explainably-correct, but much more brittle inference methods based on first order logic.  

Berger et al.~\citeyear{Berger:00} proposed that this "lexical chasm" might be partially bridged by repurposing statistical machine translation (SMT) models for QA. Instead of translating text from one language to another, these monolingual alignment models learn to translate from question to answer\footnote{In practice, alignment for QA is often done from answer to question, as answers tend to be longer and provide more opportunity for association~\cite{Surdeanu:11}.}, learning common associations from question terms such as {\em eat} or {\em breakfast} to answer terms like {\em kitchen, pancakes, or cereal}.

While monolingual alignment models have enjoyed a good deal of recent success in QA (see related work), they have expensive training data requirements,  
requiring a large set of aligned in-domain question-answer pairs for training.
% ms: this footnote dilutes the message: too specific for this discussion.
%\footnote{We have empirically observed that alignment models tend to generalize better between training and test folds when the alignment model is trained on its own fold, further increasing the number of high-quality QA pairs required.}.  
%In most domains these pairs are expensive to generate, and one of the current methodological challenges in QA is locating or building high-quality QA pairs for training and testing. Even large open-domain international evaluations and workshops such as the Text REtrieval Conference (TREC)\footnote{\url{http://trec.nist.gov}} and the Cross Language Evaluation Forum (CLEF),\footnote{\url{http://www.clef-initiative.eu}} are often limited to sets of a few hundred factoid questions, many of which are highly related.  As a result, for open domain QA one often makes use of Community Question Answering (CQA) data from websites such as Yahoo! Answers or Stack Overflow, which offer tens of thousands of questions, but of highly variable quality.  
For low-resource languages or specialized domains like science or biology, often the only option is to enlist a domain expert to generate gold QA pairs --  a process that is both expensive and time consuming.  All of this means that only in rare cases are we accorded the luxury of having enough high-quality QA pairs to properly train an alignment model, and so these models are often underutilized or left struggling for resources. 

Making use of recent advancements in discourse parsing \cite{feng12}, here we address this issue, and investigate whether alignment models for QA can be trained from artificial question-answer pairs generated from discourse structures imposed on free text.
% by imposing structure on inexpensive free text resources instead of using QA pairs.  
We evaluate our methods on two corpora, generating alignment models for an open-domain  community QA task using Gigaword\footnote{LDC catalog number LDC2012T21}, and for a biology-domain QA task using a biology textbook. 

The contributions of this work are:
\begin{enumerate}
\vspace{-3mm}
\item We demonstrate that by exploiting the discourse structure of free text, monolingual alignment models can be trained to surpass the performance of models built from expensive in-domain question-answer pairs. 
\vspace{-3mm}
\item We compare two methods of discourse parsing: a simple sequential model, and a deep model based on Rhetorical Structure Theory (RST)~\cite{mann88}.  We show that the RST-based method captures within and across-sentence alignments and performs better than the sequential model, but the sequential model is an acceptable approximation when a discourse parser is not available.  
\vspace{-3mm}
\item We evaluate the proposed methods on two corpora, including a low-resource domain where training data is expensive (biology).
\vspace{-3mm}
\item We experimentally demonstrate that monolingual alignment models trained using our method considerably outperform state-of-the-art neural network language models in low resource domains.
\end{enumerate}










%the task of question answering (QA) has received considerable attention. However, most of this effort has focused on factoid questions rather than more complex non-factoid (NF) questions, such as manner, reason, or causation questions. Moreover, the vast majority of QA models explore only 
%%similarity models based on 
%local linguistic structures, such as syntactic dependencies or semantic role frames, which are generally restricted to individual sentences. This is problematic for NF QA, where questions are answered 
% not by atomic facts, but 
%by larger cross-sentence conceptual structures that convey the desired answers. Thus, to answer NF questions, one needs a model of what these answer structures look like.
%
%Driven by this observation, our main hypothesis is that the discourse structure of NF answers provides complementary information to state-of-the-art QA models that measure the similarity (either lexical and/or semantic) between question and answer. 
%We propose a novel answer reranking (AR) model that combines lexical semantics (LS) with discourse information, driven by two representations of discourse: a shallow representation centered around discourse markers and surface text information, and a deep one based on the Rhetorical Structure Theory (RST) discourse framework~\cite{mann88}.
%To the best of our knowledge, this work is the first to systematically explore within- and cross-sentence structured discourse features for NF AR. The contributions of this work are:
%\begin{enumerate}
%\vspace{-3mm}
%\item We demonstrate that modeling discourse is greatly beneficial for NF AR for two types of NF questions, manner ({\em ``how"}) and reason ({\em ``why"}), across two large datasets from different genres and domains -- one from the community question-answering (CQA) site of Yahoo! Answers\footnote{\url{http://answers.yahoo.com}}, and one from a biology textbook.  
%%Our results show statistically significant improvements of over 20\%, up to 37\% (relative) on precision at 1 (P@1) when discourse is considered. Crucially, these improvements hold even on top of state-of-the-art LS models~\cite{yih13}.
%Our results show statistically significant improvements of up to 24\% on top of state-of-the-art LS models~\cite{yih13}.
%\vspace{-3mm}
%\item We demonstrate that both shallow and deep discourse representations are useful, and, in general, their combination performs best.
%\vspace{-3mm}
%\item We show that discourse-based QA models using inter-sentence features considerably outperform single-sentence models when answers span multiple sentences.
%\vspace{-3mm}
%\item We demonstrate good domain transfer performance between these corpora, suggesting that answer discourse structures are largely independent of domain, and thus broadly applicable to NF QA. 
%\end{enumerate}
%

\chapter{RELATED WORK\label{chapter:related_work}}

In one sense, QA systems can be described in terms of their position along a formality continuum ranging from shallow models that rely on information retrieval, lexical semantics, or alignment, to highly structured models based on first order logic (as depicted in Figure \todo{make a figure for intro}).

On the shallower end of the spectrum,  QA models can be constructed either from structured text, such as question--answer pairs, or unstructured text.  These models are largely based on finding patterns in word association scores, which serve as a proxy for formal inference.  For example, using one of the questions from Table \ref{tab:inferenceexamples}, \emph{Q: Which example describes an organism taking in nutrients? A: A girl eating an apple}, rather than knowing specifically that \emph{girl} is an organism and \emph{eating} is a manner of taking in nutrients, the model would look for a higher association between the question words \emph{organism, taking} and \emph{nutrients} and the correct answer words \emph{girl, eating} and \emph{apple} than between those same question words and words from an incorrect answer, such as \emph{insect, crawling} and \emph{leaf}.  Understandably, these methods struggle with more complex inference and with lure answers which are close associates with the correct answer. 

Within this category of association-based model, monolingual alignment models~\citep{Berger:00,echihabi2003noisy,Soricut:06,Riezler:etal:2007,Surdeanu:11,yao2013} have been widely employed.  These models utilize statistical machine translation techniques to learn \textit{translations} not from one language to another, but rather from question words to words likely to be found in their answers.  These alignment-based models, however, like their machine translation counterparts, require a large number of aligned text pairs for training.  In the case of the alignment models, this training data consists of aligned question--answer pairs, a burden which often limits their practical usage.  In Chapter \ref{chapter:naacl2015} we address this burden by proposing a method for using the discourse structure of free text as a surrogate for this alignment structure. %\todo{ref to a related work section in that chapter? or keep it here in a subsection?}

Lexical semantic models such as neural-network language models~\citep{jansen14,sultan-etal:2014:TACL,yih13}, on the other hand, have the advantage of being readily constructed from free text.  These models use distributional similarity via high-dimensional dense word embeddings to create sets of similarity features.  These features are intended to capture how related a question is to a given answer candidate, a proxy for likelihood of being correct.   
\citet{fried2015higher} called these approaches first-order models because associations are explicitly learned for words that co-occur in text.  In this way, if two words never co-occur, an association would never be \emph{directly} learned\footnote{While two words which never occur together would never have an association directly learned by the model, indirect learning takes place (which is, in a large part, what makes these models so robust).  That is, during the training process, words which occur with similar context words are indirectly drawn closer together in the high-dimensional embedded space as they are each independently drawn closer to their overlapping context words.  For example, words like \emph{dog} and \emph{platypus} will each be independently drawn closer to a common context word like \emph{eat}, inevitably resulting in them being indirectly drawn closer to each other.}.   To achieve a form of approximated inference based on chaining together associations, they introduced a higher-order lexical semantics QA model where indirect associations are detected through traversals of the association graph.  For example, the directly learned associations between word pairs like \textit{virus} $\leftrightarrow$ \textit{infection} and \textit{infection} $\leftrightarrow$ \textit{fever} would serve to form or strengthen the transitive association between \textit{virus} $\leftrightarrow$ \textit{fever}. 
%Other recent efforts have applied deep learning architectures to QA to learn non-linear answer scoring functions that model lexical semantics~\citep{Iyyer2014,nips15_hermann}.

These alignment and lexical semantic approaches, however do not take into account the wide variety of question types which often exist in any given question set.  Therefore, they attempt to answer \emph{all} questions with word-pair associations from the same set of standard (similarity-based) word embeddings.  To address this, we propose an approach in Chapter \ref{chapter:emnlp2016} for training a dedicated set of word embeddings that are customized to a particular semantic relation (here, causality) and demonstrate that the semantic information contained in these customized embeddings is complementary to that which is contained in standard word embeddings and useful for answering questions of the corresponding type (again, here we address \textit{causal} questions, though the method is readily extendable to other question types).

While these shallower (i.e., alignment and lexical semantic) approaches to QA have shown robust performance across a variety of tasks, one continuing disadvantage of these methods is that, even when a correct answer is selected, there is no clear human-readable justification for that selection.  This limits our ability to effectively understand model performance and adjust for errors accordingly.

Closer to the other end of the formality continuum, several approaches were proposed to not only select a correct answer, but also provide a formally valid justification for that answer.  For example, some QA systems have sought to answer questions by creating formal proofs driven by logic reasoning over sets of semantic rules~\citep[e.g.,][]{moldovan2003cogex,moldovan2007cogex,
balduccini2008knowledge,
maccartney2009natural,liang2013learning,
lewis2013combining}.
Some formal systems have made use of answer-set programming \citep{baral2006using,baral2011towards,baral2012answering,
baral2012knowledge} to computationally search through a set of answers to find one for which a formal proof (from question to answer) can be constructed. 
Still others have constructed semantic graph representations (based on semantic role information) of sentences and questions, attempting to resolve missing roles to find the answer~\citep[e.g.,][]{banarescu2012amr,sharmatowards}. 
However, the formal representations used in these systems, e.g., logic forms or semantic graphs, are both expensive to generate and tend to be brittle because they rely extensively on imperfect tools for unsolved problems (such as complete syntactic analysis and word sense disambiguation) as well as incomplete knowledge bases and rule sets.  %As a practical example, the semantic graphs rely on databases of the semantic roles of known verbs.  The coverage of these databases, however, is limited both in domain as well as language and so the systems that build upon them are as well. 

In Chapter \ref{chapter:cl2017}, we offer a directed graph representation for sentences (based on a simplification of the sentence syntax, see Section \ref{sec-cl2017:tag}) as a shallower and consequently more robust approximation of those logical forms. 
%a lightly-structured graph-based sentence representation (see Section \ref{sec-cl2017:tag}) as a shallower and consequently more robust approximation of those logical forms.  
In Section \ref{sec-cl2017:results} we show that they are well-suited for the complex questions (see Section \ref{sec:mcqa}) we tackle.
This approach allows us to aggregate information from a variety of knowledge sources to create complete, human-readable answer justifications.  
It is these justifications which we then rank in order to choose the correct answer, using a reranking perceptron extended to be able to learn to rank justifications and answers simultaneously despite having supervision only for which answer is correct.
%with a latent layer that models the correctness of those justifications.

While this approach is shallower than many of the approaches based on formal representations, it still requires decomposing free text resources into lightly-structured representations\footnote{The representations are directed graphs, and so have structure, but they are relatively lightweight -- a simplification of the sentences' dependency syntax.} and performing a somewhat expensive aggregation step.  In practice, this limits its use with extremely large text corpora.  
%Additionally, as the learning framework is an extension of a linear perceptron, it is inherently limited by its assumption of the data being linearly-separable within the feature space.  
To address this limitation, in Chapter \ref{chapter:emnlp2017} we propose a shallower version of this model that eliminates the aggregation and uses a neural network architecture to learn a task-specific embedded representation of each of the question, answer, and candidate justification texts.  These representations are then used alongside a small set of explicit features to re-rank the candidate justifications, given the question and answer.  By removing the graph-based representation, we increase the robustness, allowing us to use the model with much larger text corpora, and consequently in the domains that require larger them.   Additionally, by continuing to re-rank the candidate justifications, we maintain the model interpretability -- since the top $n$ justifications returned for each chosen answer provide insight into what the model learned was important for finding correct answers.

In both of these latter approaches, the way we have formulated our justification selection (as a re-ranking of knowledge base sentences) is related to, but yet distinct from the task of answer sentence selection \cite[][inter alia]{Wang2010ProbabilisticTM, Severyn:12,Severyn:13a,Severyn:13b,Severyn2015LearningTR,wang2015long}.  Answer sentence selection is typically framed as a fully or semi-supervised task for factoid questions (i.e., questions whose answers are limited to one or more facts, such as birth-dates or names), where the goal of the task is to correctly select a sentence that from a corpus that fully contains the answer text.  For example, to answer the question \textit{What is the capital of France?}, a system would attempt to select a sentence such as \textit{In the French capital of Paris, ...}.
Our QA task, however, is unlike answer-sentence selection.  Here, we have a variety of questions, many of which are non-factoid (i.e., questions whose answers are not facts, such as those based on \textit{how} or \textit{why} something happens).  Additionally, we have no direct supervision for our justification selection (i.e., no labels as to which sentences are good justifications for our answers), motivating our distant supervision approach where the performance on our QA task serves as supervision for selecting good justifications.  Further, we are not actually looking for sentences that \emph{contain} the answer choice, as with answer sentence selection, but rather sentences which close the "lexical chasm" \citep{Berger:00} between question and answer.  This distinction is demonstrated in the example in Table \ref{tab:question_example}, where the correct answer does not overlap lexically with the question and only minimally with the justification.  Instead, the justification serves as a bridge between the question and answer, filling in the missing information for the required inference. 
%(demonstrated in the example in Table \ref{tab:question_example}).  That is, we are seeking as justifications sentences which provide the missing step in the \emph{inference} needed to answer the question.


\begin{figure}[t]
\begin{center}
\includegraphics[width=0.3\textwidth]{arch_overall.png}
\caption{ Architecture of our question answering approach.  
Given a question, candidate answer, and a free-text knowledge base as inputs, we generate a pool of candidate justifications, from which we extract feature vectors.  We use a neural network to score each and then use max-pooling to select the current best justification. This serves as the score for the candidate answer itself.  The red border indicates the components that are trained online. }
\label{fig:arch_overall}
\vspace{-5mm}
\end{center}
\end{figure}

\section{Approach}
\label{sec:approach}
One of the primary difficulties with the explainable QA task addressed here is that, while we have supervision for the correct answer, we do not have annotated answer justifications.  
Here we tackle this challenge by using the QA task performance as supervision for the justification reranking, allowing us to 
%extending a neural QA model to jointly learn both how to 
%jointly 
learn to choose both the correct answer and a compelling, human-readable justification for that answer.

Additionally, similar to the strategy Chen and Manning~\shortcite{chen2014fast} applied to parsing, we combine representation-based features with explicit features that capture additional information that is difficult to model through embeddings, especially with limited training data.
%second contribution is that, similar to the strategy Chen and Manning~\shortcite{chen2014fast} applied to parsing, we combine representation-based features with explicit features that capture additional information that is difficult to model through embeddings.



% ms: avoid "system" too engineering-y
The architecture of our approach is summarized in Figure \ref{fig:arch_overall}.  
Given a question and a candidate answer, we first query an textual knowledge base (KB) to retrieve a pool of potential justifications for that answer candidate.  
For each justification, we extract a set of features designed to model the relations between questions, answers, and answer justifications based on word embeddings, lexical overlap with the question and answer candidate, discourse, and information retrieval (IR) (Section \ref{sec:features}).
These features are passed into a simple neural network to generate a score for each justification, given the current state of the model.  A final max-pooling layer selects the top-scoring justification for the candidate answer and this max score is used also as the score for the answer candidate.  
The system is trained using correct-incorrect answer pairs with a pairwise margin ranking loss objective function to enforce that the correct answer be ranked higher than any of the incorrect answers. 

%The key here is that we use the current state of the model to select the best justification for a given answer candidate from a pool of many candidate justifications.  To do this, we modify the training procedure such that at the start of each epoch \todo{minibatch instead of epoch?}, we first compute a forward pass with each candidate justification to find the top-scoring justification for each candidate answer.
%For a given question, answer candidate, and justification, we combine features based on word embeddings, lexical overlap, discourse, and information retrieval (IR) together in a simple neural architecture to generate a score for the answer candidate.    We then use this selected justification to calculate our gradients for updating the model parameters.  

With this end-to-end approach, the model learns to select justifications that allow it to correctly answer questions.  We hypothesize that this approach enables the model to indirectly learn to choose justifications that provide good explanations as to why the answer is correct. We empirically test this hypothesis in Section \ref{sec:results}, where we show that indeed the model learns to correctly answer questions, as well as to select high-quality justifications for those answers. 
% ms: misleading; it reads as if answer selection is better than IR
% better than a strong IR baseline. 



%----------------------TABLE FORMAT FROM TACL 2015--------

\begin{table*}[h]{}

\begin{scriptsize}

        \centering
        %\resizebox{7.9cm}{!}{
        {
        %\begin{tabular}{p{1cm}|p{3cm}|l}
        \begin{tabular}{p{0.3cm}|p{5cm}|p{8cm}}
        \hspace*{-7pt}  & Feature Group & Feature Descriptions \\
        \toprule
        \multirow{10}{*}
        {\rotatebox[origin=c]{90}{~~~~~~~~~Alignment Models}}
        & { Global Alignment Probability } & p(Q$|$A) according to IBM Model 1 \cite{Brown:93}\\ 
        & {} & {}\\
        & { Jenson-Shannon Distance (JSD) } & {Pairwise JSDs were found between the probability distribution of each content word in the question and those in the answer.  The \textbf{mean, minimum, and maximum JSD values} were used as features. Additionally, composite vectors were formed which represented the entire question and the entire answer and the \textbf{overall JSD} between these two vectors was also included as a feature. See Fried et. al \citeyear{fried2015higher} for additional details.} \\
        \midrule
        \multirow{10}{*}
        {\rotatebox[origin=c]{90}{~~~~~~~~~~~~~~~~~~~~~~~~~~~~~~~~~~RNNLM}}
        & { Cosine Similarity } & {Similar to Jansen et al.~\citeyear{jansen14}, we include as features the {\bf maximum and average pairwise cosine similarity} between question and answer words, as well as the {\bf overall similarity} between the composite question and answer vectors.} \\
        \bottomrule
        \end{tabular}
        }
        %}

\end{scriptsize}

        %\caption{$10$ most important features of each sieve.}
        \caption{Feature descriptions for alignment models and RNNLM baseline.}
        \label{tab:Features}
	\vspace{-6mm}

\end{table*}

%------------------END COPIED TABLE----------------
%space{-1mm}
\section{Models and Features}
\label{sec-naacl2015:models}

We evaluate the contribution of these alignment models using a standard reranking architecture~\cite{jansen14}.
The initial ranking of candidate answers is done using a shallow candidate retrieval (CR) component.\footnote{We use the same cosine similarity between question and answer lemmas as Jansen et al.~\citeyear{jansen14}, weighted using {\em tf.idf}.} % (see Ch. 6, \cite{manning08}).}  
Then, these answers are reranked using a more expressive model that incorporates alignment features alongside the CR score.  As a learning framework we use \svmr , a Support Vector Machine tailored for ranking.\footnote{ \url{http://www.cs.cornell.edu/people/tj/svm_light/svm_rank.html}}
We compare this alignment-based reranking model against one that uses a state-of-the-art recurrent neural network language model (RNNLM)~\cite{mikolov10,mikolov13}, which has been successfully applied to QA previously~\cite{yih13}.


%IBM Model 1 \cite{Brown:93} \todo{GIZA++?} was used to generate the discourse pair matrices described in section \ref{sec-naacl2015:approach} for the alignment models.  Each of these models, along with a Recurrent Neural Network Language Model (RNNLM), was used to rerank candidate answers in a shallow retrieval model which used \svmr , a Support vector Machine tailored for ranking.\footnote{ \url{http://www.cs.cornell.edu/people/tj/svm_light/svm_rank.html}}  
%(See Jansen~\citeyear{jansen14} for detailed architecture.\footnote{We also use his \svmr c of 0.1.}) 

{\flushleft {\bf Alignment Model:}}  The alignment matrices were generated with IBM Model 1 \cite{Brown:93} using GIZA++~\cite{och03}, and the corresponding models were implemented as per Surdeanu et al.~\citeyear{Surdeanu:11} with a global alignment probability. 
%, computing a {\bf global alignment probability}, which is the conditional probability of observing a question given an answer.
% ms: the global prob. operates for the whole Q and A!
%question word given an answer word.
%~\footnote{Within Surdeanu et al.'s~\citeyear{Surdeanu:11} framework, we lightly tuned the smoothing parameter, $\lambda$, to 0.4 and we redistributed the probability mass for each word such that the probability of a word translating to itself was 0.5.}  
We extend this alignment model with features from Fried et al.~\citeyear{fried2015higher} that treat each (source) word's probability distribution (over destination words) in the alignment matrix as a distributed semantic representation, and make use the Jensen-Shannon distance (JSD)\footnote{Jensen-Shannon distance is based on Kullback-Liebler divergence but is a distance metric (finite and symmetric).} between these conditional distributions.  A summary of all these features is shown in Table \ref{tab:Features}.% to derive features representing the {\bf minimum, maximum, and average JSDs} between question and answer words, as well as the {\bf overall JSD} between the composite question and answer distributions. 
%Using the Jensen-Shannon distance (JSD) between conditional distributions, content words in each question were compared with the content words in its answers.  Features were made from the maximum, minimum, and average JSDs.  Composite distributions were also created for the entire question and for each entire candidate answer, and the JSD between these composite distributions was used as a feature. 


{\flushleft {\bf RNNLM:}} We learned word embeddings using the \texttt{word2vec} RNNLM of Mikolov et al. \citeyear{mikolov13}, and include the cosine similarity-based features described in Table \ref{tab:Features}. %Similar to Jansen et al.~\citeyear{jansen14}, we include as features the {\bf maximum and average pairwise cosine similarity} between question and answer words, as well as the {\bf overall similarity} between the composite question and answer vectors. 

%{\flushleft {\bf Hyperparameters}}We used the following very lightly tuned hyperparameters: for \svmr, we used c = 0.1 and for IBM model 1, lambda = 0.4 and self-association of 0.5. \todo{re-write this!! or spread to appropriate places} 


%\section{Experiments and Evaluation}
%\label{sec:experiments}
%%\vspace{-2mm}
%
%We evaluate our approach at two levels.  First, we directly evaluate the quality of the embedding space in regards to the desired semantic relation.  Then, we evaluate the utility of the approach in an open-domain QA task.

\section{Direct Evaluation}
\label{sec:directeval}

To determine whether or not the proposed approach is able to capture the semantic relation of interest, causality, we replicate the quantitative evaluation of Levy and Goldberg~\citeyear{levy2014dependency}.  To demonstrate that their embeddings encoded functional similarity as opposed to topical similarity, they used their embeddings to rank a set of word pairs (each of which reflected one of the two types of similarity) using cosine similarity and showed that the word pairs which were functionally similar tended to be ranked higher than those with topical similarity.  Here, we do the same.

%\flushleft{\textbf{Data:}} 
\subsection{Data:}
We evaluate on a set of word pairs drawn from the SemEval 2010 Task 8 \todo{cite}, a multiway classification of semantic relations between nominals.  From the training set of this task, we used a total of XX nominal pairs, XX of which were cause-effect and XX which were randomly selected from the other XX relations.  This set was then randomly divided into equally-sized development and test partitions.

%\flushleft{\textbf{Baselines:}} 
\subsection{Baselines:}
We compared our embeddings against a standard \texttt{word2vec} model with a sliding window of XX as well as a random baseline where pairs were randomly shuffled. Additionally, we compared against a look-up baseline which counted the number of times a given nominal pair was found in the extracted cause-effect event database. 

\begin{table}[t!]
\begin{center}
%\begin{scriptsize}
\begin{footnotesize}
\begin{tabular}{ll}
\hline
\multicolumn{1}{l}{ Model } & \multicolumn{1}{l}{MAP} \\ %\multicolumn{1}{l}{Impr.} \\
%\cline{1-2}

\hline
%\multicolumn{2}{l}{\textit{Yahoo! Answers}} \\ % 185q (sent) ret=1p c=0.1 
%\hline
Random 			& 48.9 	\\
Lookup			& 93.1 	\\
Standard  		& 58.4	\\
Casual  			& 68.2*	\\

\end{tabular}
\end{footnotesize}
\caption{{\small Ability of each of the models and baselines to rank causal nominal pairs above non-causal pairs, as measured with mean average precision (MAP). Statistical significance (indicated by *) was determined through bootstrap resampling with 10,000 iterations.}}
\label{tab:MAP}
\end{center}
\end{table}
%MAP for Custom Vectors: 0.6816675505751166
%MAP for E2C Vectors: 0.6871338630607531
%MAP for Bidir Vectors: 0.6684350003582593
%MAP for Comparison (Baseline) Vectors: 0.5835158858435683
%MAP for Translation Model with lamda of 0.5 : 0.6156522257806402
%MAP for counting Matches: 0.9312613087523468
%MAP for Keras: 0.6752727545259546
%MAP for Random: 0.4892479543525109
%p < 0.01

\begin{figure}[t!]
\begin{center}
%\includegraphics[width=75mm]{rpcurves_all.png}
%\vspace{-4mm}
\caption{{\small Recall-precision curve showing the ability of each model to rank causal pairs above non-causal pairs. }}
\vspace{-6mm}
\label{fig:rpcurve_all}
\end{center}
\end{figure}

%\flushleft{\textbf{Results:}} 
\subsection{Results:}
In Table \ref{tab:MAP} we report the mean average precision (MAP) for each of the models and baselines.  The highest by far is for the look-up baseline.  However, this score is positively skewed by the fact that when pairs are found, there is extremely high confidence that they are of the relation of interest coupled with the fact that approximately 65\% of the pairs were not found in the database and so were all tied in last place.  As a consequence, there were far fewer average precisions to be combined when calculating the MAP, and most of the ones which were there had extremely high precision.  The MAP when using the customized vectors was significantly higher than that of the standard \texttt{word2vec} vectors (68\% versus 58\%), and both were higher than the baseline.  This suggests that while the standard implementation of \texttt{word2vec} encodes some causality information, our method encodes it far more directly. \todo{better word}.

\subsection{Discussion:}
We also examined the recall-precision curves for all models, shown in Figure \ref{fig:rpcurve}.  The curve for the customized vectors shows an atypical shape, where, despite the success of the customized vectors once recall reaches approximately 15\%, the highest ranked pairs (i.e. lower recall) had far \emph{worse} precision rather than higher precision.  To examine this, we analyzed the top-ranked 15\% of the pairs from the causal vectors.  

\todo{HEDGE THIS - more?}
Rather than finding that the high associations were due to noise in the training data, we found that the model appeared to be performing approximate inference, but here it was noisy inference.  For example, the top-ranked pair was (\emph{platform}, \emph{scaffold}) and there were no instances in the extracted causal events where \emph{platform} and \emph{scaffold} were found together in the same cause-effect pair.  

Instead, we found that there were only three extracted events with \emph{scaffold} in the effect text.  We checked to see if the words in the cause text of those events had other effect texts which overlapped lexically with \emph{platform}.  Indeed, we found the link illustrated in Figure \ref{fig:noisyinf}, where both \emph{platform} and \emph{malfunction} cause \emph{loss} thus bringing them closer together in the target embedding space, since they cause the same things.  This resulted in \emph{platform} being close to the effects of \emph{malfunction}, including \emph{scaffold}.  This demonstrates that the inference is influenced by frequency effects, as words like \emph{scaffold} and \emph{platform} are too infrequent to have robust representations in the embedding space.  

As this effect is entirely directional, we trained a set of vectors by reversing the input, such that the effects served as the targets and the causes were the contexts.  The recall-precision curve when using these embeddings to rank the SemEval pairs is shown in Figure \ref{fig:rpcurve_all}, labelled as E2C.  As expected, the curve follows that of the original vectors, as it suffers from the same issues, but with different noisy pairs ranked highly.  We then ranked the SemEval pairs using an average of the scores returned by the two causal embeddings, to mitigate the frequency effects.  That final, bidirectional curve is also shown in Figure \ref{fig:rpcurve_all}.
%Instead, we found that many of the causal events which whose cause argument contained the word \emph{platform} had effects which overlapped lexically with those whose cause arguments contained the word \emph{malfunction}, and that there w.  To illustrate, consider the following sentences 
%we had several extracted causal events whose cause contained the word \emph{platform} and whose effect contained the words \emph{}

\section{Indirect Evaluation}
\label{sec:indirecteval}

After having determined that the learned embeddings encoded causality, we designed an experiment to evaluate their utility in a down-stream task.  We chose question-answering (QA) as it has been shown that many questions require complex inference, including causality, in order to be explanably solved \todo{cite}.  

\subsection{Data:}
As we trained our embeddings using events extracted from open-domain resources, we chose to evaluate on a set of causal questions from Yahoo! Answers \todo{link}, hand-extracted with very simple surface patterns such as \emph{What causes ...}\todo{footnote with details?}.  There were a total of \todo{XX} questions, from which we used \todo{50\%} for training, \todo{25\%} for development, and \todo{25\%} for testing.
These questions were filtered such that each had at least four candidate answers
\todo{detail about the avg num answers, etc}

\subsection{Experimental Design:}
We evaluate the contribution of the causal embeddings model using the standard reranking architecture \todo{cite} of \todo{cite naacl and peters acl}.
%In pilot experiments, we found that aligning only nouns, verbs, adjectives, and adverbs yielded higher performance. TALK ABOUT?
In this architecture, the candidate answers are initially ranked using a shallow candidate retrieval (CR) component, then they are re-ranked using the model features along with the CR score. 
We used SVM rank\todo{cite/link}, a Support Vector Machine adapted for ranking, for our learning framework.
We compare the performance of the causal embeddings model against the same standard skipgram model used in Section \ref{sec:directeval}.
As features, we use the same set as \todo{cite naacl and acl}: the maximum and average pairwise cosine similarity between question and answer words, as well as the overall similarity between the composite question and answer vectors.  When using the causal embeddings however, we first determine (through lexical pattern matching) whether the question text is the cause or the effect, thus determining which embeddings to use for the question and candidate answer texts.  For example, in a question such as "Q: What causes X? A: Y", the cosine similarities would be found using the effect vectors for the question words and the cause vectors for the answer candidate words. \todo{good grief, redo this}   


%\section{Discussion}
\label{sec-cl2017:discussion}

To further characterize the performance of our QA approach, we address the following questions: 


%%
%% Comparison with TableILP
%%
%\begin{table}[t]
%\caption{{ Comparison with other models }} % 
%\small
%\begin{center}
%\begin{tabular}{lccc}
%%\cline{2-3}
%%\begin{tabular}{p{20mm}cc}
%\hline
%\multicolumn{1}{c}{Model} & \multicolumn{1}{c}{Questions} &\multicolumn{1}{c}{P@1} \\
%\cline{2-3}
%\hline
%CR $\cup$ (1G\textsubscript{CT} + 2G\textsubscript{CT}) $\cup$ 3G\textsubscript{CT} 			 	& 1000		&	44.6\%	\\
%Khashabi et. al (2016)																				& 200 		&	45.6\%	\\
%
%\end{tabular}
%\label{tab:tableilp}
%\end{center}
%\end{table}

{\flushleft {\bf How does performance compare with methods using manually constructed knowledge bases?}} The TAG system automatically aggregates sentences from six free-text corpora first by building graphlets from those sentences using syntactic dependencies, then connecting those graphlets together into multisentence text aggregation graphs that are then used to both answer questions and provide a compelling human-readable justification for the selected answers.  Recently, Khashabi et al. \citeyear{Khashabi2016QuestionAV} demonstrated that graphs for elementary science QA can also be constructed using a semistructured knowledge base of tables.  In this formalism, dozens of themed tables are manually or semi-automatically constructed, each around a particular theme.  A table's theme is encoded in it's columns, i.e., a table for the color of objects contains two rows, one for the object of interest (e.g., ``leaf''), and another for it's color (e.g., ``green''), while separate instances (e.g., leaf -- green, trunk -- brown) are encoded as different rows.  Each table has between two and five columns.  

The TableILP algorithm answers questions by chaining facts between different table rows, starting from a row that contains question terms, then traversing to a new table row that contains some lexical overlap with the previous row(s), until answer terms are found.  The TAG and TableILP systems are conceptually similar, with the central differences being: (1) The TableILP table row is roughly equivalent to a TAG graphlet with flat structure, and limited to 2--5 information nuggets containing only single terms, (2) TAG graphlets are read automatically from free text corpora, where TableILP tables are largely manually constructed, with methods to automate this construction being actively developed, and (3) the traversal algorithms are different, with TableILP graph building being modelled as an integer linear programming (ILP) problem which finds paths that maximize QA performance.  

The TableILP system reported by Khashabi et al. \citeyear{Khashabi2016QuestionAV} contains 69 tables containing a total of 7,600 rows, with 64 of these tables (approximately 5,000 rows) designed around material in the study guides and a development corpus, and the remaining 2,600 rows distributed amoung 4 automatically constructed tables.  On a corpus of 200 questions drawn from the 1,000 questions used here, TableILP achieved a score of 45.6\% P@1\footnote{We wish to thank Khashabi et al. \citeyear{Khashabi2016QuestionAV} for providing us with this performance figure.}, compared to the 44.6\% P@1 from the best-performing TAG model in Table~\ref{tab:combinedmodels}. The performance difference between these systems is likely not statistically significant.\footnote{We did not have access to system output. However, in our experiments on this dataset, we observed that only differences in P@1 scores of 3\% or higher (absolute) tend to be statistically significant at $p < 0.05$.}

We view these systems as complementary, converging, and with each capable of exploring different aspects of graph-based inference for science QA.  While the TAG focuses on automatically building graphs from free text, this is currently a challenging and noisy process, and as we have shown in Table~\ref{tab:pathlength} and Fried et al.~\citeyear{fried2015higher}, highly susceptable to inference drift as the amount of information required to be aggregated becomes large.  On the other hand, building graphs from manually constructed knowledge bases allow us to investigate the graph-building process in isolation, reducing inference drift due to noise, and further moving this area forward.  

%
% Performance by grade level
%
\begin{table}[t]
\caption{{ Precision@1 by grade level. }} % 
\small
\begin{center}
\begin{tabular}{lccc}
%\cline{2-3}
%\begin{tabular}{p{20mm}cc}
\hline
\multicolumn{1}{c}{Grade Level} & \multicolumn{1}{c}{Questions} &\multicolumn{1}{c}{CR} & \multicolumn{1}{c}{TAG}  \\
\cline{2-3}
\hline
Grade 3 	& 60		&	28.3\%		& 49.2\%	\\
Grade 4		& 69 	&	50.7\%		& 41.3\%	\\
Grade 5		& 871	&	40.2\%		& 42.6\%	\\

\end{tabular}
\label{tab:gradelevel}
\end{center}
\end{table}


{\flushleft {\bf How does performance vary by grade level?}} The question corpus contains third, fourth, and fifth grade questions.  A human with a level of knowledge equivalent to a fourth grade science student might be expected to show better performance for the simpler third grade questions, and decreasing performance as question difficulty increases from fourth to fifth grades.  Table~\ref{tab:gradelevel} shows P@1 performance by grade level for both the CR and best performing TAG model (1G\textsubscript{CT} + 2G\textsubscript{CT}).  The TAG model shows decreasing performance as question difficulty increases, dropping from 49\% for third grade questions to 42\% for fourth and fifth grade questions.  The CR baseline, however, displays a qualitatively different pattern, with a peak performance of 51\% for fourth grade questions, and {\em near chance} performance for third grade questions. 
We believe that observing such a pattern in performance may suggest that the TAG model is a closer approximation of human inference than the baseline based solely on information retrieval.  Here, the relatively small number of third and fourth grade questions prevents us from drawing any conclusions, but suggests that crafting question sets to allow evaluating the distribution of performance by grade level may provide a further measure of comparison between human and machine performance. 



%
% Justification performance by knowledge resource
%
\begin{table}[t]
\caption{{ Most useful knowledge resources for justifications classified as "good".}}
\small
\begin{center}
\begin{tabular}{lccc}
%\cline{2-3}
%\begin{tabular}{p{20mm}cc}
\hline
\multicolumn{1}{l}{Resource} & \multicolumn{1}{c}{Sentences} &\multicolumn{1}{c}{CR} & \multicolumn{1}{c}{TAG}  \\
\cline{2-3}
\hline
Barrons SG 			& 1,200		&	39.3\%		& 43.0\%	\\
Flashcards			& 283		&	16.2\%		& 8.2\%	\\
Teacher's Guide		& 302		&	7.1\%		& 7.0\%	\\
Virginia SG			& 1,314		&	9.1\%		& 9.2\%	\\
Science Dictionary	& 733		&	20.8\%		& 17.8\%	\\
Simple Wiktionary	& 17,473		&	7.5\%		& 14.8\%	\\

\end{tabular}

\label{tab:justificationknowledgeresources}
\end{center}
\end{table}

{\flushleft {\bf Which knowledge resources are generating the most useful answer justifications?}} Shown in Table~\ref{tab:justificationknowledgeresources}, the Barron's Study Guide (SG) contributes more of the \emph{good} justification sentences than any other source, followed by the science dictionary, then the other resources.  Interestingly, the Simple Wicktionary contributes the fewest sentences to the \emph{good} justifications for the CR system (7.5\%), but for the TAG system it is the third largest contributor (14.8\%).  That is, while the CR system is typically unable to find a \emph{good} justification from the Wiktionary, likely owing to it's general nature, the TAG system is able to successfully aggregate these sentences with sentences from other domain-specific sources to build complete justifications.

The vast majority of the \emph{good} justifications generated by the TAG system are aggregates from non-adjacent text: 67\% of the justifications aggregate sentences from \emph{different} corpora, 30\% aggregate non-adjacent sentences within a \emph{single} corpus, while only 3\% of \emph{good} justifications contain sentences that were adjacent in their original corpus. 
This is clear evidence that information aggregation (or fusion) is fundamental for answer justification.


{\flushleft {\bf How orthogonal is the performance of the TAG model when compared to CR?}} Both the TAG and CR models use the same knowledge resources, which on the surface suggests the models may be similar, answering many of the same questions correctly.  The voting models in Table~\ref{tab:combinedmodels} appear to support this, where combining the TAG and CR models increases performance by just under 2\% P@1 over the best-performing TAG model.  To investigate this, we conducted an orthogonality analysis to determine the number of questions both models answer correctly, and the number of questions each model uniquely answers correctly.

Comparing the TAG (1G\textsubscript{CT} + 2G\textsubscript{CT}) and CR models, nearly half of questions are answered correctly by one model and incorrectly by the other.  When combined into a two-way voting model, this causes a large number of ties -- which, resolved at chance, would perform at 42\%, with ceiling performance (i.e., all ties resolved correctly) at 60\%.  This indicates that while the TAG and CR models share about half of their performance, each model is sensitive to different kinds of questions, suggesting that further combination strategies between TAG and CR are worth exploring.


%\section{Discussion}
\label{sec-cl2017:discussion}

To further characterize the performance of our QA approach, we address the following questions: 


%%
%% Comparison with TableILP
%%
%\begin{table}[t]
%\caption{{ Comparison with other models }} % 
%\small
%\begin{center}
%\begin{tabular}{lccc}
%%\cline{2-3}
%%\begin{tabular}{p{20mm}cc}
%\hline
%\multicolumn{1}{c}{Model} & \multicolumn{1}{c}{Questions} &\multicolumn{1}{c}{P@1} \\
%\cline{2-3}
%\hline
%CR $\cup$ (1G\textsubscript{CT} + 2G\textsubscript{CT}) $\cup$ 3G\textsubscript{CT} 			 	& 1000		&	44.6\%	\\
%Khashabi et. al (2016)																				& 200 		&	45.6\%	\\
%
%\end{tabular}
%\label{tab:tableilp}
%\end{center}
%\end{table}

{\flushleft {\bf How does performance compare with methods using manually constructed knowledge bases?}} The TAG system automatically aggregates sentences from six free-text corpora first by building graphlets from those sentences using syntactic dependencies, then connecting those graphlets together into multisentence text aggregation graphs that are then used to both answer questions and provide a compelling human-readable justification for the selected answers.  Recently, Khashabi et al. \citeyear{Khashabi2016QuestionAV} demonstrated that graphs for elementary science QA can also be constructed using a semistructured knowledge base of tables.  In this formalism, dozens of themed tables are manually or semi-automatically constructed, each around a particular theme.  A table's theme is encoded in it's columns, i.e., a table for the color of objects contains two rows, one for the object of interest (e.g., ``leaf''), and another for it's color (e.g., ``green''), while separate instances (e.g., leaf -- green, trunk -- brown) are encoded as different rows.  Each table has between two and five columns.  

The TableILP algorithm answers questions by chaining facts between different table rows, starting from a row that contains question terms, then traversing to a new table row that contains some lexical overlap with the previous row(s), until answer terms are found.  The TAG and TableILP systems are conceptually similar, with the central differences being: (1) The TableILP table row is roughly equivalent to a TAG graphlet with flat structure, and limited to 2--5 information nuggets containing only single terms, (2) TAG graphlets are read automatically from free text corpora, where TableILP tables are largely manually constructed, with methods to automate this construction being actively developed, and (3) the traversal algorithms are different, with TableILP graph building being modelled as an integer linear programming (ILP) problem which finds paths that maximize QA performance.  

The TableILP system reported by Khashabi et al. \citeyear{Khashabi2016QuestionAV} contains 69 tables containing a total of 7,600 rows, with 64 of these tables (approximately 5,000 rows) designed around material in the study guides and a development corpus, and the remaining 2,600 rows distributed amoung 4 automatically constructed tables.  On a corpus of 200 questions drawn from the 1,000 questions used here, TableILP achieved a score of 45.6\% P@1\footnote{We wish to thank Khashabi et al. \citeyear{Khashabi2016QuestionAV} for providing us with this performance figure.}, compared to the 44.6\% P@1 from the best-performing TAG model in Table~\ref{tab:combinedmodels}. The performance difference between these systems is likely not statistically significant.\footnote{We did not have access to system output. However, in our experiments on this dataset, we observed that only differences in P@1 scores of 3\% or higher (absolute) tend to be statistically significant at $p < 0.05$.}

We view these systems as complementary, converging, and with each capable of exploring different aspects of graph-based inference for science QA.  While the TAG focuses on automatically building graphs from free text, this is currently a challenging and noisy process, and as we have shown in Table~\ref{tab:pathlength} and Fried et al.~\citeyear{fried2015higher}, highly susceptable to inference drift as the amount of information required to be aggregated becomes large.  On the other hand, building graphs from manually constructed knowledge bases allow us to investigate the graph-building process in isolation, reducing inference drift due to noise, and further moving this area forward.  

%
% Performance by grade level
%
\begin{table}[t]
\caption{{ Precision@1 by grade level. }} % 
\small
\begin{center}
\begin{tabular}{lccc}
%\cline{2-3}
%\begin{tabular}{p{20mm}cc}
\hline
\multicolumn{1}{c}{Grade Level} & \multicolumn{1}{c}{Questions} &\multicolumn{1}{c}{CR} & \multicolumn{1}{c}{TAG}  \\
\cline{2-3}
\hline
Grade 3 	& 60		&	28.3\%		& 49.2\%	\\
Grade 4		& 69 	&	50.7\%		& 41.3\%	\\
Grade 5		& 871	&	40.2\%		& 42.6\%	\\

\end{tabular}
\label{tab:gradelevel}
\end{center}
\end{table}


{\flushleft {\bf How does performance vary by grade level?}} The question corpus contains third, fourth, and fifth grade questions.  A human with a level of knowledge equivalent to a fourth grade science student might be expected to show better performance for the simpler third grade questions, and decreasing performance as question difficulty increases from fourth to fifth grades.  Table~\ref{tab:gradelevel} shows P@1 performance by grade level for both the CR and best performing TAG model (1G\textsubscript{CT} + 2G\textsubscript{CT}).  The TAG model shows decreasing performance as question difficulty increases, dropping from 49\% for third grade questions to 42\% for fourth and fifth grade questions.  The CR baseline, however, displays a qualitatively different pattern, with a peak performance of 51\% for fourth grade questions, and {\em near chance} performance for third grade questions. 
We believe that observing such a pattern in performance may suggest that the TAG model is a closer approximation of human inference than the baseline based solely on information retrieval.  Here, the relatively small number of third and fourth grade questions prevents us from drawing any conclusions, but suggests that crafting question sets to allow evaluating the distribution of performance by grade level may provide a further measure of comparison between human and machine performance. 



%
% Justification performance by knowledge resource
%
\begin{table}[t]
\caption{{ Most useful knowledge resources for justifications classified as "good".}}
\small
\begin{center}
\begin{tabular}{lccc}
%\cline{2-3}
%\begin{tabular}{p{20mm}cc}
\hline
\multicolumn{1}{l}{Resource} & \multicolumn{1}{c}{Sentences} &\multicolumn{1}{c}{CR} & \multicolumn{1}{c}{TAG}  \\
\cline{2-3}
\hline
Barrons SG 			& 1,200		&	39.3\%		& 43.0\%	\\
Flashcards			& 283		&	16.2\%		& 8.2\%	\\
Teacher's Guide		& 302		&	7.1\%		& 7.0\%	\\
Virginia SG			& 1,314		&	9.1\%		& 9.2\%	\\
Science Dictionary	& 733		&	20.8\%		& 17.8\%	\\
Simple Wiktionary	& 17,473		&	7.5\%		& 14.8\%	\\

\end{tabular}

\label{tab:justificationknowledgeresources}
\end{center}
\end{table}

{\flushleft {\bf Which knowledge resources are generating the most useful answer justifications?}} Shown in Table~\ref{tab:justificationknowledgeresources}, the Barron's Study Guide (SG) contributes more of the \emph{good} justification sentences than any other source, followed by the science dictionary, then the other resources.  Interestingly, the Simple Wicktionary contributes the fewest sentences to the \emph{good} justifications for the CR system (7.5\%), but for the TAG system it is the third largest contributor (14.8\%).  That is, while the CR system is typically unable to find a \emph{good} justification from the Wiktionary, likely owing to it's general nature, the TAG system is able to successfully aggregate these sentences with sentences from other domain-specific sources to build complete justifications.

The vast majority of the \emph{good} justifications generated by the TAG system are aggregates from non-adjacent text: 67\% of the justifications aggregate sentences from \emph{different} corpora, 30\% aggregate non-adjacent sentences within a \emph{single} corpus, while only 3\% of \emph{good} justifications contain sentences that were adjacent in their original corpus. 
This is clear evidence that information aggregation (or fusion) is fundamental for answer justification.


{\flushleft {\bf How orthogonal is the performance of the TAG model when compared to CR?}} Both the TAG and CR models use the same knowledge resources, which on the surface suggests the models may be similar, answering many of the same questions correctly.  The voting models in Table~\ref{tab:combinedmodels} appear to support this, where combining the TAG and CR models increases performance by just under 2\% P@1 over the best-performing TAG model.  To investigate this, we conducted an orthogonality analysis to determine the number of questions both models answer correctly, and the number of questions each model uniquely answers correctly.

Comparing the TAG (1G\textsubscript{CT} + 2G\textsubscript{CT}) and CR models, nearly half of questions are answered correctly by one model and incorrectly by the other.  When combined into a two-way voting model, this causes a large number of ties -- which, resolved at chance, would perform at 42\%, with ceiling performance (i.e., all ties resolved correctly) at 60\%.  This indicates that while the TAG and CR models share about half of their performance, each model is sensitive to different kinds of questions, suggesting that further combination strategies between TAG and CR are worth exploring.



\vspace{-2mm}
%\begin{small}
\section*{Acknowledgments}
\vspace{-2mm}
We thank the Allen Institute for AI for funding this work.
%\end{small}

%\bibliographystyle{acl}
%\bibliography{citations} % add your references to citations, then run bibtex

% If you use BibTeX with a bib file named eacl2014.bib, 
% you should add the following two lines:
\bibliographystyle{acl}
\bibliography{acl2014}
%\bibliography{refs}
%\bibliography{citations}

\end{document}
