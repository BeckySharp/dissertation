
\section{Introduction}
\vspace{-2mm}

 Question Answering (QA) is a challenging task that draws upon many aspects of NLP.  Unlike search or information retrieval, answers infrequently contain lexical overlap with the question (e.g. {\em What should we eat for breakfast? -- Zoe's Diner has good pancakes}), and require QA models to draw upon more complex methods to bridge this "lexical chasm" \cite{Berger:00}.  These methods range from robust shallow models based on lexical semantics, to deeper, explainably-correct, but much more brittle inference methods based on first order logic.  

Berger et al.~\shortcite{Berger:00} proposed that this "lexical chasm" might be partially bridged by repurposing statistical machine translation (SMT) models for QA. Instead of translating text from one language to another, these monolingual alignment models learn to translate from question to answer\footnote{In practice, alignment for QA is often done from answer to question, as answers tend to be longer and provide more opportunity for association~\cite{Surdeanu:11}.}, learning common associations from question terms such as {\em eat} or {\em breakfast} to answer terms like {\em kitchen, pancakes, or cereal}.

While monolingual alignment models have enjoyed a good deal of recent success in QA (see related work), they have expensive training data requirements,  
requiring a large set of aligned in-domain question-answer pairs for training.
% ms: this footnote dilutes the message: too specific for this discussion.
%\footnote{We have empirically observed that alignment models tend to generalize better between training and test folds when the alignment model is trained on its own fold, further increasing the number of high-quality QA pairs required.}.  
%In most domains these pairs are expensive to generate, and one of the current methodological challenges in QA is locating or building high-quality QA pairs for training and testing. Even large open-domain international evaluations and workshops such as the Text REtrieval Conference (TREC)\footnote{\url{http://trec.nist.gov}} and the Cross Language Evaluation Forum (CLEF),\footnote{\url{http://www.clef-initiative.eu}} are often limited to sets of a few hundred factoid questions, many of which are highly related.  As a result, for open domain QA one often makes use of Community Question Answering (CQA) data from websites such as Yahoo! Answers or Stack Overflow, which offer tens of thousands of questions, but of highly variable quality.  
For low-resource languages or specialized domains like science or biology, often the only option is to enlist a domain expert to generate gold QA pairs --  a process that is both expensive and time consuming.  All of this means that only in rare cases are we accorded the luxury of having enough high-quality QA pairs to properly train an alignment model, and so these models are often underutilized or left struggling for resources. 

Making use of recent advancements in discourse parsing \cite{feng12}, here we address this issue, and investigate whether alignment models for QA can be trained from artificial question-answer pairs generated from discourse structures imposed on free text.
% by imposing structure on inexpensive free text resources instead of using QA pairs.  
We evaluate our methods on two corpora, generating alignment models for an open-domain  community QA task using Gigaword\footnote{LDC catalog number LDC2012T21}, and for a biology-domain QA task using a biology textbook. 

The contributions of this work are:
\begin{enumerate}
\vspace{-3mm}
\item We demonstrate that by exploiting the discourse structure of free text, monolingual alignment models can be trained to surpass the performance of models built from expensive in-domain question-answer pairs. 
\vspace{-3mm}
\item We compare two methods of discourse parsing: a simple sequential model, and a deep model based on Rhetorical Structure Theory (RST)~\cite{mann88}.  We show that the RST-based method captures within and across-sentence alignments and performs better than the sequential model, but the sequential model is an acceptable approximation when a discourse parser is not available.  
\vspace{-3mm}
\item We evaluate the proposed methods on two corpora, including a low-resource domain where training data is expensive (biology).
\vspace{-3mm}
\item We experimentally demonstrate that monolingual alignment models trained using our method considerably outperform state-of-the-art neural network language models in low resource domains.
\end{enumerate}










%the task of question answering (QA) has received considerable attention. However, most of this effort has focused on factoid questions rather than more complex non-factoid (NF) questions, such as manner, reason, or causation questions. Moreover, the vast majority of QA models explore only 
%%similarity models based on 
%local linguistic structures, such as syntactic dependencies or semantic role frames, which are generally restricted to individual sentences. This is problematic for NF QA, where questions are answered 
% not by atomic facts, but 
%by larger cross-sentence conceptual structures that convey the desired answers. Thus, to answer NF questions, one needs a model of what these answer structures look like.
%
%Driven by this observation, our main hypothesis is that the discourse structure of NF answers provides complementary information to state-of-the-art QA models that measure the similarity (either lexical and/or semantic) between question and answer. 
%We propose a novel answer reranking (AR) model that combines lexical semantics (LS) with discourse information, driven by two representations of discourse: a shallow representation centered around discourse markers and surface text information, and a deep one based on the Rhetorical Structure Theory (RST) discourse framework~\cite{mann88}.
%To the best of our knowledge, this work is the first to systematically explore within- and cross-sentence structured discourse features for NF AR. The contributions of this work are:
%\begin{enumerate}
%\vspace{-3mm}
%\item We demonstrate that modeling discourse is greatly beneficial for NF AR for two types of NF questions, manner ({\em ``how"}) and reason ({\em ``why"}), across two large datasets from different genres and domains -- one from the community question-answering (CQA) site of Yahoo! Answers\footnote{\url{http://answers.yahoo.com}}, and one from a biology textbook.  
%%Our results show statistically significant improvements of over 20\%, up to 37\% (relative) on precision at 1 (P@1) when discourse is considered. Crucially, these improvements hold even on top of state-of-the-art LS models~\cite{yih13}.
%Our results show statistically significant improvements of up to 24\% on top of state-of-the-art LS models~\cite{yih13}.
%\vspace{-3mm}
%\item We demonstrate that both shallow and deep discourse representations are useful, and, in general, their combination performs best.
%\vspace{-3mm}
%\item We show that discourse-based QA models using inter-sentence features considerably outperform single-sentence models when answers span multiple sentences.
%\vspace{-3mm}
%\item We demonstrate good domain transfer performance between these corpora, suggesting that answer discourse structures are largely independent of domain, and thus broadly applicable to NF QA. 
%\end{enumerate}
%
