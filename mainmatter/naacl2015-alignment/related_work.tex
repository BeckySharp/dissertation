
\section{Related Work}
\label{sec:related work}
%\vspace{-2mm}


Lexical semantic models have shown promise in bridging Berger et al.'s \shortcite{Berger:00} "lexical chasm."  In general, these models can be classified into alignment models \cite{echihabi2003noisy,Soricut:06,Riezler:etal:2007,Surdeanu:11,yao2013} which require structured training data, and language models ~\cite{jansen14,sultan-etal:2014:TACL,yih13}, which operate over free text.  Here, we close this gap in resource availability by developing a method to train an alignment model over free text by making use of discourse structures. 

  
  
%  are focusing here on alignment models, which have shown great promise but also have previously been limited by availability of training data.  We address the need for larger amounts of high quality aligned pairs by investigating methods of imposing structure over free text.... Rhetorical Structure Theory (RST) discourse framework ~\cite{mann88}.

Discourse has been previously applied to QA to help identify answer candidates that contain explanatory text (e.g. Verberne et al. \shortcite{Verberne:2007}).
%conducted an initial analysis of using discourse features derived from Rhetorical Structure Theory (RST)~\cite{mann88} for answer candidate selection, and concluded that while discourse features appeared useful, automated discourse parsing tools were required to test the idea on a larger scale.  
Jansen et al. \shortcite{jansen14} proposed a reranking model that used both shallow and deep discourse features to identify answer structures in large answer collections across different tasks and genres.  Here we use discourse to impose structure on free text to create inexpensive knowledge resources for monolingual alignment. Our work is conceptually complementary to that of Jansen et al. -- where they explored largely unlexicalized discourse structures to identify explanatory text, we use discourse to learn lexicalized models for semantic similarity.

Our work is conceptually closest to that of Hickl et al. \shortcite{hickl2006recognizing}, who created artificially aligned pairs for textual entailment.  Taking advantage of the structure of news articles, wherein the first sentence tends to provide a broad summary of the article's contents, Hickl et al. aligned the first sentence of each article with its headline.  By making use of automated discourse parsing, here we go further and impose alignment structure over an entire text.


%Here, by imposing alignment structure using RST we are able to make use of an entire text instead of being limited to a single sentence.

